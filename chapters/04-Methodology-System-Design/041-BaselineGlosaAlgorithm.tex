\section{Baseline GLOSA Algorithm}
\label{sec:Baseline_Glosa_Algorithm}

The Baseline \ac{glosa} algorithm, which serves as the foundation for this research, is the enhanced implementation proposed by Lenz \cite{Lenz2024} and integrated into the official \ac{sumo} distribution. This section details the operational principles of this throughput-oriented controller, which will serve as the benchmark against which the eco-driving algorithm, presented in Section~\vref{sec:Proposed_Eco_Driving_GLOSA_Algorithm}, is evaluated. The logic described is directly derived from the core \ac{glosa} module within the \ac{sumo} source code.

\subsection{Vehicle Speed Factor and Model}
\label{sec:Glosa_Speed_Factor_Model}

Within \ac{sumo}, a vehicle's desired speed is determined by the interplay between the legal speed limit of the current road segment, \gls{vallowed}, and the vehicle's individual speed factor, \gls{sf}. The target velocity of the ego vehicle, \gls{vego}, is thus computed as:
\begin{equation}
    \gls{vego} = \gls{sf} \cdot \gls{vallowed}
\end{equation}
By default, this speed factor is initialized from a distribution, but the \ac{glosa} device dynamically adjusts it to synchronize the vehicle's arrival at an upcoming traffic light with a green phase. This adjustment is constrained by two factors: the resulting speed must not exceed the vehicle's own maximum speed (\gls{vmax}), and the speed factor itself must remain within a user-configurable range, defined by a minimum (\gls{minsf}) and maximum (\gls{maxsf}) value. If a calculated advisory would require a speed factor outside this permissible range, no advisory is issued, and the vehicle maintains its current behaviour.

\subsection{Signal Phase and Queue Timing Logic}
\label{sec:Glosa_Signal_Phase_Timing}

The controller's ability to plan for future signal phases relies on its timing logic, which calculates the time until a target green phase begins. This is not a single calculation, but an iterative process implemented within the core algorithm. The procedure steps through the signal plan's sequence of phases, systematically accumulating the durations of all non-green phases to compute the corrected time to the next green phase, denoted as \gls{ttscorr}.
\mynewline
Furthermore, an optional queue-awareness module can be enabled, described in Section~\vref{sec:Glosa_Queue_Length}. When active, this module estimates the queue-dissipation time, \gls{tadd}, based on the measured queue length (\gls{ells}). If the vehicle is approaching a red light, this delay is added directly to \gls{ttscorr}, ensuring the advisory accounts for the time needed to clear the waiting vehicles. This timing calculation provides the temporal foundation for the two distinct advisory strategies described below.

\subsection{Green Light Approach Strategy}
\label{sec:Glosa_Green_Light_Strategy}

When a \ac{glosa}-equipped vehicle approaches a \textit{green} light, the controller's primary objective is to pass the intersection without stopping. First, the algorithm evaluates whether the vehicle’s current estimated time to the junction, \gls{ttj}, is less than the time remaining in the current green phase.
\mynewline
If the vehicle is projected to arrive too late, the controller then evaluates if accelerating is a viable option. It calculates the fastest possible arrival time, which assumes the vehicle accelerates at its maximum rate. If this accelerated arrival time falls within the remaining green window (plus a small buffer of $2.1\unit{\second}$ for the yellow phase), the controller issues a simple advisory: it sets the vehicle's speed factor to its maximum permissible value, \gls{maxsf}, prompting it to speed up. If accelerating is not sufficient, the current green phase is deemed unreachable, and the algorithm proceeds to treat the situation as if it were approaching a red light. This strategy prioritizes throughput by maximizing the chance of clearing the intersection.

\subsection{Red Light Approach and Speed Adaptation}
\label{sec:Glosa_Red_Light_Strategy}

When a vehicle approaches a \textit{red} light, or when a green light is deemed unreachable, the system's objective shifts from simply crossing to arriving in the most efficient manner. In this case, the algorithm delegates the task to a dedicated procedure, which computes a precise optimal advisory speed, \gls{vopt}.

\subsection{Advisory Speed Calculation and Application}
\label{sec:Glosa_Speed_Adaptation}

The core task at this stage is to compute a theoretical optimal velocity, \gls{vopt}, that enables the vehicle to arrive at the intersection in synchrony with the target green phase. To achieve this, the algorithm models the vehicle's trajectory as a three-phase kinematic sequence: an initial deceleration from the current speed, \gls{vego}, to the optimal speed, \gls{vopt}; a constant-speed cruising phase at \gls{vopt}; and a final acceleration back to the desired maximum speed, \gls{vmax}.
\mynewline
The derivation solves for the target speed, \gls{vopt}, by considering the total distance to the intersection, \gls{dup}, and the effective time-to-switch, \gls{ttscorr}, while respecting the vehicle's maximum acceleration, \gls{amax}, and maximum deceleration, \gls{bmax}. First, the algorithm calculates a preliminary target speed to determine whether the vehicle needs to accelerate or decelerate, and then solves the full quadratic equation for the optimal speed, \gls{vopt}. The key calculation is the solution to the root argument of the quadratic formula:
\begin{equation}
    \mathrm{root\_arg} = ad \cdot \left( 2d(s - wt) - (v-w)^2 + a(dt^2 + 2(s - tv)) \right)
\end{equation}
where $v$ is the current speed (\gls{vego}), $s$ is the distance (\gls{dup}), $t$ is the time (\gls{ttscorr}), $a$ is max acceleration (\gls{amax}), $d$ is max deceleration (\gls{bmax}), and $w$ is the target maximum speed (\gls{vmax}). A valid, real solution exists only if this root argument is non-negative.
\mynewline
Once the optimal speed \gls{vopt} is calculated from this equation, it undergoes the validation checks described in Section~\vref{sec:Glosa_Speed_Factor_Model}. If the advisory is valid, the algorithm updates the vehicle’s speed factor to guide it along the computed optimal trajectory.

\subsection{Optional Queue Length Consideration}
\label{sec:Glosa_Queue_Length}

To account for the effect of vehicular queues upstream of the stop line, the baseline algorithm includes an optional module that can be enabled to incorporate real-time measurements of queue length, denoted \gls{ells}. This functionality uses \ac{sumo}'s built-in lane-area detectors, which directly measure the total queued distance in metres, obviating the need for indirect estimation. The algorithm then computes the expected queue-induced delay, \gls{tadd}, using the empirically derived linear model proposed by Lenz \cite{Lenz2024}:
\begin{equation}
    \gls{tadd} = (\gls{ells} \times 0.21\,\unit{\second\per\metre}) + 3\,\unit{\second}
\end{equation}
where the coefficient represents the queue discharge rate and the additive term accounts for the reaction and start-up time of the vehicles. This calculated delay is then integrated into the advisory logic in two distinct ways, depending on the current state of the traffic signal.
\mynewline
First, if the signal is red, the queue delay \gls{tadd} is added directly to the time-to-switch, \gls{tts}. This ensures that the advised speed profile targets a green phase that accounts for the time required for the existing queue to clear.
\mynewline
Second, if the signal is already green, the algorithm performs a feasibility check to determine if the ego vehicle can pass the intersection during the current green interval. It calculates an adjusted time-to-junction, which incorporates the queue delay relative to the time already elapsed in the green phase, \gls{telapsed}:
\begin{equation}
    \gls{ttj}' = \gls{ttj} + (\gls{tadd} - \gls{telapsed})
\end{equation}
If this adjusted arrival time is less than or equal to the remaining green time ($\gls{ttj}' \le \gls{tts}$), the vehicle is deemed able to clear the intersection, and no modification to the speed advisory is necessary. However, if the adjusted time exceeds the remaining green time, the current green phase is considered unreachable. In this case, the algorithm proceeds as if the light were red, beginning its iterative search for the next feasible green cycle. This optional queue-awareness module can be toggled on or off, allowing its impact on system performance to be isolated and evaluated.

\subsection{Safety and Model Override Parameters}
\label{sec:Glosa_Safety_Overrides}

In addition to its core logic, the \ac{glosa} device in \ac{sumo} includes two advanced boolean parameters that allow for fine-tuning its interaction with the standard traffic model and safety protocols. These are intended for research purposes to isolate specific behaviours.
\mynewline
The first parameter, \texttt{myOverrideSafety}, allows the \ac{glosa} advisory to ignore the actual, real-time state of the traffic light. When enabled, the vehicle will follow the speed advice derived from the signal plan's schedule, even if an unforeseen event (like an emergency vehicle preemption) causes the physical light to differ from its expected state. This is primarily used to test the algorithm's performance based purely on the scheduled signal data.
\mynewline
The second parameter, \texttt{myIgnoreCFModel}, permits the advisory to disregard non-critical speed adjustments suggested by the underlying car-following model. In a standard simulation, the car-following model might slightly reduce a vehicle's speed for safety, even if no collision is imminent. When this parameter is active, the \ac{glosa} device will strictly enforce its calculated optimal speed, overriding these minor, non-essential adjustments from the car-following logic. For all experiments in this thesis, both of these parameters were left at their default setting (disabled) to ensure the advisories operated under realistic safety constraints.