\chapter{Simulation Setup and Configuration}
\label{ch:SimulationSetupConfiguration}

For testing \ac{glosa} schemes at the signalized \emph{Neckartor} junction, this chapter sets up a standard way to do experiments. The investigation employs a microscopic network developed in \ac{sumo} and controlled via the \ac{traci} interface. This digital twin accurately replicates the junction's lane geometry, \ac{spat} phasing, and local diesel-ban regulations. Synthetically generated traffic demand, anchored to measured flow data, spans a wide range of conditions from free-flow to oversaturation.
\mynewline
The chapter is organized into three main sections. Section~\ref{sec:SimEnvironment} outlines the simulation environment, including the network configuration, vehicle fleet composition, \ac{eidm} parameters, and the emission logging setup. Section~\ref{sec:exec_protocol} specifies the execution protocol, which systematically varies traffic flow, \ac{glosa} market penetration, communication range, and other key parameters such as the slack time, \gls{tslack}. Finally, Section~\ref{sec:performance_evaluation} defines the performance metrics used for evaluation, including vehicle stop counts, mean speed, fuel consumption, pollutant emissions, and break-even penetration rates.

% --------------------------------------------------------------------
\section{Simulation Environment}
\label{sec:SimEnvironment}
% --------------------------------------------------------------------
The experiments are executed in the open-source \ac{sumo} framework. The platform delivers microscale vehicle dynamics and deterministic reproducibility. All inputs are supplied through static configuration files, and no on-line calibration is performed. Numerical integration uses the default solver provided by \ac{sumo}.
\mynewline
The digital network represents the signalised \emph{Neckartor} junction in Stuttgart. Geometry, lane layout and dedicated turn pockets are reproduced exactly. Signal timing is encoded through fixed \gls{spat} and \gls{map} plans extracted from the roadside controller. The plans are assumed invariant over the entire run. Diesel vehicles that do not meet Euro 6 are excluded, because a local driving ban blocks Euro 5 and older classes. Vehicle counts measured in 2023 report a peak demand of 2800\,veh\,h\(^{-1}\).
\mynewline
Traffic demand is created synthetically but anchored to real loop data collected on the B14 corridor heading toward the city centre. Flow levels range from light to heavy congestion. Vehicles enter with uniform inter-arrival times within each scenario. Route assignment follows an equal probability rule across all legitimate movements, ensuring balanced saturation of every lane group. Market penetration of \ac{glosa} is treated as a categorical variable. Percentages from 0\% to 100\% are tested in 10\% steps. Two independent insertion streams keep \ac{glosa}-equipped and non-equipped vehicles statistically separate, while still sharing the same network conditions.
\mynewline
Longitudinal behaviour is governed by the \ac{eidm}. The choice reflects its proven ability to handle stop-and-go waves and queue discharge dynamics.  A heterogeneous fleet of 486 virtual vehicles is synthesised by combining five variable parameters with nine fixed ones. Tables \ref{tab:EIDMFixed} and \ref{tab:EIDMVar} list the settings. Variable parameters are sampled uniformly across the stated bounds for every vehicle instance at spawn time. The resulting population covers a plausible span of car sizes, aggressiveness levels and reaction times without biasing the results toward a single calibration point.
\mynewline
The temporal horizon of each simulation run is 43.33 min. The first 10-min act as a warm-up period, allowing queues and signal coordination to reach a stable regime. Only the remaining 33.33 min enter the statistical analysis. The simulation time step is fixed to 0.1 s, providing sufficient resolution for the \ac{eidm} equation set and for accurate emission integration. Data collection is spatially clipped to a 1 km upstream cordon and a 200 m downstream cordon centred at the stop line. Sampling outside this window is disabled to focus on the area most sensitive to speed advice.
\mynewline
Every vehicle writes a dedicated log that stores time stamp, position, speed, acceleration and instantaneous emission rates. All emission quantities are calculated by \ac{sumo}’s internal look-up tables. Logs are flushed at every simulation step, enabling post-processing with external statistical scripts. The same configuration governs every scenario, ensuring comparability across flow, penetration, and model parameter sweeps.

\begin{table}[tbp]
  \centering
  \caption{Fixed \ac{eidm} parameters shared by all vehicles.}
  \label{tab:EIDMFixed}
  \begin{tabular}{lcc}
    \toprule
    Parameter & Symbol & Value \\
    \midrule
    Desired time headway           & \(T\)               & 1.50\,s \\
    Comfortable deceleration       & \(\gls{bmax}\)      & 4.00\,m\,s\(^{-2}\) \\
    Acceleration exponent          & \(\delta\)          & 4 \\
    Vehicle length                 & \(\ell\)            & 4.50\,m \\
    Minimum speed                  & \(\gls{vmin}\)      & 0.00\,m\,s\(^{-1}\) \\
    Maximum deceleration jerk      & \(j_{\text{dec}}\)  & 1.50\,m\,s\(^{-3}\) \\
    Maximum acceleration jerk      & \(j_{\text{acc}}\)  & 1.50\,m\,s\(^{-3}\) \\
    Coolness factor upper bound    & \(c_{\max}\)        & 1.00\,– \\
    Lane-change politeness         & \(p\)               & 0.50\,– \\
    \bottomrule
  \end{tabular}
\end{table}

\begin{table}[tbp]
  \centering
  \caption{Variable \ac{eidm} parameters used to generate the 486-vehicle fleet.  Each parameter is drawn independently from a uniform distribution over the given range.}
  \label{tab:EIDMVar}
  \begin{tabular}{lccc}
    \toprule
    Parameter                    & Symbol         & Range        & Unit \\
    \midrule
    Maximum acceleration         & \(\gls{amax}\) & 1.0–3.5      & m\,s\(^{-2}\) \\
    Minimum gap                  & \(s_{0}\)      & 1.0–3.0      & m \\
    Reaction time                & \(\tau\)       & 0.8–1.4      & s \\
    Coolness factor              & \(c\)          & 0.0–1.0      & – \\
    Speed factor                 & \gls{sf}       & 0.8–1.2      & – \\
    \bottomrule
  \end{tabular}
\end{table}

\section{Execution Protocol and Parameterization}
\label{sec:exec_protocol}

The execution protocol translates the conceptual design into a reproducible set of simulation runs that sample all relevant boundary conditions while keeping computational cost within practical limits. Reproducibility is guaranteed by fixing the random seed to \texttt{12345}; thus, every pseudo-random operation, including route choice, lane assignment, and vehicle parameter draws, yields identical sequences across repeated executions.
\mynewline
Each run evaluates a unique combination of algorithm, traffic flow, market penetration, and fuel model, showed in table~\vref{tab:ScenarioMatrix}. The full parameter sweep, detailed below, results in a total of $352$ distinct simulation scenarios.
\mynewline

\begin{enumerate}
    \item \textbf{Controller Algorithm.} Each scenario is run with one of two \ac{glosa} algorithm variants: (i) the \emph{baseline} or \emph{flow-optimised} controller, which targets minimum delay, and (ii) the \emph{eco-driving} controller, which minimises fuel consumption.
    
    \item \textbf{Traffic Flow.} Eight demand levels are enforced by modulating the number of inserted vehicles, corresponding to flows of $69$, $138$, $346$, $692$, $1385$, $2077$, $2769$, and $3462\unit{\veh\per\hour}$, respectively. This range covers traffic states from free-flow to oversaturated conditions.
    
    \item \textbf{Market Penetration.} The \ac{glosa} \ac{mpr} is varied from $0\%$ to $100\%$ in $10$-percentage-point increments. A Bernoulli draw at spawn time classifies each vehicle as equipped or non-equipped.

    \item \textbf{Fuel Model.} To assess sensitivity to the underlying emission calculations, each scenario is executed twice: once using the polynomial-based \ac{hbefa}4 model and once using the more detailed, physics-based \textsc{PHEMlight}5 model, as specified in Section~\ref{subsubsec:detailed_emission_models}.

    \item \textbf{Communication range.} Prior work by Lenz \cite{Lenz2024} found no statistically significant difference between $500~\unit{\meter}$ and $1000~\unit{\meter}$ communication horizons; therefore only the $500~\unit{\meter}$ radius is tested. The advisory is sent whenever an equipped vehicle’s distance to the stop line is $\gls{dup} \leq 500~\unit{\meter}$ upstream (i.e., up to $500~\unit{\meter}$ before the stop line) and $\gls{ddown} \leq 200~\unit{\meter}$ downstream (i.e., up to $200~\unit{\meter}$ after the stop line).
\end{enumerate}

\begin{table}[htb]
  \centering
  \caption[Experimental Design Matrix]{Primary factors of the experimental design. The full factorial combination of these parameters results in a total of 352 unique simulation runs.}
  \label{tab:ScenarioMatrix}
  \begin{tabular}{l c c c}
    \toprule
    \textbf{Factor} & \textbf{Levels} & \textbf{Values} & \textbf{Unit} \\
    \midrule
    Controller Algorithm & $2$ & Flow-Optimised, Eco-Driving & - \\
    Traffic Flow ($q$) & $8$ & $69-3462$ & $\unit{\veh\per\hour}$ \\
    Market Penetration ($p$) & $11$ & $0-100$ (in steps of $10$) & $\%$ \\
    Fuel Model & $2$ & HBEFA4, PHEMlight5 & - \\
    \midrule
    \multicolumn{2}{l}{\textbf{Total Scenarios}} & \multicolumn{2}{c}{$2 \times 8 \times 11 \times 2 = 352$} \\
    \bottomrule
  \end{tabular}
\end{table}

All other simulation and controller settings, detailed in Table~\ref{tab:ConstantSimParams}, were held constant across the entire experimental sweep. This approach ensures a controlled comparison by isolating the impact of the primary factors defined in the experimental design (Table~\ref{tab:ScenarioMatrix}). The fixed parameters govern the core logic of the optimisation routine and the physical assumptions of the simulation environment.
\mynewline

\begin{table}[htb]
  \centering
  \caption[Constant Simulation and Controller Parameters]{Constant parameters applied across all simulation scenarios to ensure consistency and isolate the effects of the independent variables.}
  \label{tab:ConstantSimParams}
  \resizebox{\textwidth}{!}{%
    \begin{tabular}{l l c l}
      \toprule
      \textbf{Parameter} & \textbf{Symbol} & \textbf{Value} & \textbf{Description} \\
      \midrule
      Signal Switch Time Buffer & - & $2.1\unit{s}$ & Fixed safety time to account for signal actuation delays. \\
      Optimizer Slack Time & \gls{tslack} & $[0.1, 5]\unit{s}$ & Uniformly sampled uncertainty in residual phase duration. \\
      Downstream Horizon & $\gls{ddown}$ & $200\unit{m}$ & Fixed recovery distance for the post-junction trajectory segment. \\
      Yellow Phase Buffer & \gls{epsyellow} & $0.5\unit{s}$ & Safety margin to prevent advising trajectories through an amber phase. \\
      Optimizer Search Step & $\gls{deltaa}$ & $0.01\unit{m\,s^{-2}}$ & Numerical search increment for the acceleration line search. \\
      Terminal Speed Weight & $\gls{alphaspeed}$ & $1.0$ & Weighting factor for the terminal speed penalty in the cost function. \\
      Arrival Time Weight & $\gls{alphatime}$ & $0.5$ & Weighting factor for the arrival time penalty in the cost function. \\
      \bottomrule
    \end{tabular}%
  }
\end{table}

A key methodological choice was to disable dynamic queue estimation in the controller. The queue length was forcibly set to zero in all optimiser calls to isolate the pure speed-advice effects from the confounding variable of queue-prediction accuracy. This ensures that the observed performance differences are attributable solely to the core logic of the \ac{flow-glosa} versus the \ac{eco-glosa}.
\mynewline
The eco-driving controller's optimiser evaluates potential speed profiles within the advisory range, which extends from $500\unit{m}$ upstream to a downstream horizon of $200\unit{m}$ past the stop line. During this process, the upstream vehicle motion is constrained by the \ac{eidm} car-following envelope, as detailed in Section~\ref{sec:SimEnvironment}. The controller performs a univariate line search over the upstream acceleration ($\gls{aup}$) using the search step ($\gls{deltaa}$) defined in Table~\ref{tab:ConstantSimParams}. Any resulting trajectory that violates the terminal conditions, such as the minimum speed (\gls{vmin}) or maximum comfortable deceleration (\gls{bmax}), is discarded as infeasible.
\mynewline
The resulting dataset from these $352$ runs provides the foundation for all performance metrics analysed in Section~\ref{sec:performance_evaluation}. By controlling every stochastic degree of freedom and generating exhaustive logs, the protocol ensures that the study is fully reproducible using the same configuration files and seed. This transparent and modular design allows future research to build upon this work by simply appending new factors to the existing experimental matrix.

\section{Performance Metrics and Evaluation Methodology}
\label{sec:performance_evaluation}

The assessment framework is designed to expose the trade-off between smoother traffic flow and ecological benefit introduced by the two \ac{glosa} variants. All quantities are derived from the trajectory and emission logs produced by \ac{sumo}. Each scenario is executed exactly once and therefore yields a single deterministic outcome; confidence intervals are not applicable.

\paragraph{Microscopic traffic efficiency.}
Four indicators quantify longitudinal driving quality.

\begin{enumerate}[label=\textbf{(\roman*)}]
\item \emph{Stops per vehicle} $\gls{nstop}$ count transitions where the instantaneous speed falls below \(2\,\mathrm{m\,s^{-1}}\) after being above that threshold --- mirroring the empirical definition used in field studies:
\begin{equation}
    \gls{nstop}_{,i}= \sum_{k=2}^{K}{\mathbb I}\!\left(v_{i,k}<2 \land v_{i,k-1}\ge 2\right).
\end{equation}
Scenario means follow <span class="math-inline">\\bar \\gls\{nstop\} \= \\tfrac1N\\sum\_i \{\\gls\{nstop\}\}\_\{,i\}</span>.

\item \emph{Mean speed} averages all time stamps of all vehicles,
\begin{equation}
    \bar v = \frac1{N K}\sum_{i=1}^{N}\sum_{k=1}^{K} v_{i,k},
\end{equation}
with \(K\) the number of stored steps per vehicle.

\item \emph{Mean absolute acceleration} is computed likewise,
\begin{equation}
    \overline{|a|} = \frac1{N K}\sum_{i,k}|a_{i,k}|,
\end{equation}
and serves as a surrogate for passenger comfort and tyre wear.
\end{enumerate}

\paragraph{Macroscopic throughput.}
Throughput is quantified as the mean number of vehicles exiting the study segment per signal cycle, rather than as an hourly rate. Each cycle of length $T_{\mathrm{cycle}}=120\,$s comprises two green‐phase intervals, $I_{1}=[2,35]\,$s and $I_{2}=[62,96]\,$s. For cycle $c$, let
\begin{equation}
    N_{c} \;=\; \#\bigl\{\,i : t_{i,\mathrm{exit}}\in I_{1}\cup I_{2}\bigr\},
\end{equation}
where $t_{i,\mathrm{exit}}$ is the exit timestamp of vehicle $i$. The average per-cycle throughput is then
\begin{equation}
    \overline{N}_{\mathrm{cycle}}\;=\;\frac{1}{C}\sum_{c=1}^{C}N_{c}.
\end{equation}

\paragraph{Energy demand and pollutant emissions.}
Environmental impact is characterised by per‐vehicle pollutant rates, with $CO_{2}$ as the primary metric (fuel mass is proportional and thus not shown separately). Emissions are computed over the 1.2\,km evaluation corridor by summing the instantaneous rates in the log, multiplying by the time step $\Delta t = 0.1\,$s, converting to grams and normalising by distance:
\begin{equation}
    e_{i}
    = \frac{1}{1000\,L} \sum_{k=s_i}^{e_i} \dot m^{CO_{2}}_{k}\,\Delta t,
    \quad
    L = 1.2\ \mathrm{km},
    \quad
    \Delta t = 0.1\ \mathrm{s}.
\end{equation}
The same form applies to $NO_{x}$ and $PM_{x}$, using $\dot m^{NO_{x}}_{k}$ and $\dot m^{PM_{x}}_{k}$ from the emission log. Two inventories, \ac{hbefa}~4 and \textsc{PHEMlight}~5, are processed in parallel, ensuring that any model‐specific bias appears consistently across all scenarios. 

\paragraph{Relative performance baseline.}
Each metric $y$ is expressed as a percentage deviation from the flow‐matched, 0\% penetration scenario:

\begin{equation}
    \Delta y(p) = \frac{y(p) - y(0)}{y(0)} \times 100\%.
\end{equation}

Separate traces $\Delta y_{\mathrm{flow}}$ and $\Delta y_{\mathrm{eco}}$ are plotted for the flow-optimised and eco-driving controllers, respectively. Negative values indicate an improvement.  

\paragraph{Break-even penetration rate.}
For each flow level the critical \ac{mpr} $p^{\star}$ is defined by the solution of
\begin{equation}
    \Delta y_{\mathrm{flow}}(p^{\star})
    \;=\;
    \Delta y_{\mathrm{eco}}(p^{\star}),
\end{equation}
where $\Delta y_{\mathrm{flow}}$ and $\Delta y_{\mathrm{eco}}$ are the percentage‐change curves for the flow-optimised and eco-driving controllers, respectively. Rather than simple linear interpolation, we fit continuous cubic splines $S_{\mathrm{flow}}(p)$ and $S_{\mathrm{eco}}(p)$ through the discrete penetration points $\{p_{k},\,\Delta y(p_{k})\}$. Then $p^{\star}$ is obtained by numerically solving
\begin{equation}
    S_{\mathrm{flow}}(p)\;-\;S_{\mathrm{eco}}(p)\;=\;0
    \quad\text{for}\quad p\in[0,100]\%.
\end{equation}
The root‐finding (e.g.\ Brent’s method) ensures sub‐percent accuracy. We compute one $p^{\star}$ per demand level; the collection of eight values is summarised by its median and range.  

\paragraph{Computational viability.}
Feasibility is assessed by recording the wall‐clock time per scenario via \ac{traci}. Let $T_{\mathrm{traci,0}}$ and $T_{\mathrm{traci,eco}}$ be the runtimes for the baseline and eco‐driving controllers, respectively, and let $T_{\mathrm{sumo,0}}$ denote the pure \ac{sumo} execution time without any external interface. We define the Python‐interface overhead as
\begin{equation}
    \phi \;=\; \frac{T_{\mathrm{traci,0}}}{T_{\mathrm{sumo,0}}},
\end{equation}
which quantifies the relative slowdown introduced by the \ac{traci} loop. The incremental cost of the eco‐driving optimisation is then
\begin{equation}
    \delta_{\mathrm{eco}} 
    \;=\; \frac{T_{\mathrm{traci,eco}} - T_{\mathrm{traci,0}}}{T_{\mathrm{traci,0}}},
\end{equation}
expressing the extra runtime as a fraction of the baseline. Both absolute runtimes (in $s \cdot h^{-1}$) and relative overheads (in \%) are reported to balance environmental gains against computational expense.

\paragraph{Visualisation.}
Results are conveyed using a harmonised set of graphical summaries that balance clarity and detail. 
\begin{enumerate}[label=\textbf{(\alph*)},leftmargin=*]
    \item Line charts depict mean speed $\bar v$, mean absolute acceleration $\overline{|a|}$, average stops per vehicle $\bar N_{\mathrm{stop}}$, per‐route travel time $t_{\mathrm{route}}$ and per‐vehicle CO$_2$ emissions as functions of market penetration. Each demand level occupies its panel to facilitate direct comparisons across flows.
    \item Heat maps illustrate the percentage change in CO$_2$, $\Delta\mathrm{CO}_{2}$, over the two‐dimensional grid defined by traffic flow and penetration. This view highlights the region where eco‐driving achieves net environmental benefit.
    \item Bar plots summarise computational overheads by displaying the interface factor $\phi$ and the eco‐optimisation cost $\delta_{\mathrm{eco}}$ for each scenario, making it easy to weigh runtime against performance gains.
    \item Break‐even penetration rates $p^{\star}$ are marked as vertical dashed lines on the CO$_2$ line plots, directly annotating the threshold at which eco‐driving and flow‐optimised controllers perform equally.
\end{enumerate}
No error bars are shown, since each point is derived from a single deterministic run rather than a statistical ensemble.  

\paragraph{Interpretation focus.}
The ensuing discussion concentrates on two aspects: (i) the absolute and relative reduction of \(\mathrm{CO_{2}}\) emissions achieved by the eco-driving algorithm, and (ii) the penetration–demand corridor in which this reduction outweighs any throughput penalty. By presenting results normalised to the baseline and indexed by market penetration, the methodology yields transparent, scalable insights that remain valid even if additional algorithms or network geometries are appended in future work.