\section{Principles and Architectures of GLOSA Systems}
\label{sec:glosa}

Building upon the preceding discussion of \ac{sumo} and its capabilities for microscopic traffic simulation, this section introduces the fundamental concepts and overall system architecture of \ac{glosa} systems, laying the groundwork for subsequent exploration of algorithmic approaches and deployment considerations.

\subsection{Fundamental Concepts and System Architecture}
\label{subsec:glosa_concepts_architecture}

\ac{glosa} systems are an \ac{adas} that advises the driver to adopt an optimal speed, computed via an objective function balancing total energy consumption and travel time, based on traffic signal phase and timing information. By doing so, they minimise stops at red lights, thereby reducing idling and re-acceleration events, and decrease overall trip duration. \cite{RealTimeGLOSA2020}
\mynewline
At the core of \ac{glosa} operational principles is the utilisation of \ac{spat} messages, which provide dynamic signal phase status and timing information, including current signal states and forthcoming phase transitions, enabling in-vehicle applications to optimise speed recommendations. These messages are standardised in the United States under SAE J2735 (Society of Automotive Engineers) \cite{USDOTSPaT2022} and in Europe under ETSI TS 102 724 (European Telecommunications Standards Institute) \cite{ETSI1027242012}, ensuring interoperability across diverse vendors and deployments. \ac{glosa} algorithms solve a moving-horizon dynamic optimisation problem to compute the remaining \ac{ttg} and to identify when the current green interval will end, thereby determining if a deceleration phase is required, to derive dynamic \acp{asl} which enable vehicles to traverse intersections during green phases with minimal idling and re-acceleration. To enhance safety and compliance, these systems may also integrate red-light-running prevention strategies by issuing warnings when recommended speed profiles would violate red phase boundaries. \cite{BusesGLOSA2022}

\begin{figure}[htbp]
  \centering
  % Placeholder for architecture diagram
  \fbox{\parbox[t][4cm][c]{0.8\textwidth}{\centering Architecture diagram placeholder}}
  \caption{Zeit bis grün}
  \label{fig:glosa_architecture}
\end{figure}

\ac{glosa} systems can be categorised along two primary dimensions: communication types and levels of automation. In terms of communication, \ac{glosa} may rely on \ac{i2v} broadcasts from \acp{rsu}, \ac{v2i} reporting for vehicle position and speed feedback, and optionally \ac{v2v} exchanges to support cooperative speed harmonisation among neighbouring vehicles. \cite{Seredynski2013} With respect to automation, \ac{glosa} systems are typically classified into two main categories. Manual advisory systems present speed recommendations to the driver via a \ac{hmi}, without any direct influence on vehicle control \cite{BusesGLOSA2022}. In contrast, semi-automated and fully automated systems embed \ac{glosa} logic within higher-level \acp{adas}, allowing the vehicle to adjust its speed autonomously—either with limited driver oversight or entirely without driver involvement. \cite{Almannaa2019}
\mynewline
The system architecture of a typical \ac{glosa} deployment splits into two parts: infrastructure\=/side components and vehicle-side components. On the infrastructure side, traffic signal controllers generate \ac{spat} messages. These messages are broadcast by \acp{rsu} over the 5 GHz band using either IEEE 802.11p (ITS-G5) or \ac{c-v2x} technologies. Often, \acp{rsu} sit alongside adaptive signal controllers and link to a \ac{tmc}. The \ac{tmc} aggregates data from loop detectors and \ac{fcd} sources. This setup enables predictive signal timing and centralised control strategies.
On the vehicle side, \acp{obu} receive the \ac{spat} broadcasts. They then perform \ac{map} matching, localisation and velocity estimation to compute energy-optimal speed advisories. \cite{Sambeek2015} Low-cost smartphone apps can also collect \ac{spat} and other V2X messages via cellular or Bluetooth links, offering a cheaper alternative to dedicated \acp{obu} \cite{Gao2016}. Some deployments add a cloud-based V2X platform—such as the \ac{v2x} Hub—to relay messages over redundant paths, extend coverage beyond short-range broadcasts, and enrich advisories with contextual data (e.g.\ weather or incident alerts). \cite{Hadi2023}
\mynewline
The data flow in a \ac{glosa} system follows a defined sequence of operations:

\begin{enumerate}[leftmargin=*, label=\textbf{Step \arabic*:}]
  \item The traffic signal controller forecasts upcoming phase timings and emits \ac{spat} messages.
  \item \acp{rsu} broadcast the \ac{spat} messages to all approaching vehicles.
  \item Each \ac{obu} or compatible mobile device \ac{map}-matches the received data and estimates its current position and speed.
  \item The advisory algorithm combines \ac{ttg}, \ac{ttr}, and a vehicle dynamics model to compute an energy-optimal speed profile.
  \item The resulting advisory is delivered through the \ac{hmi}, for example via head-up display, dashboard indicator, or audio prompt.
\end{enumerate}

Together, these steps form a seamless end-to-end pipeline that provides drivers with accurate, timely speed recommendations to enhance safety, traffic flow and environmental performance. In fully or semi-autonomous vehicles, the computed advisory bypasses the human–machine interface and is fed directly into the vehicle’s longitudinal control system, enabling automatic speed adjustment in response to signal timing information.

\begin{figure}[htbp]
  \centering
  % Placeholder for architecture diagram
  \fbox{\parbox[t][4cm][c]{0.8\textwidth}{\centering Architecture diagram placeholder}}
  \caption{Data Flow}
  \label{fig:glosa_architecture}
\end{figure}

Vehicle-side components in \ac{glosa} systems typically rely on advanced multi-rate sensor fusion techniques. This approach integrates GNSS Real-Time Kinematic (RTK), inertial measurement units (IMU), \ac{map} matching algorithms, and vehicle kinematic models via Kalman filtering, achieving highly accurate, decimetre-level positioning and precise velocity estimations. Accurate and reliable positioning data are essential for \ac{glosa} applications, as even slight inaccuracies can significantly degrade the quality of speed recommendations and reduce the potential fuel-saving benefits of the system. \cite{Vignarca2023} Computed advisories are presented through carefully engineered \acp{hmi}, designed specifically to convey information clearly, promptly, and with minimal driver distraction. Typical interface elements include graphical speed bars, countdown timers indicating upcoming signal phase changes, and auditory alerts activated when the vehicle significantly deviates from recommended speeds. The clarity and effectiveness of the \ac{hmi} directly influence driver compliance with speed recommendations, thus impacting the overall efficiency and safety benefits of the \ac{glosa} system.

Collectively, these components constitute an integrated and precise information pipeline that enables \ac{glosa} to achieve its core objectives of reducing vehicle stops, smoothing traffic flow, and lowering emissions, while simultaneously maintaining or improving intersection throughput and safety.


\subsection{Algorithmic Approaches and Deployment Challenges}
\label{subsec:glosa_algorithms_challenges}

Algorithmic approaches to \ac{glosa} range from cooperative, network-level optimisation to egoistic, single-vehicle strategies, employing methods from simple heuristics to advanced optimisation and learning techniques. The following discussion examines the infrastructure-side inputs required, highlights representative field implementations, outlines standard evaluation metrics, and identifies principal deployment challenges --- such as phase synchronisation, communication reliability, privacy, and scalability --- before surveying emerging trends in connected and automated traffic systems.

\subsubsection{Classification of Algorithmic Strategies}
\label{subsubsec:classification_algorithms}

Two principal orientations characterise advisory algorithms for signal‐based speed recommendation. In the cooperative orientation, vehicles and infrastructure collaborate to optimise metrics at the network level. Each vehicle receives \ac{spat} and state information from adjacent intersections and neighbouring vehicles. A centralised or distributed optimiser then computes a set of speed advisories that collectively minimise aggregate fuel consumption, emissions, and delay across multiple traffic streams. This approach benefits from high penetration rates and reliable communication, but incurs overhead in data exchange and requires synchronisation among optimisation agents.
By contrast, the egoistic orientation treats each vehicle as an independent decision‐maker. Advisories are derived solely from local \ac{spat} messages and individual vehicle dynamics models. Algorithms determine the speed profile that minimises the single vehicle’s fuel use or travel time, without regard for effects on other road users. Egoistic methods scale efficiently to low penetration scenarios and impose minimal demands on communication infrastructure. However, they cannot exploit potential synergies offered by coordinated control and may yield suboptimal network performance when adoption reaches critical mass.
Both orientations involve trade‐offs between global optimality, scalability, communication requirements and robustness to partial system participation.  
\mynewline
Building on these two orientations, practical implementations rely on a continuum of decision-making paradigms that span from simple rule sets to sophisticated optimisation frameworks. Rule‐based methods derive advisory speeds using explicit decision rules at each control interval. For example, Masera et al. \cite{Masera2019} implement a mean‐green method that computes a constant speed by dividing the distance to the intersection by the remaining green phase duration, constrained within vehicle acceleration limits and legal speed bounds. Katsaros et al. \cite{Katsaros2011} using polynomial smoothing techniques to fit a continuous speed trajectory through predicted phase transition times, thereby enforcing smooth acceleration and deceleration to limit jerk. Additional heuristics trigger adjustments only when \ac{ttr} falls below a fixed threshold or when distance‐to‐signal exceeds a preset margin, further reducing computational and communication overhead. These schemes require only basic inputs --- \ac{ttg}, \ac{ttr} and current vehicle state --- and execute with low latency, enabling deployment on lightweight, resource‐constrained devices. Their simplicity, however, can lead to conservative or suboptimal advisories under highly variable traffic conditions or inaccurate signal predictions, motivating the development of more advanced methods.
\mynewline
While rule-based schemes offer low latency and ease of deployment, they often fall short under complex conditions, which motivates the turn to iterative heuristic and metaheuristic searches. Such Algorithms generate approximate speed advisories by exploring the space of candidate speed profiles through iterative search. Simchon and Rabinovici \cite{Simchon2020} develop a multi-segment dynamic \ac{glosa} formulation for \acp{ev} that yields a nonconvex optimisation problem; they apply a relaxation procedure to obtain a tractable program and implement a real-time advisory algorithm. Numerical simulations demonstrate average energy savings of 44\% compared to a naive driver strategy, with computation times below 0.5\,s per update. Seredynski et al. \cite{Seredynski2013} introduce a genetic algorithm that optimises advisory speeds across multiple signalised segments simultaneously; in free-flow scenarios, their multi-segment \ac{ga} achieves significant reductions in travel time and fuel consumption relative to single-segment heuristics. Zhou et al. \cite{Zhou2015} prove the \ac{np}-completeness of the general trajectory planning problem and propose a parsimonious shooting heuristic. Their method decomposes each vehicle’s trajectory into analytically solvable segments under finite acceleration bounds, guaranteeing feasible solutions; simulations verify marked improvements in trajectory smoothness and intersection throughput. Zheng et al. \cite{Zheng2015} formally establish the \ac{np}-hardness of green light optimal velocity planning with binary speed choices and design a tailored genetic algorithm that directly handles multiple intersections. Their experiments confirm superior fuel economy gains over single-intersection methods and demonstrate the approach’s suitability for resource-constrained on-board or edge computing platforms. 
\mynewline
Despite their flexibility, heuristic methods provide no optimality guarantees, prompting researchers to adopt formal optimisation models such as \ac{milp}. \acp{milp} offer a unified framework to integrate signal timing, vehicle dynamics and safety constraints into a single linear program with integer decision variables. Fayazi et al. \cite{Fayazi2017} develop a single‐junction \ac{milp} that assigns each autonomous vehicle a discrete entry time by minimising the overall makespan and penalising deviations from desired arrival windows. Solved every 6s on a backend server using CPLEX, their scheme reduced the number of stopped vehicles from 1162 to 11 over one hour, improved average travel time by 7.5\% and cut stopped‐delay per vehicle by 52.4\% compared to a fixed‐time signal baseline. Ashtiani et al. \cite{Ashtiani2018} extend this concept to a 3×3 grid of intersections via a distributed \ac{milp} at each node; intersections exchange scheduled access times to coordinate flow, resulting in over 60\% reduction in queue lengths at the central junction and fuel‐economy gains of several \ac{mpg} relative to both uncoordinated \ac{milp} and fixed‐time control. Nguyen et al. \cite{Nguyen2022} propose a bi‐level \ac{milp} architecture that alternates between a network‐level dynamic traffic assignment and link‐level speed advisory optimisation. Their simulations show notable emission reductions, more uniform queue formation and smoother network flow, demonstrating the method’s capacity to reconcile macroscopic objectives with microscopic trajectory planning. Despite their provable optimality and flexible constraint handling, \acp{milp} incur substantial computational overhead and long solve times. To address this, researchers have explored receding‐horizon implementations, warm‐start techniques and problem decomposition. Nevertheless, real‐time deployment requires powerful onboard or edge computing resources and reliable, low‐latency communication links.   
\mynewline
Dynamic Programming offers an alternative to integer‐programming by discretising both state and time into a finite grid. \acp{dp} solve the Bellman equation via backward recursion, propagating value functions from a specified terminal condition. They can optimise multiple criteria, such as travel time and energy consumption, through a single composite cost function. The resolution of the grid governs the balance between accuracy and computational burden, since finer discretisation multiplies the number of states and transitions. For detailed \ac{dp} formulations and implementations addressing both flow‐oriented and energy‐aware objectives, see Sections~\ref{sec:flow_glosa} and \ref{sec:eco_glosa}.


\subsubsection{Performance Evaluation and Benchmarking}
\label{subsubsec:performance_evaluation}

A rigorous evaluation of \ac{glosa} algorithms must cover environmental, operational, comfort and safety dimensions. Environmental performance is quantified by fuel consumption and pollutant emissions (CO\textsubscript{2}, NO\textsubscript{x}). \cite{Kloeppel2019,Lenz2024} In simulation studies, these are computed using microscopic traffic models (e.g. \ac{sumo}) coupled with HBEFA emission modules (see section \ref{subsubsec:detailed_emission_models}), assigning each vehicle an emission class according to fleet distributions. Operational efficiency is assessed via total travel time, average delay per vehicle and number of complete stops at intersections, where a stop is defined as a speed drop below and subsequent rise above 2m/s per trip. Intersection throughput is measured by the mean number of vehicles passing each green phase. Comfort metrics capture driving smoothness through mean absolute acceleration, computed by averaging acceleration magnitudes over all time steps and vehicles. Safety assessment records red‐light violation events and critical conflict indicators at intersections to ensure advisories do not compromise safety. \cite{Kloeppel2019,Lenz2024}
\mynewline
To enable appropriate benchmarking, studies must standardise scenario parameters: signal control logic (fixed‐time versus adaptive)\cite{Kloeppel2020}, traffic demand profiles (volume, vehicle mix), vehicle fleet composition by emission class, communication models (latency, packet loss, penetration rate) and driver compliance assumptions. The evaluation period should exclude initial warm‐up frames within simulation settings to focus on steady‐state behaviour. \cite{Lenz2024} Emission and fuel‐consumption models, uniform definitions of stops and delay, and statistical analysis procedures (ANOVA, confidence intervals) must be applied consistently. \cite{Kloeppel2019} Such standardisation isolates algorithmic effects from scenario artifacts and promotes reproducibility across \ac{glosa} research. \cite{Kloeppel2020}


\subsection{Practical Implementations and Case Studies}
\label{subsec:practical_implementations}

Since their inception in research, \acp{glosa} have progressed into several field deployments and commercial products. This section reviews three representative implementations --- Audi’s Traffic Light Information in production vehicles, the European C-ROADS pilot deployments, and U.S. statewide V2I services --- before discussing interdependencies with related \ac{its} applications.
\mynewline
Audi was the first \ac{oem} to offer a consumer‐level \ac{glosa} feature under its \ac{tli} brand. Launched in August 2016 \cite{AudiV2I2016}, the system uses \ac{v2i} data provided by Traffic Technology Services to calculate \ac{ttg} and optimal driving speeds, displayed in the Virtual Cockpit and head-up display. Initial studies reported a 21\% reduction in time spent at signals and an average 20\% decrease in complete stops when drivers followed the advisory. \cite{AudiTechTalk} By 2019, Audi and TTS had expanded coverage to over 4 700 intersections in 13 U.S. cities, and European rollout began in Düsseldorf and other test sites. The service operates on a cloud platform, Personal Signal Assistant, that ingests live \ac{spat} data from roadside units and performs short‐term phase predictions before sending speed recommendations back to vehicles. \cite{TTS2019}
In Europe, the C-ROADS platform has coordinated large-scale \ac{c-its} pilot deployments across eighteen member states. The Day 1 C-ITS application GLOSA is one of the core service packages defined under Service Package SP57 by the ARC-IT framework. \cite{ARCITSP57} A mapping study of 64 publications revealed that C-ROADS pilots consistently deploy \ac{glosa} with interoperable \acp{rsu} and common message profiles (\ac{spat}/\ac{map} over ITS-G5) to ensure cross-border functionality. \cite{Mellegard2020} The 2022 Annual Pilot Overview Report documents over 1200 roadside stations installed and operational in urban and interurban corridors, with live demonstrations in Vienna, Antwerp–Helmond, and Stockholm. \cite{CROADSPilot2022} These multinational deployments highlight harmonisation of standards, centralized data portals for signal timing, and partner coordination across public and private stakeholders.
In the United States, Oregon DOT became the first state to activate a statewide \ac{v2i} service in December 2018, leveraging the TTS Personal Signal Assistant platform. Oregon’s service covers more than 1000 signalised intersections, providing \ac{spat} predictions and speed advisories directly to equipped vehicles. Concurrently, Virginia DOT expanded its “SmarterRoads” portal to serve over 1450 intersections, the largest single provider in North America, and plans to reach 3000 intersections statewide. \cite{TTS2019} These deployments illustrate how state DOTs can partner with technology providers to integrate signal data into consumer and fleet vehicles, improving traffic flow, safety and fuel economy across mixed‐fleet environments.
\mynewline
In real-world deployments, \ac{glosa} rarely operates in isolation. To maximise benefits and leverage existing infrastructure, \ac{glosa} is typically integrated with adjacent \ac{its} services that share SPaT/\ac{map} data and communication channels. The following discussion examines these interdependencies and how they amplify efficiency, environmental and safety gains.

\paragraph{Interdependencies with other ITS Applications}

\acp{glosa} offer natural synergies with adjacent \ac{its} functions that share \ac{spat} data and aim for energy efficiency, traffic smoothness and safety. \acp{ead} use the same \ac{spat} information as \ac{glosa} to plan both deceleration before an intersection and acceleration after it. Xia et al. \cite{Xia2014} demonstrate an \ac{ead} strategy that computes a deceleration trajectory to arrive at the stop line exactly when the signal turns green, minimising engine idling and brake usage; their simulation results show fuel savings of up to 12\% and CO\textsubscript{2} reductions of 10\% compared to a baseline with no advisory. The FHWA JPO report \cite{FHWA2016} consolidates multiple pilot studies, noting that \ac{ead} deployment on arterial corridors yields average fuel consumption decreases of 8–15\% and emission cuts of 5–12\% when applied alone. When fused with \ac{glosa}, the combined approach extends the advisory horizon both upstream and downstream of the intersection. Vehicles receive a continuous speed band that spans approach, stop-and-go and departure phases, which further amplifies fuel and emission benefits beyond isolated intersection optimisation.

Integration with \ac{acc} and its eco-cooperative variant, named \ac{ecoacc}, enables automated execution of advisory speeds without driver intervention. Almannaa et al. \cite{Almannaa2019} implement an \ac{ecoacc} prototype that links the \ac{glosa} advisory module with the vehicle’s longitudinal control system. In closed-track tests, their system achieved average fuel savings of 31\% and travel-time reductions of 9\% under mixed manual and automated conditions, while maintaining safe headways and comfortable acceleration profiles. The \ac{ecoacc} algorithm continuously adjusts the vehicle’s set-point speed to track the \ac{glosa} recommendation, smoothing both deceleration and acceleration phases. This tight coupling of \ac{glosa} advisories and \ac{acc} control loops removes variability due to driver behaviour and ensures consistent compliance with optimal speed bands.

Safety-oriented \ac{its} applications also benefit from the \ac{spat} and \ac{map} data streams used by \ac{glosa}. \acp{icw} analyse \ac{spat} updates and vehicle position to predict conflict points and alert drivers or automated driving functions to imminent crossing hazards. The ARC-IT service package SP57 defines shared message profiles that support both \ac{glosa} and \ac{icw} without additional infrastructure, enabling a single \ac{rsu} deployment to serve multiple safety and efficiency services. \cite{ARCITSP57} Dynamic traffic-jam warning systems can aggregate \ac{glosa} compliance data to forecast queue formation several hundred metres upstream and notify following vehicles to adjust speed proactively. By leveraging the same communication channels and data models, \ac{glosa}-enabled deployments can offer layered \ac{its} services, energy-saving advisories, automated speed control and safety warnings, through a unified architecture.

These interdependencies underscore the value of standardised SPaT/\ac{map} formats, communication protocols and \ac{hmi} designs. A unified human–machine interface can display \ac{glosa} speed bands, \ac{acc} status and safety alerts in a coherent dashboard widget or head-up display. This reduces driver distraction and cognitive load compared to separate interfaces for each service. Furthermore, cloud-based platforms that aggregate \ac{spat} data for \ac{glosa} can simultaneously feed predictive analytics for \ac{ead}, \ac{ecoacc} and \ac{icw}, enabling centralised monitoring and coordinated updates. Such integrated solutions promise compound benefits: fuel and emission reductions that exceed the sum of individual applications, smoother traffic flows that alleviate congestion propagation, and enhanced safety through timely alerts. The joint deployment of \ac{glosa}, \ac{ead}, \ac{acc}/\ac{ecoacc} and safety-oriented \ac{its} functions thus represents a mature, multiservice \ac{its} ecosystem capable of delivering significant gains in sustainability, efficiency and road safety.


\subsubsection{Challenges in Real-World Deployment}
\label{subsubsec:deployment_challenges}

Real-world deployment of \ac{glosa} systems faces four major challenge categories: signal timing synchronisation, communication reliability, privacy and compliance, and economic and scalability considerations.
\mynewline
\textit{Signal Timing Synchronisation} requires accurate alignment between roadside controllers and vehicle on-board units. Clock drift in both traffic signal controllers and \ac{rsu} modules can introduce timing errors that accumulate over hours of operation. Eckhoff et al. \cite{Eckhoff2013} report that a 50\,ms drift in \ac{spat} message timestamps can shift recommended speeds by 2--3\,km/h, causing vehicles to mistime green phases and incur additional stops. Latency in \ac{spat} data transmission further degrades synchronisation: Stahlmann et al. \cite{Stahlmann2018} observed end-to-end delays of up to 200\,ms in IEEE 802.11p field tests, leading to mismatch between predicted and actual phase changes. Uncertainty in signal prediction accuracy also stems from adaptive traffic controllers that modify phase durations in response to local traffic conditions. Such dynamic adjustments, while beneficial for throughput, undermine \ac{glosa} algorithms that rely on static phase plans. To mitigate these issues, deployment must include periodic clock synchronisation (e.g.\ via GPS) and robust phase prediction algorithms that account for adaptive controller behaviour.
\mynewline
\textit{Communication Reliability} poses a second key challenge. Packet loss and intermittent connectivity in \ac{v2i} links can cause advisory gaps or stale data. Martínez et al. \cite{Martinez2020} compare low-cost technologies for \ac{v2i} and find that nRF24L01+ modules exhibit packet error rates of 10\% at 500\,m in urban environments, whereas ZigBee and Wi-Fi maintain errors below 2\%. In mixed-technology deployments, such heterogeneity exacerbates reliability issues. Katsaros et al. \cite{Katsaros2011} demonstrate in simulation that 30\% packet loss reduces \ac{glosa} fuel savings by half and increases stop counts by 15\%. Latency variability also impacts driver comfort: delays above 100\,ms introduce jerky speed advisories and reduce compliance. Field trials by Stahlmann et al. \cite{Stahlmann2018} show that communication black spots near tall buildings can create advisory shadows up to 100\,m long, forcing vehicles to revert to conservative driving profiles and negating environmental benefits. To improve reliability, multi-path communication (e.g.\ cellular backup), adaptive message rates, and error-resilient coding are essential.
\mynewline
\textit{Privacy Considerations} constitute a third challenge in public acceptance and regulatory compliance. \ac{spat} and \ac{map} messages include precise vehicle position reports and timestamps, potentially enabling tracking of individual drivers. Yoshizawa et al. \cite{Yoshizawa2022} identify that current \ac{v2x} standards lack uniform privacy protections, with inconsistent pseudonym change strategies across deployments. Without adequate safeguards, adversaries can correlate advisory requests to specific vehicles and infer travel patterns. Driver compliance also depends on trust: surveys indicate that 45\% of drivers reject \ac{glosa} advisories when uncertain about data security. \cite{Application2011} To address privacy, deployments must implement pseudonymous credentials with frequent rotation, end-to-end encryption of \ac{spat} channels, and clear data retention policies compliant with \ac{gdpr} and other regulations.
\mynewline
\textit{Economic and Scalability} factors finally limit widespread adoption. Infrastructure investments include installation of \ac{rsu} hardware, integration with traffic controllers, and cloud platforms for data aggregation. Otto et al. \cite{Otto2023} present a planning methodology for \ac{c-its} infrastructure and report per-intersection costs of 15\,000--25\,000\,\euro{} for \ac{rsu} installation and configuration. Operating costs include maintenance, software updates and licensing. Stakeholder coordination among municipalities, road operators and vehicle manufacturers adds complexity and delays. Eckhoff et al. \cite{Eckhoff2013} show that benefits of \ac{glosa} diminish at penetration rates below 17\%, raising questions about return on investment in early-market phases. Scalability also strongly depends on infrastructure compatibility; existing traffic controllers often lack standardized communication interfaces, necessitating costly retrofitting and integration efforts during deployment. To overcome these barriers, phased rollouts focusing on high-impact corridors, shared infrastructure across \ac{its} services and public–private partnerships are recommended.
\mynewline
Addressing these challenges requires a holistic approach that combines robust technical solutions --- such as clock synchronisation, multi-path communication and privacy-preserving protocols --- with economic strategies for cost sharing and phased deployment. Only by solving synchronisation, reliability, privacy, and scalability issues together can \ac{glosa} systems fulfil their promise of reducing emissions, improving traffic flow and enhancing safety at scale.


\subsection{Future Directions and Emerging Trends}
\label{subsec:future_trends}

Building on the advances and challenges discussed in previous sections, this subsection outlines three key avenues for future research and deployment in \ac{glosa} systems.
\mynewline
\acp{cav} promise to substantially boost \ac{glosa} performance by embedding advisory logic directly within the vehicle’s automated driving stack. Early field experiments demonstrate that integrating eco-approach algorithms with connected vehicles can yield significant reductions in fuel consumption and stop events. For example, a study of an enhanced eco-approach application in a BMW research vehicle showed consistent phase-aware speed recommendations, computed on a cloud–vehicle link, leading to fuel savings and smoother traffic flow. \cite{Xia2013} Embedding \ac{glosa} advisories into higher levels of automation would eliminate human reaction delays, ensure exact compliance, and allow coordinated platoon control under mixed-traffic conditions.
\mynewline
A natural extension of \ac{cav} integration is the adoption of cloud-based predictive services. Offloading computational tasks from on-board units and roadside infrastructure to scalable server farms enables fusion of \ac{spat} data with floating-car data, weather, and incident reports. In a field test, a BMW 535i sedan communicated over 4G/LTE with an AWS EC2 instance, which fused real-time \ac{spat} messages and vehicle telemetry to generate optimal speed trajectories with minimal latency. \cite{Cloud2016} Such architectures support dynamic preview beyond local broadcast range, rapid software updates, and network-wide analytics. Key research questions include ensuring secure, low-latency pipelines, fault-tolerant server failover, and cost-effective scaling for metropolitan deployments.
\mynewline
Finally, artificial intelligence and advanced data analytics hold promise for dynamic optimisation of both traffic signals and advisory systems. Machine learning models trained on floating-car data, incorporating statistical features such as speed distribution skewness and kurtosis, can predict short-term traffic flow and speed with high accuracy. A neural-network–based framework applied to Thessaloniki taxi data achieved robust speed forecasts by combining probe data with link‐level features, demonstrating the feasibility of anticipatory \ac{glosa} adjustments in urban contexts. \cite{Aifadopoulou2019} Future work should explore reinforcement learning and multi-objective optimisation to self-tune advisory parameters and signal timings in response to stochastic demand, incidents, and seasonal variations.
\mynewline
These emerging directions --- \ac{cav} integration, cloud-based predictive services, and AI-driven optimisation --- point toward a multi-layered \ac{its} ecosystem in which \ac{glosa} forms a core component of a broader, cloud-empowered urban mobility infrastructure.
