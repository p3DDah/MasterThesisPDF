\section{Execution Protocol and Parameterization}
\label{sec:exec_protocol}

The execution protocol translates the conceptual design into a reproducible set of simulation runs that sample all relevant boundary conditions while keeping computational cost within practical limits. Reproducibility is guaranteed by fixing the random seed to \texttt{12345}; thus, every pseudo-random operation, including route choice, lane assignment, and vehicle parameter draws, yields identical sequences across repeated executions.
\mynewline
Each run evaluates a unique combination of algorithm, traffic flow, market penetration, and fuel model, showed in table~\vref{tab:ScenarioMatrix}. The full parameter sweep, detailed below, results in a total of $352$ distinct simulation scenarios.
\mynewline

\begin{enumerate}
    \item \textbf{Controller Algorithm.} Each scenario is run with one of two \ac{glosa} algorithm variants: (i) the \emph{baseline} or \emph{flow-optimised} controller, which targets minimum delay, and (ii) the \emph{eco-driving} controller, which minimises fuel consumption.
    
    \item \textbf{Traffic Flow.} Eight demand levels are enforced by modulating the number of inserted vehicles, corresponding to flows of $69$, $138$, $346$, $692$, $1385$, $2077$, $2769$, and $3462\unit{\veh\per\hour}$, respectively. This range covers traffic states from free-flow to oversaturated conditions.
    
    \item \textbf{Market Penetration.} The \ac{glosa} \ac{mpr} is varied from $0\%$ to $100\%$ in $10$-percentage-point increments. A Bernoulli draw at spawn time classifies each vehicle as equipped or non-equipped.

    \item \textbf{Fuel Model.} To assess sensitivity to the underlying emission calculations, each scenario is executed twice: once using the polynomial-based \ac{hbefa}4 model and once using the more detailed, physics-based \textsc{PHEMlight}5 model, as specified in Section~\ref{subsubsec:detailed_emission_models}.

    \item \textbf{Communication range.} Prior work by Lenz \cite{Lenz2024} found no statistically significant difference between $500~\unit{\meter}$ and $1000~\unit{\meter}$ communication horizons; therefore only the $500~\unit{\meter}$ radius is tested. The advisory is sent whenever an equipped vehicle’s distance to the stop line is $\gls{dup} \leq 500~\unit{\meter}$ upstream (i.e., up to $500~\unit{\meter}$ before the stop line) and $\gls{ddown} \leq 200~\unit{\meter}$ downstream (i.e., up to $200~\unit{\meter}$ after the stop line).
\end{enumerate}

\begin{table}[htb]
  \centering
  \caption[Experimental Design Matrix]{Primary factors of the experimental design. The full factorial combination of these parameters results in a total of 352 unique simulation runs.}
  \label{tab:ScenarioMatrix}
  \begin{tabular}{l c c c}
    \toprule
    \textbf{Factor} & \textbf{Levels} & \textbf{Values} & \textbf{Unit} \\
    \midrule
    Controller Algorithm & $2$ & Flow-Optimised, Eco-Driving & - \\
    Traffic Flow ($q$) & $8$ & $69-3462$ & $\unit{\veh\per\hour}$ \\
    Market Penetration ($p$) & $11$ & $0-100$ (in steps of $10$) & $\%$ \\
    Fuel Model & $2$ & HBEFA4, PHEMlight5 & - \\
    \midrule
    \multicolumn{2}{l}{\textbf{Total Scenarios}} & \multicolumn{2}{c}{$2 \times 8 \times 11 \times 2 = 352$} \\
    \bottomrule
  \end{tabular}
\end{table}

All other simulation and controller settings, detailed in Table~\ref{tab:ConstantSimParams}, were held constant across the entire experimental sweep. This approach ensures a controlled comparison by isolating the impact of the primary factors defined in the experimental design (Table~\ref{tab:ScenarioMatrix}). The fixed parameters govern the core logic of the optimisation routine and the physical assumptions of the simulation environment.
\mynewline

\begin{table}[htb]
  \centering
  \caption[Constant Simulation and Controller Parameters]{Constant parameters applied across all simulation scenarios to ensure consistency and isolate the effects of the independent variables.}
  \label{tab:ConstantSimParams}
  \resizebox{\textwidth}{!}{%
    \begin{tabular}{l l c l}
      \toprule
      \textbf{Parameter} & \textbf{Symbol} & \textbf{Value} & \textbf{Description} \\
      \midrule
      Signal Switch Time Buffer & - & $2.1\unit{s}$ & Fixed safety time to account for signal actuation delays. \\
      Optimizer Slack Time & \gls{tslack} & $[0.1, 5]\unit{s}$ & Uniformly sampled uncertainty in residual phase duration. \\
      Downstream Horizon & $\gls{ddown}$ & $200\unit{m}$ & Fixed recovery distance for the post-junction trajectory segment. \\
      Yellow Phase Buffer & \gls{epsyellow} & $0.5\unit{s}$ & Safety margin to prevent advising trajectories through an amber phase. \\
      Optimizer Search Step & $\gls{deltaa}$ & $0.01\unit{m\,s^{-2}}$ & Numerical search increment for the acceleration line search. \\
      Terminal Speed Weight & $\gls{alphaspeed}$ & $1.0$ & Weighting factor for the terminal speed penalty in the cost function. \\
      Arrival Time Weight & $\gls{alphatime}$ & $0.5$ & Weighting factor for the arrival time penalty in the cost function. \\
      \bottomrule
    \end{tabular}%
  }
\end{table}

A key methodological choice was to disable dynamic queue estimation in the controller. The queue length was forcibly set to zero in all optimiser calls to isolate the pure speed-advice effects from the confounding variable of queue-prediction accuracy. This ensures that the observed performance differences are attributable solely to the core logic of the \ac{flow-glosa} versus the \ac{eco-glosa}.
\mynewline
The eco-driving controller's optimiser evaluates potential speed profiles within the advisory range, which extends from $500\unit{m}$ upstream to a downstream horizon of $200\unit{m}$ past the stop line. During this process, the upstream vehicle motion is constrained by the \ac{eidm} car-following envelope, as detailed in Section~\ref{sec:SimEnvironment}. The controller performs a univariate line search over the upstream acceleration ($\gls{aup}$) using the search step ($\gls{deltaa}$) defined in Table~\ref{tab:ConstantSimParams}. Any resulting trajectory that violates the terminal conditions, such as the minimum speed (\gls{vmin}) or maximum comfortable deceleration (\gls{bmax}), is discarded as infeasible.
\mynewline
The resulting dataset from these $352$ runs provides the foundation for all performance metrics analysed in Section~\ref{sec:performance_evaluation}. By controlling every stochastic degree of freedom and generating exhaustive logs, the protocol ensures that the study is fully reproducible using the same configuration files and seed. This transparent and modular design allows future research to build upon this work by simply appending new factors to the existing experimental matrix.