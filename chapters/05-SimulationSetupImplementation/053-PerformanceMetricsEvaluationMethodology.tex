\section{Performance Metrics and Evaluation Methodology}
\label{sec:performance_evaluation}

The evaluation framework is designed to quantify the trade-offs between traffic flow efficiency and the ecological benefits introduced by the two \ac{glosa} controller variants. All performance metrics are derived from the per-vehicle trajectory and emission logs generated by \ac{sumo}. As each simulation scenario is executed once with a fixed random seed, the outcomes are deterministic; therefore, statistical confidence intervals are not applicable.

\paragraph{Microscopic Traffic Efficiency.}
The quality of longitudinal driving behaviour and vehicle-level efficiency is quantified using four primary indicators:

\begin{enumerate}[label=\textbf{(\roman*)}]
    \item \textbf{Stops per Vehicle:} The stop frequency, \gls{nstop}, is defined as the number of times a vehicle's speed drops below a threshold of $2\unit{\metre\per\second}$ after having been above it. This mirrors the empirical definition used in field studies. For a single vehicle $i$ over $K$ time steps, this is calculated as:
    \begin{equation}
        N_{\mathrm{stop},i} = \sum_{k=2}^{K}{\mathbbm{1}}(v_{i,k} < 2 \land v_{i,k-1} \ge 2)
    \end{equation}
    The scenario average, $\bar{N}_{\mathrm{stop}}$, is the mean of this value over all $N$ vehicles in the simulation.

    \item \textbf{Mean Vehicle Speed:} The average speed, $\bar{v}$, is computed over all vehicles and all time steps to provide a measure of overall network mobility:
    \begin{equation}
        \bar{v} = \frac{1}{NK}\sum_{i=1}^{N}\sum_{k=1}^{K} v_{i,k}
    \end{equation}

    \item \textbf{Mean Absolute Acceleration:} As a proxy for passenger comfort, powertrain strain, and non-exhaust emissions (e.g., tyre and brake wear), the mean absolute acceleration is calculated as:
    \begin{equation}
        \overline{|a|} = \frac{1}{NK}\sum_{i=1}^{N}\sum_{k=1}^{K} |a_{i,k}|
    \end{equation}
\end{enumerate}

\paragraph{Macroscopic Throughput.}
Intersection throughput is measured by the number of vehicles successfully exiting the evaluation segment per signal cycle. For a cycle of length $T_{\mathrm{cycle}}=120\unit{\second}$ with two distinct green intervals, $I_{1}=[2,35]\unit{\second}$ and $I_{2}=[62,96]\unit{\second}$, the number of vehicles passing in cycle $c$ is:
\begin{equation}
    N_{c} = \#\lbrace i : t_{i,\mathrm{exit}} \in I_{1} \cup I_{2} \rbrace
\end{equation}
where $t_{i,\mathrm{exit}}$ is the exit timestamp of vehicle $i$. The average per-cycle throughput over all $C$ cycles is then $\overline{N}_{\mathrm{cycle}} = \frac{1}{C}\sum_{c=1}^{C} N_{c}$.

\paragraph{Energy Demand and Pollutant Emissions.}
The environmental impact is characterised by per-vehicle pollutant rates, with \ac{co2} serving as the primary metric for fuel consumption. Emissions are calculated over the $1.2\unit{\kilo\meter}$ evaluation corridor. The instantaneous emission rates ($\dot{m}$) from the simulation log are integrated over time and normalised by distance. For a vehicle $i$, the emission rate $e_i$ in grams per kilometre is:
\begin{equation}
    e_{i} = \frac{1}{1000 \cdot L} \sum_{k=s_i}^{e_i} \dot{m}^{\mathrm{CO_2}}_{k} \cdot \Delta t, \quad \text{where } L = 1.2\,\mathrm{km}, \Delta t = 0.1\,\mathrm{s}
\end{equation}
The same formulation is applied for \ac{nox} and \ac{pm} using their respective logged rates. Both the HBEFA4 and PHEMlight5 emission models are processed in parallel for each scenario to ensure that any model-specific biases are applied consistently.

\paragraph{Relative performance baseline.}
Each metric $y$ is expressed as a percentage deviation from the flow‐matched, 0\% penetration scenario:

\begin{equation}
    \Delta y(p) = \frac{y(p) - y(0)}{y(0)} \times 100\%.
\end{equation}

Separate traces $\Delta y_{\mathrm{flow}}$ and $\Delta y_{\mathrm{eco}}$ are plotted for the flow-optimised and eco-driving controllers, respectively. Negative values indicate an improvement.  

\paragraph{Break-even penetration rate.}
For each flow level the critical \ac{mpr} $p^{\star}$ is defined by the solution of
\begin{equation}
    \Delta y_{\mathrm{flow}}(p^{\star})
    \;=\;
    \Delta y_{\mathrm{eco}}(p^{\star}),
\end{equation}
where $\Delta y_{\mathrm{flow}}$ and $\Delta y_{\mathrm{eco}}$ are the percentage‐change curves for the flow-optimised and eco-driving controllers, respectively. Rather than simple linear interpolation, we fit continuous cubic splines $S_{\mathrm{flow}}(p)$ and $S_{\mathrm{eco}}(p)$ through the discrete penetration points $\{p_{k},\,\Delta y(p_{k})\}$. Then $p^{\star}$ is obtained by numerically solving
\begin{equation}
    S_{\mathrm{flow}}(p)\;-\;S_{\mathrm{eco}}(p)\;=\;0
    \quad\text{for}\quad p\in[0,100]\%.
\end{equation}
The root‐finding (e.g.\ Brent’s method) ensures sub‐percent accuracy. We compute one $p^{\star}$ per demand level; the collection of eight values is summarised by its median and range.  

\paragraph{Computational viability.}
Feasibility is assessed by recording the wall‐clock time per scenario via \ac{traci}. Let $T_{\mathrm{traci,0}}$ and $T_{\mathrm{traci,eco}}$ be the runtimes for the baseline and eco‐driving controllers, respectively, and let $T_{\mathrm{sumo,0}}$ denote the pure \ac{sumo} execution time without any external interface. We define the Python‐interface overhead as
\begin{equation}
    \phi \;=\; \frac{T_{\mathrm{traci,0}}}{T_{\mathrm{sumo,0}}},
\end{equation}
which quantifies the relative slowdown introduced by the \ac{traci} loop. The incremental cost of the eco‐driving optimisation is then
\begin{equation}
    \delta_{\mathrm{eco}} 
    \;=\; \frac{T_{\mathrm{traci,eco}} - T_{\mathrm{traci,0}}}{T_{\mathrm{traci,0}}},
\end{equation}
expressing the extra runtime as a fraction of the baseline. Both absolute runtimes (in $s \cdot h^{-1}$) and relative overheads (in \%) are reported to balance environmental gains against computational expense.

\paragraph{Visualisation.}
Results are conveyed using a harmonised set of graphical summaries that balance clarity and detail. 
\begin{enumerate}[label=\textbf{(\alph*)},leftmargin=*]
    \item Line charts depict mean speed $\bar v$, mean absolute acceleration $\overline{|a|}$, average stops per vehicle $\bar N_{\mathrm{stop}}$, per‐route travel time $t_{\mathrm{route}}$ and per‐vehicle CO$_2$ emissions as functions of market penetration. Each demand level occupies its panel to facilitate direct comparisons across flows.
    \item Heat maps illustrate the percentage change in CO$_2$, $\Delta\mathrm{CO}_{2}$, over the two‐dimensional grid defined by traffic flow and penetration. This view highlights the region where eco‐driving achieves net environmental benefit.
    \item Bar plots summarise computational overheads by displaying the interface factor $\phi$ and the eco‐optimisation cost $\delta_{\mathrm{eco}}$ for each scenario, making it easy to weigh runtime against performance gains.
    \item Break‐even penetration rates $p^{\star}$ are marked as vertical dashed lines on the CO$_2$ line plots, directly annotating the threshold at which eco‐driving and flow‐optimised controllers perform equally.
\end{enumerate}
No error bars are shown, since each point is derived from a single deterministic run rather than a statistical ensemble.  

\paragraph{Interpretation focus.}
The ensuing discussion concentrates on two aspects: (i) the absolute and relative reduction of \(\mathrm{CO_{2}}\) emissions achieved by the eco-driving algorithm, and (ii) the penetration–demand corridor in which this reduction outweighs any throughput penalty. By presenting results normalised to the baseline and indexed by market penetration, the methodology yields transparent, scalable insights that remain valid even if additional algorithms or network geometries are appended in future work.