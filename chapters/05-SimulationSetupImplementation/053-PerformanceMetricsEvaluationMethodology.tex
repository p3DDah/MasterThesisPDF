\section{Performance Metrics and Evaluation Methodology}
\label{sec:performance_evaluation}

The assessment framework is designed to expose the trade-off between smoother traffic flow and ecological benefit introduced by the two \ac{glosa} variants. All quantities are derived from the trajectory and emission logs produced by \ac{sumo}. Each scenario is executed exactly once with a fixed random seed, yielding a single deterministic outcome; therefore, statistical confidence intervals are not applicable.

\paragraph{Microscopic Traffic Efficiency.}
The quality of longitudinal driving behaviour is quantified using four primary indicators:

\begin{enumerate}[label=\textbf{(\roman*)}]
    \item \textbf{Stops per Vehicle:} The stop frequency, \gls{nstop}, is a key metric for traffic disruption. A stop is defined as an event where a vehicle's speed drops below a threshold of $2\unit{\metre\per\second}$ after having been above it. For a single vehicle $i$, the total number of stops is calculated as:
    \begin{equation}
        N_{\mathrm{stop},i} = \sum_{k=2}^{K}{\mathbbm{1}}(v_{i,k} < 2 \land v_{i,k-1} \ge 2)
    \end{equation}
    The scenario average, $\bar{N}_{\mathrm{stop}}$, is the mean of this value over all $N$ vehicles.

    \item \textbf{Mean Vehicle Speed:} The average speed, $\bar{v}$, serves as a measure of overall network mobility and is computed over all vehicles and all recorded time steps:
    \begin{equation}
        \bar{v} = \frac{1}{NK}\sum_{i=1}^{N}\sum_{k=1}^{K} v_{i,k}
    \end{equation}

    \item \textbf{Mean Absolute Acceleration:} As a surrogate for passenger comfort and powertrain strain, the mean absolute acceleration is calculated as:
    \begin{equation}
        \overline{|a|} = \frac{1}{NK}\sum_{i=1}^{N}\sum_{k=1}^{K} |a_{i,k}|
    \end{equation}

    \item \textbf{Mean Absolute Jerk:} To quantify the smoothness of acceleration changes, the mean absolute jerk, \gls{j}, is computed. Lower values indicate smoother, more comfortable driving profiles.
    \begin{equation}
        \overline{|j|} = \frac{1}{NK}\sum_{i=1}^{N}\sum_{k=1}^{K} |j_{i,k}|
    \end{equation}
\end{enumerate}

\paragraph{Macroscopic Throughput.}
Throughput is quantified by the mean number of vehicles crossing the intersection's stop-line detector per signal cycle. For a cycle length of $T_{\mathrm{cycle}}=120\unit{\second}$ with two green intervals, $I_{1}=[2,35]\unit{\second}$ and $I_{2}=[62,96]\unit{\second}$, the number of vehicles $N_{c}$ detected during cycle $c$ is:
\begin{equation}
    N_{c} = \#\lbrace i : t_{i,\mathrm{detection}} \in I_{1} \cup I_{2} \rbrace
\end{equation}
where $t_{i,\mathrm{detection}}$ is the time vehicle $i$ is detected at the stop line. The average per-cycle throughput is then $\overline{N}_{\mathrm{cycle}} = \frac{1}{C}\sum_{c=1}^{C} N_{c}$.

\paragraph{Energy Demand and Pollutant Emissions.}
The environmental impact is characterised by per-vehicle pollutant rates, with \ac{co2} as the primary metric for fuel consumption. Emissions are computed over the $1.2\unit{\kilo\metre}$ evaluation corridor by summing the instantaneous rates in the log, multiplying by the time step $\Delta t = 0.1\unit{\second}$, and normalising by distance:
\begin{equation}
    e_{i} = \frac{1}{1000 \cdot L} \sum_{k=s_i}^{e_i} \dot{m}^{\mathrm{CO_2}}_{k} \cdot \Delta t, \quad \text{where } L = 1.2\,\mathrm{km}
\end{equation}
The same form applies to \ac{nox} emissions. The analysis is performed in parallel for both the HBEFA4 and PHEMlight5 emission models.

\paragraph{Relative Performance and Break-Even Analysis.}
To facilitate comparisons, each metric $y$ is expressed as a percentage deviation from the uncontrolled baseline (0\% \ac{mpr}). A break-even analysis is conducted by directly comparing the performance of the \ac{eco-glosa} and \ac{flow-glosa} controllers at each discrete combination of traffic flow and \ac{mpr}. This identifies which controller provides a superior outcome under specific conditions.

\paragraph{Computational Viability.}
The practical feasibility of each algorithm is assessed by recording and comparing the absolute wall-clock time required to complete each simulation scenario. The incremental cost of the eco-driving algorithm is evaluated by comparing its runtime, $T_{\mathrm{traci,eco}}$, against that of the less complex flow-optimised baseline, $T_{\mathrm{traci,flow}}$.

\paragraph{Visualisation Strategy.}
The results are conveyed using a harmonised set of graphical summaries:
\begin{enumerate}[label=\textbf{(\alph*)},leftmargin=*]
    \item \textbf{Line Charts:} Key metrics such as mean speed, stop frequency, and \ac{co2} emissions are plotted as functions of \ac{mpr}. Each traffic demand level is shown in a separate panel to facilitate direct comparison across different levels of congestion.
    \item \textbf{Heat Maps:} The relative performance difference in \ac{co2} emissions between the two controllers is displayed on a two-dimensional grid of traffic flow versus penetration rate. This provides a clear overview of the operating regimes where each strategy delivers a net environmental benefit.
    \item \textbf{Break-Even Boundary:} On the 2D heat maps, the break-even penetration rate, $p^{\star}$, is traced as a boundary line. The line is solid black where the performance crossover is clear and decisive, and a dashed gray line where the transition is ambiguous or the performance difference is negligible.
\end{enumerate}
As each data point is derived from a single deterministic simulation run, no statistical error bars are presented.

\paragraph{Interpretation Focus.}
The subsequent analysis of results concentrates on two primary objectives: (i) to quantify the absolute and relative reductions in \ac{co2} emissions achieved by the eco-driving algorithm, and (ii) to identify the specific penetration-demand corridors where these ecological benefits can be achieved without significantly degrading traffic throughput. By normalizing all performance metrics against a baseline and indexing them by market penetration, this evaluation methodology yields transparent and scalable insights, ensuring the findings provide a robust foundation for future work involving new algorithms or network geometries.
\mynewline
The remainder of this chapter is dedicated to a detailed discussion of these results, structured by the key performance indicators. Section~\vref{sec:Results_GreenPhaseFlow} begins by evaluating the intersection's macroscopic throughput. The analysis then moves to microscopic traffic efficiency, examining the average vehicle speed in Section~\vref{sec:Results_MeanSpeed} and the vehicle stop frequency in Section~\vref{sec:Results_Stops}. Driving smoothness is assessed in Section~\vref{sec:Results_Smoothness} by analysing both the mean absolute acceleration and mean absolute jerk. Subsequently, Section~\vref{sec:Results_Emissions} details the environmental impact, focusing on \ac{co2} and \ac{nox} emissions. The core trade-offs are synthesised in Section~\vref{sec:Results_BreakEven}, which presents the break-even analysis. Finally, the practical feasibility is evaluated in Section~\vref{sec:Results_Computational} by comparing the computational runtimes of the two algorithms.
