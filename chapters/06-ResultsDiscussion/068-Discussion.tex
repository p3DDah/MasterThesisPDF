\section{Discussion}
\label{sec:Results_Discussion}

This discussion synthesises the extensive observations from the simulation campaign, providing a holistic interpretation of the empirical results. The primary goal is to address the three research questions articulated in Section~\vref{sec:Proposed_Eco_Driving_GLOSA_Algorithm} by integrating quantitative findings from the performance metrics with qualitative insights from trajectory-level analyses. By examining the performance of the \ac{eco-glosa} controller against both the Standard (uncontrolled) and \ac{flow-glosa} benchmarks, it is possible to delineate the precise operating conditions under which each strategy is most effective. The central narrative emerging from this analysis is that of a fundamental trade-off between vehicle-level eco-efficiency and network-level traffic stability, a choice whose balance is critically dependent on traffic density and technology penetration.

\subsection*{Addressing Research Question 1: Fuel Efficiency}
The first research question sought to quantify the extent to which the \ac{eco-glosa} algorithm reduces fuel consumption. The findings reveal a conclusive, bimodal performance profile: the controller's effectiveness is entirely dependent on traffic density, and its myopic design philosophy is both its greatest strength in light traffic and its critical weak point in congested conditions.
\mynewline
In under-saturated conditions, the controller successfully achieves its objective. The core logic, which optimises an individual vehicle's trajectory to avoid stop-start cycles and high-power transient states, is highly effective when traffic is sparse. By shifting acceleration into the engine's most efficient operating bands, this strategy yields significant and consistent \ac{co2} reductions, with benefits reaching up to $7.7\%$ compared to the Standard scenario. This confirms that, in the absence of network-level constraints, a vehicle-centric eco-driving approach is beneficial.
\mynewline
However, the results conclusively show that this narrow-minded optimisation becomes a significant liability as traffic density surpasses a critical threshold. The controller's failure to account for network stability leads it to advise speed profiles that, while optimal for a single vehicle in theory, induce system-wide traffic oscillations in practice. This flawed logic is the direct cause of the observed performance inversion, where the strategy becomes counterproductive and results in significant emission increases, such as the penalty of over $135~\unit{\gram\per\kilo\metre}$ observed at $2769~\unit{\veh\per\hour}$. This demonstrates that a locally optimal energy strategy can be globally detrimental.
\mynewline
This leads to a crucial conclusion regarding controller selection. A direct comparison (Figure~\vref{fig:BE_EcoFlow}) shows that the throughput-aware \ac{flow-glosa} is a more robust choice in nearly all scenarios. The more physically accurate PHEMlight5 model clarifies that while \ac{eco-glosa} holds a distinct advantage in very light traffic (e.g., below $300~\unit{\veh\per\hour}$), its benefits vanish quickly as demand grows, making \ac{flow-glosa} the superior strategy. In the end, these findings underscore that a successful eco-driving controller cannot be myopic; it must integrate high-fidelity powertrain models with network-level traffic stability objectives to prevent locally efficient decisions from causing system-wide failure.

\subsection*{Addressing Research Question 2: Traffic-Flow Effects}
The second research question addressed the impact of the \ac{eco-glosa} controller on network-level traffic performance. The evidence unequivocally concludes that a throughput-oriented strategy, as embodied by \ac{flow-glosa}, systematically outperforms a myopic eco-driving strategy in terms of network throughput, stability, and kinematic smoothness, especially when it matters most: in congested conditions.
\mynewline
The fundamental cause of this divergence lies in the objective mismatch between the controllers. The cost function for the \ac{eco-glosa} controller strictly prioritises the vehicle's fuel consumption, penalising deviation from a calculated optimal speed. Crucially, it omits any term related to network throughput or queue length. Consequently, the optimiser often schedules extended slack times to allow for slower, more gradual approaches. While beneficial for a single vehicle in a free space, this behaviour in dense traffic creates voids in the traffic stream, reducing the vehicle discharge rate, inadvertently promoting queue growth, and forcing following vehicles to brake.
\mynewline
This objective mismatch has significant consequences for traffic flow, particularly under heavy demand. At $2769~\unit{\veh\per\hour}$ and a high penetration rate of $90\%$, the \ac{flow-glosa} controller successfully maintains traffic momentum, achieving a high average speed of $12.46~\unit{\metre\per\second}$ while reducing stops to a negligible $0.035~\unit{\stops\per\veh}$ (HBEFA4). In stark contrast, the \ac{eco-glosa} strategy induces a severe traffic jam under the same conditions. Its average speed collapses to just $2.57~\unit{\metre\per\second}$, and the stop frequency increases to nearly $11.6~\unit{\stops\per\veh}$ (under the PHEMlight5 model). This deficit in throughput is confirmed by the green-phase capacity analysis. At this demand level, the number of vehicles dispatched per green phase under \ac{eco-glosa} falls by nearly $18\%$, whereas \ac{flow-glosa} not only preserves but enhances throughput by up to $17.5\%$ at $3462~\unit{\veh\per\hour}$.
\mynewline
The controllers' differing philosophies also manifest in the driving smoothness metrics. At a moderate demand of $692~\unit{\veh\per\hour}$, the eco-guidance successfully reduces mean acceleration by nearly $19\%$ compared to standard driving. However, the frequent speed revisions required to meet its energy-saving targets can increase mean jerk by up to $8\%$, indicating a less smooth ride. Conversely, \ac{flow-glosa} improves both acceleration and jerk profiles once a high penetration is reached, as it encourages vehicles to form stable platoons that traverse the green window at a near-constant speed. This is a direct result of its singular focus on maintaining traffic momentum, which naturally leads to smoother collective behaviour.

\subsection*{Addressing Research Question 3: Critical Thresholds and Operating Regimes}
The third research question sought to identify the critical thresholds where the benefits and drawbacks of each controller strategy become significant. By synthesising the full suite of performance metrics, including emissions, throughput, speed, and stop frequency, the analysis concludes that the optimal controller choice is dictated by three distinct operating regimes, which are a function of traffic density ($\gls{q}$) and market penetration rate ($\gls{p}$).

\paragraph{The Under-Saturated Regime ($\gls{q} < 700~\unit{\veh\per\hour}$)}
In low-density traffic, the conclusion is that \textit{\ac{eco-glosa} can be an effective strategy for improving fuel efficiency}. The analysis of the more physically accurate PHEMlight5 model shows that \ac{eco-glosa} consistently outperforms the Standard scenario, achieving a peak \ac{co2} reduction of $+12.03~\unit{\gram\per\kilo\metre}$. When compared to \ac{flow-glosa}, it also holds an advantage in this very light traffic. Crucially, in this regime, the negative impact of \ac{eco-glosa} on key traffic flow metrics is minimal. This combination of factors makes it a low-risk, high-reward strategy in light traffic conditions where network stability is not a concern.

\paragraph{The Transition Regime ($700 \le q \le 2300~\unit{\veh\per\hour}$)}
This intermediate range represents a critical and unstable \enquote{tipping point} where the analysis leads to the conclusion that \textit{\ac{flow-glosa} becomes the more robust and reliable choice}. The myopic, fuel-saving logic of the \ac{eco-glosa} controller becomes a significant liability here. As established in the previous sections, its attempts to advise slower speeds can induce stop-and-go waves that negate any potential fuel savings. The break-even analysis further confirms this, showing that any emission advantage \ac{eco-glosa} has over \ac{flow-glosa} vanishes entirely within this regime. Given the severe emission penalties associated with failure, the risk of deploying \ac{eco-glosa} is high, making the throughput-focused \ac{flow-glosa} the best choice.

\paragraph{The Saturated Regime ($q > 2300~\unit{\veh\per\hour}$)}
In heavy and over-saturated traffic, the findings support the conclusion that \textit{\ac{flow-glosa} is unequivocally the dominant and only effective strategy}. The primary challenge in this regime is not optimising individual vehicle efficiency but preventing total network collapse. The \ac{eco-glosa} controller consistently produces congestion, increasing \ac{co2} emissions by up to $41\%$ and accumulating nearly $18~\unit{\stops\per\veh}$. In significant contrast, \ac{flow-glosa}, at a penetration of $80\%$ or higher, successfully prevents the onset of severe gridlock. This intervention maintains free-flow speeds and, by eliminating the highly inefficient stop-start cycles, delivers profound \ac{co2} reductions of up to $66\%$. In this regime, maximising throughput is conclusively the most effective ecological strategy.

\paragraph{Final Answer to Research Question 3}
In conclusion, there is no single superior controller; the optimal strategy is dictated by the real-time traffic state. The critical thresholds are not fixed points but rather dynamic boundaries between three core operating regimes. For demands below approximately $700~\unit{\veh\per\hour}$, \ac{eco-glosa} can be deployed to achieve moderate energy savings with minimal risk to traffic flow. For demands above this threshold, the risk of \ac{eco-glosa} inducing congestion makes the throughput-focused \ac{flow-glosa} the most efficient and reliable choice. Finally, for any corridor operating near or above capacity ($q > 2300~\unit{\veh\per\hour}$), \ac{flow-glosa} is essential for preventing gridlock and yielding the largest environmental co-benefits.

\subsection*{Supplementary Observations and Practical Considerations}
Beyond the primary performance metrics, two practical considerations emerge from the analysis: the substantial computational burden of the \ac{eco-glosa} controller and the critical influence of the chosen emission model on the results.
\mynewline
First, the computational cost of the \ac{eco-glosa} optimiser is significant. In the most demanding scenario of $3462~\unit{\veh\per\hour}$ with full penetration, a simulation run under the PHEMlight5 model requires over $250,000~\unit{\second}$ of computation time. This is more than $32$ times longer than the $7,714~\unit{\second}$ needed for the simpler \ac{flow-glosa} logic under the same conditions (Table~\vref{tab:CompViability_TraciComparison}). This massive overhead, which scales with the number of advisory updates and engine-map lookups, indicates that real-world, online deployment of such a detailed optimisation algorithm would necessitate either dedicated hardware acceleration or the use of computationally cheaper surrogate models to approximate the fuel consumption function.
\mynewline
Second, the choice of emission back-end model critically influences the strategic conclusions that can be drawn. The high-fidelity PHEMlight5 model reveals a well-defined, albeit small, operational window where the \ac{eco-glosa} controller is beneficial compared to \ac{flow-glosa}. As seen in the break-even charts (Figure~\vref{fig:BE_EcoFlow_PHEM}), its performance boundaries are clear, providing a predictable understanding of where the strategy succeeds and where it fails. By comparison, the performance map generated by the HBEFA4 model is extremely unstable and fragmented (Figure~\vref{fig:BE_EcoFlow_HBEFA4}). Its relative insensitivity to transient driving states results in an unpredictable mix of benefits and penalties, which makes a clear and confident evaluation of the controller's effectiveness far more challenging. This underscores a key methodological insight: for developing and assessing advanced control strategies, a physics-based model like PHEMlight5 is invaluable. While it may reveal a more constrained operating range, the clarity and predictability it provides are essential for defining reliable deployment guidelines.

\subsection*{Limitations of the Study}
For academic rigour, it is essential to acknowledge the limitations of this work. The findings should be interpreted in the context of the following methodological constraints:
\begin{enumerate}
    \item \textbf{Network Scope:} The analysis is confined to a single, isolated intersection corridor with a length of $1.2~\unit{\kilo\metre}$. Consequently, network-level phenomena such as multi-junction signal coordination and queue spill-back from downstream bottlenecks are not modelled.

    \item \textbf{Fleet Homogeneity:} The study assumes a homogenous powertrain fleet. It does not capture the complex dynamic and emission interactions that would occur in a mixed fleet of passenger cars, heavy-duty vehicles, and hybrid or electric vehicles.

    \item \textbf{Ideal Communications:} An ideal \ac{v2i} communication channel is assumed, with no modelling of real-world factors such as data latency, packet loss, or communication range limitations, which could affect controller performance.

    \item \textbf{Driver Compliance:} The model presupposes that all drivers comply perfectly with the issued advisories. Real-world driver behaviour would likely involve probabilistic compliance, which would effectively alter the operational market penetration rate.

    \item \textbf{Static Signal Control:} The analysis is based on a static, fixed-time signal plan. The results may differ in corridors that employ modern adaptive traffic signal control systems, which dynamically adjust phasing in response to traffic demand.
\end{enumerate}

\subsection*{Conclusion}
In aggregate, this research provides a nuanced and data-driven answer to the question of how to best leverage \ac{glosa} systems for environmental and traffic benefits. The primary conclusion is that a \enquote{one-size-fits-all} approach is suboptimal. The optimal controller strategy is not fixed but is instead a function of the real-time traffic state, with a clear trade-off emerging between vehicle-level eco-efficiency and network-level stability.
\mynewline
The \ac{eco-glosa} strategy demonstrates this trade-off clearly. It yields significant fuel savings of up to $7.7\%$ in low-density conditions by smoothing driving profiles and avoiding unnecessary stops. However, its effectiveness erodes rapidly with increasing demand. The controller's myopic focus on per-vehicle optimisation induces system-level instabilities, leading to increased congestion and, paradoxically, a surge in emissions by over $160\%$ in some scenarios. In contrast, the \ac{flow-glosa} strategy, by synchronising vehicle platoons, proves to be a powerful tool for preventing queue growth and maintaining free-flow conditions. In doing so, it delivers profound emission reductions of over $66\%$ in saturated regimes, demonstrating that in congested cities, the most effective green driving strategy is one that keeps traffic moving. These performance characteristics are framed by practical considerations, including the substantial computational overhead of the \ac{eco-glosa} algorithm, which can be over $30\times$ slower than \ac{flow-glosa}, and the critical influence of the emission model, where the higher fidelity of PHEMlight5 is essential for accurately revealing the controller's failure modes in congestion.
\mynewline
In direct answer to the research questions, this thesis finds that: \textit{\vref{rq1}} --- the \ac{eco-glosa} controller's impact on fuel efficiency is inconsistent, offering modest savings in light traffic but causing severe penalties in congestion; \textit{\vref{rq2}} --- the \ac{eco-glosa} controller severely degrades network performance in congestion, whereas the \ac{flow-glosa} strategy successfully preserves throughput; and \textit{\vref{rq3}} --- the optimal controller is determined by traffic density, shifting from \ac{eco-glosa} in light traffic to \ac{flow-glosa} as the only viable strategy in saturated conditions.
\mynewline
Therefore, a key recommendation for practitioners is to transition from static to adaptive, hybrid control systems. Such controllers would dynamically balance energy efficiency and traffic flow objectives using real-time estimates of traffic density and market penetration. These systems could operate in an eco-centric mode during periods of lower demand and smoothly switch to a flow-centric mode as congestion builds. A detailed discussion of these adaptive control strategies and other future work is provided in Chapter~\vref{ch:SummaryOutlook}.