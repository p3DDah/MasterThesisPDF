\section{Discussion}
\label{sec:Results_Discussion}

This discussion synthesises the extensive observations from the simulation campaign detailed in Section~\ref{sec:Proposed_Eco_Driving_GLOSA_Algorithm}, providing a holistic interpretation of the empirical results. The primary goal is to address the three \acp{rq} articulated in Section~\ref{subsec:Research_Questions} by integrating quantitative findings from emission and flow metrics (e.g., Tables~\ref{tab:metrics_summary}, \ref{tab:emissions}) with qualitative insights from trajectory-level analyses (e.g., Figures~\ref{fig:co2_mpr_low}, \ref{fig:flow_comparison}). By examining the performance of the \ac{eco-glosa} controller against both the Standard (uncontrolled) and \ac{flow-glosa} benchmarks, we can delineate the precise operating conditions under which each strategy is most effective. The central narrative emerging from this analysis is that of a fundamental trade-off between vehicle-level eco-efficiency and network-level traffic stability, a dichotomy whose balance is critically dependent on traffic density and technology penetration.

\bigskip
\textbf{Addressing Research Question 1: Fuel Efficiency}

Our first research question sought to quantify the extent to which the \ac{eco-glosa} algorithm reduces fuel consumption. The findings reveal a stark, bimodal performance profile that is highly sensitive to traffic density.
\mynewline
In under-saturated conditions, \ac{eco-glosa} consistently and significantly reduces per-vehicle fuel consumption. Under light demand ($q=69\,\unit{veh/h}$), with a $50\,\%$ \ac{mpr}, the mean \ac{co2} emissions are reduced by a substantial $27\,\%$ relative to Standard driving and by $14\,\%$ compared to \ac{flow-glosa}, as documented in Tables~\ref{tab:vehicle_emissions_comparison}. This considerable saving originates directly from the \ac{dp} controller's core logic, which actively suppresses high-power transient states. The optimal trajectory prescribed by the algorithm effectively shifts necessary acceleration phases into the internal combustion engine's most efficient operating band (around $p_\mathrm{kw}\approx50\,\unit{kW}$ for the modelled passenger car fleet) and, most importantly, avoids energy-intensive stop-start cycles by timing the arrival at the stop line with the green phase. This benefit is not confined to very low flows; it persists up to moderate demand levels ($q\approx692\,\unit{veh/h}$), where fuel usage is still diminished by a notable $18\,\%$ at a $60\,\%$ \ac{mpr}.
\mynewline
However, this paradigm of efficiency dramatically inverts as traffic density surpasses a critical threshold. At higher demand levels ($q > 1385\,\unit{veh/h}$), the \ac{eco-glosa} strategy becomes counterproductive. The controller, in its myopic pursuit of an optimal energy profile for the individual vehicle, frequently revises its speed advisories in response to the emergent queueing dynamics of the surrounding traffic. This leads to oscillatory speed profiles, increased idling losses from unexpected stops, and repeated, inefficient re-accelerations. The result is a net increase in fuel consumption. At a demand of $q=2769\,\unit{veh/h}$ and a relatively low $p=30\,\%$ penetration, vehicles under \ac{eco-glosa} guidance emit an additional $+12\,\unit{g/km}$ of \ac{co2} compared to the Standard scenario. Trajectory plots in Figure~\ref{fig:trajectory_high_demand} reveal the underlying cause: equipped vehicles repeatedly accelerate in an attempt to catch a green window, only to be forced to brake hard by the slower-moving, non-equipped vehicle queue ahead. The energy wasted in these futile manoeuvres far exceeds any potential savings from smooth cruising.
\mynewline
This behaviour means that the break-even penetration rate, $p^\star$, at which \ac{eco-glosa} becomes beneficial, shifts upwards with demand. As shown in the break-even analysis in Figure~\ref{fig:breakeven_eco_vs_standard}, for flows below $q < 700\,\unit{veh/h}$, a penetration of just $p^\star\approx30\,\%$ is sufficient to yield savings. In contrast, at flows approaching $q \approx 2000\,\unit{veh/h}$, the required penetration to overcome the negative externalities rises to $p^\star > 80\,\%$.
\mynewline
Furthermore, the choice of emission model sensitises these trends. The analysis using the physics-based PHEMlight5 model shows that the region of net benefit for \ac{eco-glosa} is narrower than predicted by the polynomial HBEFA4 model. This is because PHEMlight5 more accurately penalises the transient events (hard accelerations and braking) that \ac{eco-glosa} induces in dense traffic, causing the crossover to a net loss to occur earlier, at approximately $q > 1000\,\unit{veh/h}$. This finding underscores the critical need to select or calibrate emission back-end models that faithfully represent the real-world powertrain distributions and their response to dynamic driving conditions.

\bigskip
\textbf{Addressing Research Question 2: Traffic-Flow Effects}

The second research question addressed the impact of \ac{eco-glosa}-equipped vehicles on network-level traffic performance. Here, the evidence unequivocally demonstrates that the \ac{flow-glosa} controller systematically outperforms both the eco-variant and the no-GLOSA Standard in terms of network throughput and kinematic smoothness, especially when it matters most: in congested conditions.
\mynewline
Under heavy demand ($q = 2769\,\unit{veh/h}$) and at a high penetration rate ($p=90\,\%$), \ac{flow-glosa} increases the green-phase discharge rate by an impressive $35\,\%$ relative to the Standard scenario. This intervention is powerful enough to restore average speeds to near free-flow levels ($v\approx10.2\,\unit{m/s}$) and virtually eliminates stops, reducing them to below $0.1\,\unit{stops/veh}$ (Figure~\ref{fig:flow_metrics_high_demand}). In stark contrast, under the same conditions, \ac{eco-glosa} only achieves a mean speed of $v\approx6\,\unit{m/s}$ while inducing $3.6\,\unit{stops/veh}$.
\mynewline
The fundamental cause of this divergence lies in the objective mismatch between the controllers. The cost function $J$ for the \ac{eco-glosa} controller (Equation~\ref{eq:dp_cost_function}) meticulously prioritises the fuel consumption term $f_\mathrm{fuel}(v,a)$ and penalises deviation from a calculated optimal speed. Crucially, it omits any term related to network throughput or queue length. Consequently, the optimiser often schedules extended slack times ($T_\mathrm{slack}$) to allow for slower, more gradual approaches that reduce engine load. While beneficial for a single vehicle in a vacuum, this behaviour in dense traffic creates a void in the traffic stream, inadvertently promoting queue growth and forcing following vehicles to brake.
\mynewline
Acceleration and jerk metrics, proxies for driving comfort and stability, mirror this divergence. At a moderate demand of $q=692\,\unit{veh/h}$ and $p=50\,\%$, the eco-guidance successfully reduces the mean absolute acceleration by $12\,\%$ compared to standard driving. However, as noted previously, the frequent and abrupt changes in speed advice in denser traffic can increase the mean jerk by up to $8\,\%$, indicating a less smooth ride. Conversely, \ac{flow-glosa} smooths both acceleration and jerk profiles once penetration exceeds $p>60\,\%$, as it encourages vehicles to form stable platoons that traverse the green window at near-constant speed. This is a direct result of its singular focus on maintaining traffic momentum.
\mynewline
Finally, green-phase utilisation further highlights the eco-flow trade-offs. At low flows, \ac{eco-glosa} can achieve higher *per-vehicle* green-time utilisation ratios, as vehicles glide slowly and use a large portion of the available green. However, from a network perspective, this is inefficient. At demand levels above $q>1500\,\unit{veh/h}$, the absolute number of vehicles dispatched per green phase under \ac{eco-glosa} declines significantly relative to both the Standard and \ac{flow-glosa} benchmarks (Table~\ref{tab:green_utilisation}). \ac{flow-glosa} maximises this metric by packing as many vehicles as possible through the intersection during each cycle.

\bigskip
\textbf{Addressing Research Question 3: Critical Thresholds and Operating Regimes}

The third research question asked for the identification of critical thresholds where the benefits and drawbacks of each system become statistically significant. By mapping the performance metrics across the entire parameter space of traffic density ($q$) and penetration rate ($p$), we can define three distinct operating regimes.

\begin{itemize}
    \item \textbf{Under-saturated Regime ($q < 700\,\unit{veh/h}$):} In this low-density environment, \ac{eco-glosa} is the superior strategy for fuel savings. It consistently outperforms both \ac{flow-glosa} and the baseline once penetration surpasses a modest $p > 30\,\%$. The fuel savings plateau beyond $p > 60\,\%$, suggesting diminishing returns at very high penetration. In this regime, the impact on traffic flow metrics like speed and stop frequency is statistically insignificant, meaning the environmental benefits are achieved without any discernible negative impact on traffic operations.

    \item \textbf{Transition Regime ($700 \le q \le 2300\,\unit{veh/h}$):} This intermediate regime is characterised by a complex interplay of effects and the emergence of a \enquote{tipping point}. A critical boundary, visible in Figure~\ref{fig:breakeven_eco_vs_flow}, appears at a penetration rate $p^\star$ that shifts from approximately $60\,\%$ to $90\,\%$ as density increases. Below this penetration line, the negative externalities of \ac{eco-glosa} dominate, and the \ac{flow-glosa} strategy yields superior outcomes in both flow and, consequently, emissions. Above this line, if a very high majority of vehicles cooperate, the eco-strategy can regain a fuel-efficiency advantage, provided the emergent oscillations remain moderate. This regime is the most challenging for static deployment, as the optimal choice of controller depends heavily on achieving a near-total \ac{mpr}.

    \item \textbf{Saturated Regime ($q > 2300\,\unit{veh/h}$):} In heavy and over-saturated traffic, the \ac{flow-glosa} strategy dominates unequivocally. The primary challenge here is not fine-tuning energy usage but preventing total gridlock. At a penetration of $p > 80\,\%$, \ac{flow-glosa} actively dissolves nascent traffic jams and, by doing so, cuts \ac{co2} emissions by a remarkable $-64\,\%$ relative to the congested Standard scenario. In stark contrast, the \ac{eco-glosa} controller exacerbates the congestion, increasing emissions by $+88\,\%$ under the same conditions (Table~\ref{tab:emissions_saturated}). In this regime, optimising for flow is the most effective ecological strategy.
\end{itemize}

These regime boundaries are not arbitrary; they align with the physical properties of the traffic stream, particularly the critical density $k_\mathrm{crit}=q/v_\mathrm{free}$ and the inherent frequency of stop-and-go waves described by the \ac{eidm} car-following model.

\bigskip
\textbf{Supplementary Observations and Practical Considerations}

Several additional observations provide practical context. First, the computational burden of the \ac{eco-glosa} optimiser is substantial. A full-penetration run in a high-demand scenario requires up to $415\,\unit{s}$ of computation time under the PHEMlight5 model, compared to just $154\,\unit{s}$ for the simpler \ac{flow-glosa} logic (Table~\ref{tab:runtimes}). This overhead, which scales with the number of advisory updates and engine-map lookups, suggests that real-world, on-line deployment would necessitate hardware acceleration or the use of computationally cheaper surrogate models to approximate the fuel consumption function $f(v,a)$.
\mynewline
Second, the choice of emission back-end model significantly influences quantitative outcomes: using the more transient-sensitive PHEMlight5 backend reduces the net-benefit penetration window by approximately $15\,\%$ and shifts break-even thresholds by $10$–$15\,\%$ compared to HBEFA, underscoring the need for careful back-end calibration to match the target fleet composition.  

\bigskip
\textbf{Limitations of the Study}

For academic rigour, it is essential to acknowledge the limitations of this work. The study is confined to a single, isolated intersection corridor of $1.2\,\unit{km}$ in length. Network effects such as multi-junction signal coordination and spill-back from downstream bottlenecks are not modelled. The powertrain fleet is homogenous, neglecting the complex interactions between passenger cars, heavy-duty vehicles, and electric or hybrid vehicles. Furthermore, ideal \ac{v2i} communication is assumed, with no latency or packet loss. Real-world driver behaviour may also include probabilistic compliance with advisories, which would effectively reduce the penetration rate. Finally, the analysis is based on a static, fixed-time signal plan, whereas modern urban corridors increasingly use adaptive traffic signal control.

\bigskip
\textbf{Conclusion}

In aggregate, this research provides a nuanced and data-driven answer to the question of how to best leverage \ac{glosa} systems for environmental and traffic benefits. The primary conclusion is that a "one-size-fits-all" approach is suboptimal and potentially counterproductive.
\mynewline
The \ac{eco-glosa} strategy yields robust fuel savings of up to $27\,\%$ under low-density conditions, requiring only about a third of vehicles to be equipped to see benefits. However, its effectiveness erodes rapidly with increasing demand, as the egoistic, vehicle-level optimisation induces system-level instabilities that lead to increased congestion and emissions. In contrast, the \ac{flow-glosa} strategy, by synchronising vehicle platoons with green phases, proves to be a powerful tool for suppressing queue growth and restoring free-flow conditions. In doing so, it delivers profound emission reductions of up to $-64\,\%$ in saturated regimes, demonstrating that in congested cities, the most effective green driving strategy is one that keeps traffic moving.
\mynewline
Therefore, a key recommendation for practitioners is to transition from static to adaptive, hybrid control systems. Such adaptive controllers would dynamically balance energy efficiency and traffic flow objectives using real-time estimates of traffic density ($q$) and penetration rate ($p$). These systems could operate in an eco-centric mode during periods of lower demand, and smoothly switch to a flow-centric mode as congestion builds during peak times. A detailed discussion of these adaptive control strategies, along with considerations for future validation in more complex urban networks, mixed powertrain fleets, and integration with adaptive signal control systems, is provided in the following chapter (Chapter~\ref{ch:SummaryOutlook}).

