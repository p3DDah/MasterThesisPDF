\section{Discussion}
\label{sec:Results_Discussion}

This discussion synthesises the extensive observations from the simulation campaign, providing a holistic interpretation of the empirical results. The primary goal is to address the three research questions articulated in Section~\vref{sec:Proposed_Eco_Driving_GLOSA_Algorithm} by integrating quantitative findings from the performance metrics with qualitative insights from trajectory-level analyses. By examining the performance of the \ac{eco-glosa} controller against both the Standard (uncontrolled) and \ac{flow-glosa} benchmarks, it is possible to delineate the precise operating conditions under which each strategy is most effective. The central narrative emerging from this analysis is that of a fundamental trade-off between vehicle-level eco-efficiency and network-level traffic stability, a dichotomy whose balance is critically dependent on traffic density and technology penetration.

\subsection*{Addressing Research Question 1: Fuel Efficiency}
The first research question sought to quantify the extent to which the \ac{eco-glosa} algorithm reduces fuel consumption. The findings reveal a stark, bimodal performance profile that is highly sensitive to traffic density.
\mynewline
In under-saturated conditions, \ac{eco-glosa} consistently and significantly reduces per-vehicle fuel consumption. Under light demand ($q=69~\unit{\veh\per\hour}$), for instance, the mean \ac{co2} emissions are reduced by a substantial $15.5\%$ relative to Standard driving. This considerable saving originates directly from the \ac{dp} controller's core logic, which actively suppresses high-power transient states. The optimal trajectory prescribed by the algorithm effectively shifts necessary acceleration phases into the engine's most efficient operating band and, most importantly, avoids energy-intensive stop-start cycles by timing the arrival at the intersection with the green phase. This benefit persists up to moderate demand levels, where at $q=692~\unit{\veh\per\hour}$, fuel usage is still diminished by a notable $11.8\%$ at full penetration under the PHEMlight5 model.
\mynewline
However, this paradigm of efficiency dramatically inverts as traffic density surpasses a critical threshold. At higher demand levels ($\gls{q} > 1385~\unit{\veh\per\hour}$), the \ac{eco-glosa} strategy becomes counterproductive. The controller, in its myopic pursuit of an optimal energy profile for the individual vehicle, frequently revises its speed advisories in response to emergent queueing dynamics. This leads to oscillatory speed profiles and repeated, inefficient re-accelerations. At a demand of $2769~\unit{\veh\per\hour}$ and $30\%$ penetration, the strategy increases \ac{co2} emissions by over $35~\unit{\gram\per\kilo\metre}$ compared to the Standard scenario (under PHEMlight5). Trajectory plots reveal the cause: equipped vehicles repeatedly accelerate to catch a green window, only to brake hard due to the slower, non-equipped vehicle queue ahead.
\mynewline
This behaviour means that the break-even penetration rate, $\gls{p}^\star$, at which \ac{eco-glosa} becomes beneficial, shifts upwards with demand. As shown in Figure~\vref{fig:BE_EcoFlow}, for flows below $700~\unit{\veh\per\hour}$, a penetration of just $\gls{p}^\star\approx30\%$ is often sufficient. In contrast, at flows approaching $2000~\unit{\veh\per\hour}$, the required penetration rises to $\gls{p}^\star > 80\%$. Furthermore, the choice of emission model sensitises these trends. The physics-based PHEMlight5 model predicts a narrower region of net benefit for \ac{eco-glosa} than the polynomial HBEFA4 model, as it more accurately penalises the transient events induced by the controller in dense traffic. This underscores the need for back-end models that faithfully represent real-world powertrain dynamics.

\subsection*{Addressing Research Question 2: Traffic-Flow Effects}
The second research question addressed the impact of the \ac{eco-glosa} controller on network-level traffic performance. The evidence unequivocally demonstrates that the \ac{flow-glosa} controller systematically outperforms both the eco-variant and the no-GLOSA Standard in terms of network throughput and kinematic smoothness, especially when it matters most: in congested conditions.
\mynewline
This divergence is most apparent under heavy demand. At $2769~\unit{\veh\per\hour}$ and a high penetration rate of $90\%$, the \ac{flow-glosa} controller successfully maintains traffic flow, achieving a high average speed of $12.46~\unit{\metre\per\second}$ while virtually eliminating stops, reducing them to just $0.07~\unit{\stops\per\veh}$. In stark contrast, under the same conditions, the \ac{eco-glosa} strategy induces a severe traffic jam. Its average speed collapses to just $2.57~\unit{\metre\per\second}$, and the stop frequency explodes to over $23~\unit{\stops\per\veh}$ (under the PHEMlight5 model).
\mynewline
The fundamental cause of this divergence lies in the objective mismatch between the controllers. The cost function for the \ac{eco-glosa} controller meticulously prioritises the vehicle's fuel consumption, penalising deviation from a calculated optimal speed. Crucially, it omits any term related to network throughput or queue length. Consequently, the optimiser often schedules extended slack times to allow for slower, more gradual approaches. While beneficial for a single vehicle in a vacuum, this behaviour in dense traffic creates voids in the traffic stream, inadvertently promoting queue growth and forcing following vehicles to brake.
\mynewline
The acceleration and jerk metrics, which serve as proxies for driving comfort, mirror this divergence. At a moderate demand of $692~\unit{\veh\per\hour}$, the eco-guidance successfully reduces mean acceleration by $12\%$ compared to standard driving. However, the frequent speed revisions in denser traffic can increase mean jerk by up to $8\%$, indicating a less smooth ride. Conversely, \ac{flow-glosa} smooths both acceleration and jerk profiles once penetration exceeds $60\%$, as it encourages vehicles to form stable platoons that traverse the green window at a near-constant speed. This is a direct result of its singular focus on maintaining traffic momentum.
\mynewline
Finally, green‐phase utilisation further highlights the trade‐offs. At low flows, \ac{eco-glosa} can achieve higher per‐vehicle green‐time utilisation ratios as individual vehicles glide more slowly across the intersection. From a network perspective, however, this comes at the expense of overall capacity. At demand levels above $1500~\unit{\veh\per\hour}$, the number of vehicles dispatched per green phase under \ac{eco-glosa} declines relative to both the Standard and \ac{flow-glosa} benchmarks, for example, under PHEMlight5 at $2769~\unit{\veh\per\hour}$ dispatch falls from $46.03$ to $37.83$ vehicles ($–17.9\%$), as confirmed in Table~\vref{tab:GreenPhaseFlow}. In contrast, the \ac{flow-glosa} baseline curves not only preserve but enhance throughput at full penetration, increasing dispatch by up to $17.5\%$ at $3462~\unit{\veh\per\hour}$ (from $49.03$ to $57.60$ vehicles) by design, demonstrating superior green‐phase capacity under heavy demand.

\subsection*{Addressing Research Question 3: Critical Thresholds and Operating Regimes}
The third research question sought to identify the critical thresholds where the benefits and drawbacks of each controller strategy become significant. By synthesising the full suite of performance metrics, including emissions, throughput, speed, and stop frequency, the analysis defines three distinct operating regimes. The optimal controller choice depends critically on these regimes, which are a function of traffic density ($\gls{q}$) and market penetration rate ($\gls{p}$).

\paragraph{The Under-Saturated Regime ($\gls{q} < 700~\unit{\veh\per\hour}$)}
In low-density traffic, \textit{\ac{eco-glosa} is the superior strategy for improving fuel efficiency}. The analysis of the PHEMlight5 model, considered the more physically accurate, shows that \ac{eco-glosa} consistently outperforms both the Standard and \ac{flow-glosa} scenarios once penetration surpasses a modest $30\%$ \ac{mpr}. It achieves a peak \ac{co2} reduction of $+24.06~\unit{\gram\per\kilo\metre}$ compared to an uncontrolled vehicle. The HBEFA4 model also shows benefits, though its predictions are more volatile. Crucially, in this regime, the application of \ac{eco-glosa} does not negatively impact key traffic flow metrics; average speeds and throughput are maintained. This makes it a low-risk, high-reward strategy in light traffic conditions.

\paragraph{The Transition Regime ($700 \le q \le 2300~\unit{\veh\per\hour}$)}
This intermediate range represents a critical and unstable \enquote{tipping point} where \textit{\ac{flow-glosa} becomes the more robust and reliable choice}. The myopic, fuel-saving logic of the \ac{eco-glosa} controller becomes a significant liability here. As seen with the PHEMlight5 model, its attempts to advise slower speeds can actively induce a traffic jam where none existed, causing average speeds to collapse and emissions to increase by as much as $163.38~\unit{\gram\per\kilo\metre}$ compared to the Standard case. The break-even analysis further reveals that the required penetration for \ac{eco-glosa} to outperform \ac{flow-glosa} skyrockets from $30\%$ to over $80\%$ as density increases through this regime. Given the severe emission penalties associated with failure, deploying \ac{eco-glosa} is only advisable if near-universal adoption can be guaranteed.

\paragraph{The Saturated Regime ($q > 2300~\unit{\veh\per\hour}$)}
In heavy and over-saturated traffic, \textit{\ac{flow-glosa} is unequivocally the dominant and only effective strategy}. The primary challenge in this regime is not optimising fuel efficiency but preventing total network collapse. The \ac{eco-glosa} controller consistently exacerbates congestion, increasing \ac{co2} emissions by over $50\%$ and accumulating more than $35~\unit{\stops\per\veh}$. In stark contrast, \ac{flow-glosa}, at a penetration of $80\%$ or higher, actively dissolves traffic jams. This intervention restores free-flow speeds and, by eliminating highly inefficient stop-start cycles, delivers profound \ac{co2} reductions of up to $64\%$. In this regime, maximising throughput is the most effective ecological strategy.

\paragraph{Final Answer to Research Question 3}
In conclusion, there is no single superior controller; the optimal strategy is dictated by the real-time traffic state. The critical thresholds are not fixed points but rather dynamic boundaries between three core operating regimes. For demands below approximately $700~\unit{\veh\per\hour}$, \ac{eco-glosa} can be safely deployed to achieve moderate energy savings. For demands above $2300~\unit{\veh\per\hour}$, \ac{flow-glosa} is essential for resolving gridlock and yielding the largest environmental co-benefits. In the wide and volatile transition range between these thresholds, the risk of \ac{eco-glosa} inducing congestion makes the throughput-focused \ac{flow-glosa} the most prudent and reliable choice for deployment.

\subsection*{Supplementary Observations and Practical Considerations}
Beyond the primary performance metrics, two practical considerations emerge from the analysis: the substantial computational burden of the \ac{eco-glosa} controller and the critical influence of the chosen emission model on the results.
\mynewline
First, the computational cost of the \ac{eco-glosa} optimiser is significant. In the most demanding scenario of $3462~\unit{\veh\per\hour}$ with full penetration, a simulation run under the PHEMlight5 model requires over $250,000~\unit{\second}$ of computation time. This is more than $32$ times longer than the $7,714~\unit{\second}$ needed for the simpler \ac{flow-glosa} logic under the same conditions (Table~\vref{tab:CompViability_TraciComparison}). This massive overhead, which scales with the number of advisory updates and engine-map lookups, indicates that real-world, online deployment of such a detailed optimisation algorithm would necessitate either dedicated hardware acceleration or the use of computationally cheaper surrogate models to approximate the fuel consumption function.
\mynewline
Second, the choice of emission back-end model critically influences the strategic conclusions that can be drawn. The high-fidelity PHEMlight5 model reveals a well-defined, albeit smaller, operational window where the \ac{eco-glosa} controller is beneficial. As seen in the break-even charts (Figure~\vref{fig:BE_EcoFlow_PHEM}), its performance boundaries are clear, providing a predictable understanding of where the strategy succeeds and where it fails. By comparison, the performance map generated by the HBEFA4 model is extremely unstable and fragmented (Figure~\vref{fig:BE_EcoFlow_HBEFA4}). Its relative insensitivity to transient driving states results in an unpredictable mix of benefits and penalties, which makes a clear and confident evaluation of the controller's effectiveness far more challenging. This underscores a key methodological insight: for developing and assessing advanced control strategies, a physics-based model like PHEMlight5 is invaluable. While it may reveal a more constrained operating range, the clarity and predictability it provides are essential for defining reliable deployment guidelines.

\subsection*{Limitations of the Study}
For academic rigour, it is essential to acknowledge the limitations of this work. The findings should be interpreted in the context of the following methodological constraints:
\begin{itemize}
    \item \textbf{Network Scope:} The analysis is confined to a single, isolated intersection corridor with a length of $1.2~\unit{\kilo\metre}$. Consequently, network-level phenomena such as multi-junction signal coordination and queue spill-back from downstream bottlenecks are not modelled.

    \item \textbf{Fleet Homogeneity:} The study assumes a homogenous powertrain fleet. It does not capture the complex dynamic and emission interactions that would occur in a mixed fleet of passenger cars, heavy-duty vehicles, and hybrid or electric vehicles.

    \item \textbf{Ideal Communications:} An ideal \ac{v2i} communication channel is assumed, with no modelling of real-world factors such as data latency, packet loss, or communication range limitations, which could affect controller performance.

    \item \textbf{Driver Compliance:} The model presupposes that all drivers comply perfectly with the issued advisories. Real-world driver behaviour would likely involve probabilistic compliance, which would effectively alter the operational market penetration rate.

    \item \textbf{Static Signal Control:} The analysis is based on a static, fixed-time signal plan. The results may differ in corridors that employ modern adaptive traffic signal control systems, which dynamically adjust phasing in response to traffic demand.
\end{itemize}

\subsection*{Conclusion}
In aggregate, this research provides a nuanced and data-driven answer to the question of how to best leverage \ac{glosa} systems for environmental and traffic benefits. The primary conclusion is that a \enquote{one-size-fits-all} approach is suboptimal. The optimal controller strategy is not fixed but is instead a function of the real-time traffic state, with a clear trade-off emerging between vehicle-level eco-efficiency and network-level stability.
\mynewline
The \ac{eco-glosa} strategy demonstrates this trade-off clearly. It yields robust fuel savings of up to $15.5\%$ in low-density conditions by smoothing driving profiles and avoiding unnecessary stops. However, its effectiveness erodes rapidly with increasing demand. The controller's myopic focus on per-vehicle optimisation induces system-level instabilities, leading to increased congestion and, paradoxically, a surge in emissions by over $160\%$ in some scenarios. In contrast, the \ac{flow-glosa} strategy, by synchronising vehicle platoons, proves to be a powerful tool for suppressing queue growth and restoring free-flow conditions. In doing so, it delivers profound emission reductions of up to $64\%$ in saturated regimes, demonstrating that in congested cities, the most effective green driving strategy is one that keeps traffic moving. These performance characteristics are framed by practical considerations, including the substantial computational overhead of the \ac{eco-glosa} algorithm, which can be over $30\times$ slower than \ac{flow-glosa}, and the critical influence of the emission model, where the higher fidelity of PHEMlight5 is essential for accurately revealing the controller's failure modes in congestion.
\mynewline
In direct answer to the research questions, this thesis finds that: \textit{\vref{rq1}} --- the \ac{eco-glosa} controller's impact on fuel efficiency is dichotomous, offering modest savings in light traffic but causing severe penalties in congestion; \textit{\vref{rq2}} --- the \ac{eco-glosa} controller severely degrades network performance in congestion, whereas the \ac{flow-glosa} strategy successfully preserves throughput; and \textit{\vref{rq3}} --- the optimal controller is determined by traffic density, shifting from \ac{eco-glosa} in light traffic to \ac{flow-glosa} as the only viable strategy in saturated conditions.
\mynewline
Therefore, a key recommendation for practitioners is to transition from static to adaptive, hybrid control systems. Such controllers would dynamically balance energy efficiency and traffic flow objectives using real-time estimates of traffic density and market penetration. These systems could operate in an eco-centric mode during periods of lower demand and smoothly switch to a flow-centric mode as congestion builds. A detailed discussion of these adaptive control strategies and other future work is provided in Chapter~\vref{ch:SummaryOutlook}.