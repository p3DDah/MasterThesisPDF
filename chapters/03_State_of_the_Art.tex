\chapter{State-of-the-Art}
\label{ch:State_of_the_Art}

Motorised road transport emitted nearly \SI{8}{\giga\tonne} of CO\textsubscript{2} in 2022—accounting for over one-third of all end-use emissions. \cite{IEATransport2024}  
Reducing these emissions effectively requires an in-depth understanding of traffic dynamics and energy consumption at the individual vehicle trajectory level.   
Contemporary research thus integrates  
\begin{enumerate}[label=(\roman*)]
  \item high-fidelity microscopic traffic simulation,
  \item optimisation frameworks such as \ac{glosa},
  \item data-driven or learning-based control strategies.
\end{enumerate}
This chapter reviews the fundamental components of that pipeline. Section~\ref{sec:SUMO} introduces \ac{sumo}, the open-source simulation platform employed throughout this thesis. Section~\ref{sec:glosa} describes the underlying principles and system architecture of \ac{glosa}.
\mynewline
The subsequent discussion is organised as follows. First, we present \ac{dp}–based approaches aimed at maximising traffic throughput (Section~\ref{sec:flow_glosa}). These methods compute optimal speed profiles that minimise stop‐and‐go behaviour and thus increase network capacity. Finally, we extend the \ac{dp} framework to incorporate energy considerations, formulating an \ac{eco-glosa} algorithm (Section~\ref{sec:eco_glosa}) that prioritises minimising fuel consumption through trajectory-level speed optimisation while still enhancing traffic throughput.
\mynewline
Section~\ref{sec:research_gap} then synthesises the remaining research gaps that motivate the contributions of this work.  

\section{Dynamic Programming for Throughput-Oriented \ac{glosa}}
\label{sec:flow_glosa}

\ac{dp} is a principled method for solving sequential decision problems. In the context of \ac{glosa}, it can be used to compute speed advisories that maximise intersection throughput and minimise vehicle delays. Throughput-oriented \ac{glosa} aims to reduce cumulative delay, the number of stops, and total travel time, subject to traffic signal constraints. Fuel consumption is treated as a secondary objective in this mode; it will become the primary focus in Section~\ref{sec:eco_glosa}. A clear delineation between the two modes clarifies their respective trade-offs.
\mynewline
The core idea is to model each vehicle’s approach as a discrete-time, discrete-space control problem. State variables capture the vehicle’s position and speed at each time step. Decision variables represent a finite set of acceleration actions. A cost function assigns an incremental delay penalty per time step. A terminal constraint enforces that the vehicle crosses the stop line during a green phase. Solving this problem via \ac{dp} yields a locally or globally optimal speed profile under the given objectives and constraints.

\subsection{Objectives and Problem Formulation}
\label{subsec:flow_dp_formulation}

The primary objective of throughput-oriented \ac{glosa} is to minimise the cumulative delay experienced by a vehicle when approaching a signalised intersection. Cumulative delay is defined as the difference between actual travel time and the ideal travel time at free-flow speed. Secondary objectives include minimising the number of full stops and the total travel time. Although fuel consumption reduction is desirable, it remains a lower-priority goal in this mode; Section~\ref{sec:eco_glosa} will invert this priority.
\mynewline
To capture these objectives, we formulate a constrained optimal control problem on a discrete grid. Let time be divided into $K$ intervals of length $\Delta t$. The state at step $k$ is 
\[
s_k = \bigl(x_k,\;v_k\bigr),
\]
where $x_k$ is the distance from the stop line and $v_k$ is the vehicle speed. The action at step $k$ is a discrete acceleration $a_k\in\mathcal{A}$, where
\[
\mathcal{A}=\{a_1,\,a_2,\,\dots,\,a_{N_a}\},
\]
and $a_{\min}\le a_i\le a_{\max}$. The state evolves according to the kinematic equations:
\[
x_{k+1} = x_k + v_k\,\Delta t + \tfrac12\,a_k\,\Delta t^2,
\]
\[
v_{k+1} = \max\bigl(0,\;v_k + a_k\,\Delta t\bigr).
\]

The instantaneous cost is defined as
\[
c(s_k,a_k) 
= \Delta t \;-\; \frac{x_{k+1}-x_k}{v_{\mathrm{free}}}
\;+\;\lambda_s\,\mathbf{1}_{\{v_{k+1}=0\}},
\]
where $v_{\mathrm{free}}$ is the free-flow speed, $\lambda_s$ is a stop penalty weight, and the indicator term penalises a full stop. This cost measures delay relative to free-flow, plus a penalty for stopping.
\mynewline
Signal timing constraints are enforced via a large terminal penalty. Denote the green phase interval for the approaching movement as $[t_{g,\mathrm{start}},\,t_{g,\mathrm{end}}]$. Let $t_k=k\,\Delta t$. If the vehicle has not crossed the stop line by the end of the horizon or arrives outside the green window, a penalty $M$ is added:
\[
c_T(s_K) = 
\begin{cases}
0, & \text{if } x_K \ge L_s \text{ and } t_K\in[t_{g,\mathrm{start}},t_{g,\mathrm{end}}],\\
M, & \text{otherwise},
\end{cases}
\]
where $L_s$ is the distance to the stop line. The choice of $M\gg1$ ensures that any feasible green-phase arrival is preferred.

The full optimisation problem is therefore:
\[
\min_{\{a_k\}_{k=0}^{K-1}}
\quad
J = \sum_{k=0}^{K-1} c(s_k,a_k)\;+\;c_T(s_K),
\]
subject to the state dynamics and action bounds. The solution is obtained by standard backward induction in discrete \ac{dp}.
\mynewline
Key modelling choices include:
\begin{itemize}
  \item \textbf{State discretisation:} Positions are quantised at resolution $\delta x$, speeds at resolution $\delta v$. Finer resolution improves accuracy but increases computational load.
  \item \textbf{Action set size:} The number of accelerations $N_a$ trades off solution fidelity and solver speed.
  \item \textbf{Horizon selection:} The time horizon $T=K\,\Delta t$ must exceed the latest feasible green-phase end.
  \item \textbf{Cost weights:} The stop penalty $\lambda_s$ and green-phase violation penalty $M$ must be tuned to reflect system priorities.
\end{itemize}

In practice, the \ac{dp} grid and cost parameters are configured offline. At runtime, real-time signal status and vehicle state feed into the \ac{dp} solver. A value iteration or policy-iteration algorithm computes the optimal action sequence. The first action yields the immediate speed advisory. Subsequent steps re-optimise as new signal updates arrive.
\mynewline
Throughput-oriented \ac{dp} methods can significantly reduce vehicle delay and travel time. Guo et al.~\cite{Guo2019} report travel time reductions of 23.6–35.7\% compared to adaptive signal control. Cai et al.~\cite{Cai2008} show a 12\% delay reduction relative to baseline TRANSYT model and a 23\% travel time reduction compared to uncoordinated operations. Computational complexity grows with finer discretization and longer planning horizons, so careful selection of state and action resolutions is essential. Nevertheless, with sensible parameter settings, computation times remain well below the planning horizon, enabling real-time implementation. The following subsection~\ref{subsec:flow_dp_algorithms_limitations} surveys algorithmic approaches and their limitations.


\subsection{Algorithmic Approaches and Limitations}
\label{subsec:flow_dp_algorithms_limitations}

Throughput‐oriented \ac{dp} for \ac{glosa} has evolved considerably since the early 2000s. We classify seven main research trajectories --- ranging from exact, full‐horizon optimisers to lightweight, driver‐centric heuristics --- by their approach to the trade‐off between advisory optimality and real‐time feasibility. The methods are assessed in the subsequent section on the basis of the quantitative performance enhancements, the computational or practical expenses, and the primary constraints that have been noted.
\mynewline
We begin with the computational baseline, full-horizon grid-based \ac{sdp}, whose optimality is provable yet costly. \emph{grid-based backward \ac{dp}}, discretising both space (vehicle position relative to the stop line) and speed into a finite two-dimensional lattice, and performing backward value iteration on the Bellman equation under a terminal “green-phase” constraint. Typaldos and Papageorgiou \cite{Typaldos2021} formulate the stochastic \ac{glosa} problem as a discrete-time \ac{sdp} and report that, due to the exponential growth of the discretised state–control domain, the one-shot \ac{sdp} algorithm may require several minutes of CPU time on a standard PC, rendering on-board real-time execution infeasible. Coarsening the discretisation (i.e.\ increasing the grid step) reduces computation time but shrinks the feasible state domain, potentially excluding viable green-phase arrival states or degrading advisory quality.
\mynewline
To cut this cost without sacrificing optimality, Typaldos et al.~\cite{Typaldos2023} shrink the state space via a moving corridor. \textit{Receding‐horizon \ac{dp}} --- often realised in an \ac{mpc} framework --- has recently been accelerated through Discrete Differential Dynamic Programming (\ac{dddp}). The algorithm restricts every \ac{dp} iteration to a slender state–space corridor centred on the previously accepted trajectory and shifts this corridor forward as fresh signal-state information becomes available, thereby retaining optimality while severely curtailing the search volume. In the benchmark stochastic \ac{glosa} scenario, \ac{dddp} replicated the full-horizon \ac{sdp} optimum of $J^\star=1.17517$ yet slashed total computation time from $614\ \mathrm{s}$ to $0.69\ \mathrm{s}$ --- nearly three orders of magnitude faster. Parametric studies covering a broad range of corridor widths and discretisation steps reported per-iteration CPU times spanning $0.28\ \mathrm{s}$ to $2.30\ \mathrm{s}$ while consistently achieving the same optimal cost, making advisory refresh rates of \(\sim1\,\text{Hz}\) feasible on standard on-board hardware. The chief drawback of corridor truncation is its sensitivity to window size: if the green phase commences outside the predicted corridor, admissible arrival trajectories may be excluded, whereas overly wide corridors inflate runtime. Typaldos et al.\ therefore advocate adaptive corridor re-expansion and discretisation refinement to trade off convergence reliability against computational load. 
\mynewline
Instead of pruning the state space, Cai et al. \cite{Cai2008} approximate the value function itself. \textit{Adaptive traffic-signal control via Approximate Dynamic Programming (ADP)} embeds Bellman’s principle in a forward, rolling-horizon scheme whose value function is approximated by a linear feature map updated online with either temporal-difference reinforcement learning (TD) or numerical perturbation learning (PL). Every $0.5\;\text{s}$ (fine resolution) or $5\;\text{s}$ (coarse resolution) the controller receives 10\,s of detector data, predicts a further 10\,s, and evaluates three options --- hold, immediate stage switch, or deferred switch within the head window --- implementing only the first increment while propagating parameter updates. In a three-stage isolated intersection, the fine-resolution ADP matched the full-horizon \ac{dp} optimum of $4.27\;\text{veh·s s}^{-1}$ within $0.35\;\text{veh·s s}^{-1}$ (ADP\textsubscript{TD} $4.62$, ADP\textsubscript{PL} $4.66$) yet cut per-hour run-time from $720\;$min (DP) to $5.5\;$min and $12\;$min, respectively. Relative to optimised TRANSYT fixed-time plans ($13.95\;\text{veh·s s}^{-1}$) the ADP reduced delay by $\approx67\%$, and working at $0.5\;\text{s}$ rather than $5\;\text{s}$ yielded a further $41\%$ improvement (coarse ADP\textsubscript{PL} $8.64\;\text{veh·s s}^{-1}$). Under a four-hour, time-varying demand pprofile,both learning rules converged to statistically indistinguishable means (ADP\textsubscript{PL} $3.24$ vs. ADP\textsubscript{TD} $3.28$; $p=0.43$), with TD requiring less than half the CPU time because it computes a single Bellman update per step. Sensitivity tests showed the best performance at a high discount rate $h=0.12$, indicating that near-term costs dominate and justifying the simple linear approximation. Chief limitations are reliance on large discounting (dampening the value-function’s contribution), linear state representation, and isolation to a single junction; the authors recommend extending to network-level coordination, parallelising the per-state cost evaluation, deriving constant-time queue-update formulas, and exploring non-linear or piecewise-linear value functions for stochastic arrivals and adaptive phase ordering.
\mynewline
A complementary path compresses complexity analytically: Samra et al. \cite{Samra2015} reformulate the \ac{dp} state to obtain linear time. They propose a linear-time–and-space dynamic-programming algorithm for optimal traffic-signal duration at an isolated intersection that re-encodes the \ac{dp} state around the current time step and prunes non-promising predecessors early, collapsing the complexity of the classic \ac{cop} formulation from $O(T^{3})$ time and $O(T^{2})$ memory to $O(|P|T)$ for both metrics; a C implementation on a 2.10 GHz Intel Core2-Duo achieved $>2700\times$ speed-up over \ac{cop} for a 15-min horizon with $T\!=\!1024$ one-second steps, finishing in $0.27$ s while matching the global optimum.  The method generalises to heterogeneous minimum-green and clearance constraints by prefix-encoding constrained phases, keeping the state space linear in $T$.  In Green Light District simulations, the algorithm cut average junction waiting time by $\approx63\%$ (2.7 ×) under heavy demand (spawn rate 0.4) for a three-phase intersection and by $\approx66\%$ (3 ×) for a four-phase node relative to the strongest reinforcement-learning baseline, with consistently low run-to-run variance.  Limitations include reliance on accurate short-term arrival forecasts—prediction errors break optimality guarantees --- linear scaling only in horizon length but not yet across networks, and an $O(|P|)$ per-state cost evaluation that becomes salient for large phase sets. The authors therefore recommend (i) embedding the algorithm in coordinated frameworks such as RHODES, (ii) parallelising across cores to handle multi-intersection deployments, (iii) deriving constant-time queue updates to reduce per-state cost to $O(1)$, and (iv) extending the formulation to stochastic arrivals and adaptive phase ordering while preserving worst-case linear bounds.
\mynewline
When analytical compression is impossible, meta-heuristics such as genetic algorithms trade exactness for scalability. Researchers turned to \emph{metaheuristic graph‐search} methods to alleviate the combinatorial explosion of multi‐segment \ac{glosa}. Seredynski and Khadraoui \cite{Seredynski2013} model the advisory problem as a sequence of $n$ road segments and encode each candidate solution as a vector of segment‐wise speed advisories. They solve this via a generational \ac{ga} employing selection, single‐point crossover, and mutation operators, iterating until a stop condition is met. Simulation results in non‐congested networks with up to fifteen segments show that the multi‐segment \ac{ga} approach significantly outperforms single‐segment \ac{glosa} in both travel‐time and fuel‐efficiency metrics. Although \ac{ga} circumvents the exhaustive enumeration required by \ac{dp}, its runtime still grows exponentially with the number of segments, and scaling beyond moderate route lengths remains infeasible on standard hardware without additional heuristic pruning or parallelization. The authors suggest that incorporating admissible heuristic functions --- such as remaining distance divided by free‐flow speed --- could further prune the search space while guaranteeing constraint satisfaction. 
\mynewline
Recently, researchers have embraced integrated control, co-optimising signals and \ac{cav} trajectories. \textit{Joint optimisation of vehicle trajectories and intersection controllers} (DP–SH) by Guo et al.~\cite{Guo2019} couples a dynamic-programming signal optimiser with the Shooting Heuristic (SH) trajectory generator to co-design signal phases, green durations and \ac{cav} speed profiles in a single forward recursion; SH restricts each car’s search to at most seven analytically solvable quadratic arcs, so \ac{dp} treats phases as stages and embeds SH as a sub-routine, retrieving the optimal control whenever the horizon-wide value function improves by less than 5 \%.  With a 122 s planning horizon and 8 s step size, the algorithm computes a full four-phase plan in 9.8–15.2 s for up to twenty vehicles per approach on a 1.8 GHz laptop --- well below 10\% of real time --- and larger step sizes cut runtime to 2.8 s at only marginal performance loss.  Across twelve demand scenarios (segment lengths 400–1200 m, saturation ratios 0.6–1.5) DP–SH with subsequent SH parameter refinement trimmed mean travel time by 23.6–35.7\% and fuel by 11.8–31.5\% relative to adaptive signal control, while internal parameter tuning of acceleration bounds yielded a further 10–18\% fuel saving without sacrificing throughput.  In mixed traffic, the framework remains robust: even at 10\% \ac{cav} penetration, average delay and consumption still drop 3.1\% and 6.2\%, rising monotonically to 34.8\% and 31.5\% as penetration approaches 100\%. Sensitivity tests show computational load grows linearly with horizon length but only weakly with vehicles per phase. However, horizons below \(\approx\) 400 m may preclude feasible SH trajectories under heavy queues, requiring either minimum-length constraints or upstream spill back prediction. Principal challenges are (i) non-convex parameter tuning that can stall in local minima, (ii) reliance on accurate entry-boundary forecasts, (iii) focus on isolated intersections—spill back chains and multi-signal coordination remain open—and (iv) real-time scalability when horizon or demand scales; the authors therefore advocate parallelisation, distributed feedback control, extension to corridor-level or signal-free \ac{cav} reservation policies, and incorporation of stochastic capacity for truly network-wide deployment.
\mynewline
At the opposite end of the spectrum, DC-GLOSA achieves field-ready simplicity by relegating computation to a smartphone. \textit{Driver-centric Green Light Optimal Speed Advisory (DC-GLOSA)} proposed by Suramardhana and Jeong~\cite{Suramardhana2014} equips each vehicle with a smartphone-based on-board unit that (i) receives \ac{spat} messages from the roadside unit, (ii) estimates ego-motion via GPS, and (iii) issues one of seven instantaneous acceleration/braking advisories chosen by comparing the current time-to-cross (TTC) against six analytically defined reference mobility models—maximum acceleration, constant acceleration, constant speed, constant braking, maximum braking, and emergency deceleration to stop—thereby respecting speed bounds $v_{\min}=20\,$km h$^{-1}$ and $v_{\max}=40\,$km h$^{-1}$ while limiting longitudinal jerk to $a\in[\,{-}3.82,\,3.82\,]\,$m s$^{-2}$. The algorithm incrementally updates the advisory whenever a new \ac{spat} frame or significant state change arrives, deferring freshly triggered voice prompts until the previous one has finished, avoiding cognitive overload and queuing at most the most recent message. Field trials with a single vehicle at a three-arm signalised junction on the Pusan National University campus ($T_{\text{green}}=30$ s, $T_{\text{amber}}=3$ s, $T_{\text{red}}=100$ s) showed that DC-GLOSA reduced average intersection-stopping time by $23.9\%$ over a baseline with no advisory across 54 runs spanning approach distances $D=50\text{–}200$ m and residual-green intervals $T_{\text{ref}}=10\text{–}70$ s. While the strategy improves throughput and driver comfort, two practical issues persist: (a) advisory latency from smartphone audio playback ($\approx 2$–$3$\,s) can render a message stale in rapidly changing scenarios, and (b) the scheme relies on accurate \ac{spat} and driver compliance; failure to receive timely updates or aggressive driving can force emergency braking when $D_{\text{stop}}>D$. The authors therefore recommend future work on adaptive message suppression, multi-vehicle coordination to resolve conflicting advisories, and integration of power-train awareness plus eco-cost functions to extend the current TTC and emission-minimisation.
\mynewline
Viewed along the complexity axis we have now traversed, three patterns emerge. Exact methods (grid-based SDP and DDDP) guarantee optimality but strain real-time hardware; approximate or analytical reductions (ADP and linear-time DP) temper this cost while preserving most benefits; heuristic and driver-centric schemes (GA, DP–SH, DC-GLOSA) prioritise deployment ease and system-wide coordination, albeit at the risk of sub-optimality.
\mynewline
\textbf{Key quantitative insights across methods:}
\begin{itemize}
  \item \emph{Delay or stopping‐time reduction:} 40\,\% in grid-based SDP, 39\,\% in DDDP corridor studies, 23.9\,\% in the DC-GLOSA field trial, and 30–36\,\% in DP–SH mixed-traffic scenarios.
  \item \emph{Computation time per advisory:} 5–10\,min for full-horizon SDP, 0.28–2.30\,s for DDDP, 5.5–12\,min per simulated hour with ADP, 0.27\,s for linear-time DP, 2.8–15.2\,s for DP–SH, and sub-second for DC-GLOSA smartphone prompts.
  \item \emph{Scalability constraints:} exponential state-space growth with SDP, corridor-width sensitivity in DDDP, feature-map bias for ADP, per-state $O(|P|)$ cost for each phase in linear-time DP, combinatorial explosion as segment count increases in GA, demand-forecast reliance in DP–SH, and audio-latency plus compliance issues in DC-GLOSA.
\end{itemize}

\textbf{Principal limitations and open challenges.} First, \emph{computational tractability} remains the dominant hurdle: exact full-horizon \ac{sdp} and its corridor-pruned variant (DDDP) guarantee optimality but incur exponential or corridor-width-sensitive costs that exceed typical on-board budgets. Second, \emph{approximation bias} affects receding-horizon and feature-based ADP schemes --- incorrect corridor sizes may exclude feasible trajectories, while linear value‐function approximations and high discounting can distort long-term performance. Third, \emph{per-state cost scaling} persists: Samra et al.’s linear-time DP still requires $O(|P|)$ evaluations under large phase sets, and genetic algorithms face combinatorial explosion as segment count grows. Fourth, \emph{forecast and communication dependences} challenge DP–SH and DC-GLOSA: the former demands accurate boundary predictions and low-latency V2X links, the latter suffers 2–3s audio latency and driver compliance issues. Fifth, \emph{constraint enforcement} in approximate or continuous DP often relies on heuristic corrections for hard green-phase satisfaction, undermining formal guarantees. Sixth, \emph{secondary objectives} --- ride comfort, fuel economy and emissions --- remain largely orthogonal, multiplying the action space and complicating convergence. Finally, \emph{network scalability} is unresolved: nearly all methods target isolated intersections or single vehicles, and extending them to stochastic, multi-junction corridors with spill back and dynamic coordination stands as the grand challenge.
\mynewline
In conclusion, throughput-oriented \ac{dp} for GLOSA has delivered prototype-scale improvements --- up to 40\,\% delay reduction in grid-based \ac{sdp}, 39\,\% with receding-horizon DDDP, and 30–36\,\% in integrated co-optimisation --- while driver-centric and continuous methods produce sub-second advisories with moderate gains. Real-time deployment will require algorithms that reconcile computational efficiency, rigorous constraint satisfaction, and true multi-objective optimality. Future work must unite uncertainty-aware control, low-latency V2X communication, and network-level coordination to translate current prototypes into robust, large-scale field systems.


\section{\acl{dp} for Energy-Aware \acs{glosa}}
\label{sec:eco_glosa}

\ac{glosa} algorithms advise road users on speed adjustments that maximise the probability of crossing signalised intersections during green phases. When the objective is to minimise energy use rather than merely reduce delay, \ac{dp} offers a mathematically rigorous means of embedding high-fidelity powertrain models, signal timing constraints, and safety limits into a single optimal-control framework. The appeal of \ac{dp} lies in its ability to guarantee local optimality over a discretised state-action space, to accommodate non-convex efficiency maps and integer variables, such as gear or hybrid mode selection, and to incorporate hard constraints like arrival windows or legally mandated speed bounds.
\mynewline
An energy-conscious \ac{glosa} \ac{dp} must address three fundamental modelling questions. First, for the \textit{objective formulation}, does the cost functional target sole fuel minimisation, or does it balance energy with other factors like delay, comfort, or even pollutant exposure? Second, concerning the \textit{powertrain representation}, should the stage cost rely on tractive-energy proxies, such as mass-speed-acceleration products, or on calibrated maps like PHEMlight5, HBEFA4, or VT-CPFM, which convert wheel demands into fuel mass and \ac{co2} in grams? Third, for the \textit{constraint encoding}, how are vehicle dynamics (speed, longitudinal acceleration, jerk), legal limits, and powertrain ceilings (engine torque, battery state-of-charge, electric motor power) enforced without causing the dimension of the state space to explode? The following subsection synthesises how recent studies have tackled these three questions, highlighting common design patterns, trade-offs, and emerging trends toward multi-criteria optimisation.

\subsection{Energy and Multi-Criteria Objective Formulations}
\label{subsec:eco_dp_objectives}

An \ac{eco-glosa} controller seeks to find a longitudinal control signal, $u(t)$, which is typically a bounded acceleration or wheel torque. The resulting vehicle trajectory should minimise an energy-centric performance index while guaranteeing legal and safety compliance at signalised intersections. The system state can be represented by the vector $\mathbf x(t) = [s(t), v(t), \mathrm{SoC}(t)]^{\!\top}$, which comprises the travelled distance $s$, speed $v$, and, for electrified powertrains, the battery's state of charge. The continuous-time dynamics can be abstracted as:
\begin{equation*}
    \dot{\mathbf x}(t) = \mathbf f\bigl(\mathbf x(t), u(t)\bigr),
    \qquad
    u_{\min} \le u(t) \le u_{\max},
\end{equation*}
where $\mathbf f$ captures the longitudinal equations of motion, rolling and aerodynamic resistances, grade, and a simple battery equivalence for hybrids. Discretising this formulation on a uniform temporal grid, $t_k = k \Delta t$, yields the canonical \ac{dp} recursion:
\begin{equation*}
    V_k(\mathbf x_k) = \min_{u_k \in \mathcal U} \left\{ g_k(\mathbf x_k, u_k) + V_{k+1}\bigl(\mathbf x_{k+1} \bigr) \right\},
    \qquad
    \mathbf x_{k+1} = F_k(\mathbf x_k, u_k).
\end{equation*}
This recursion is subject to a terminal cost, $V_{N}(\mathbf x_N)=\Phi(\mathbf x_N)$, which encodes the mandatory arrival time $t_f$, the stop-bar position $s_f$, and the permissible speed range $v\in[v_{\min},v_{\max}]$ at the intersection.

\paragraph{Single-Criterion Energy Minimisation.}
The most direct objective is the integral of instantaneous fuel or electrical power $P_{\text{tr}}$:
\begin{equation*}
\label{eq:single}
    J_{\text{fuel}} = \int_{0}^{t_f} P_{\text{tr}}\bigl(\mathbf x(t), u(t)\bigr)\,\mathrm{d}t
    \quad\Longrightarrow\quad
    g_k = \Delta t \cdot P_{\text{tr}}\bigl(\mathbf x_k, u_k\bigr).
\end{equation*}
For conventional engines, $P_{\text{tr}}$ is mapped to mass flow $\dot m_{\text{fuel}}$ via calibrated look-up tables such as PHEMlight5 or HBEFA4. For battery-electric vehicles it equals traction power minus regenerative recovery, and for hybrids it combines engine fuel and battery energy through an equivalence factor $\xi$ so that $g_k = \Delta t \cdot \left[ \dot m_{\text{fuel}} + \xi P_{\text{bat}} \right]$.

\paragraph{Weighted-Sum Multi-Objective Costs.}
Minimising Equation~\eqref{eq:single} alone can drive the optimum toward unrealistically low speeds. A pragmatic remedy is to augment the single-criterion formula with travel time $T = t_f - t_0$ and comfort penalties on jerk $j = \dot a$:
\begin{equation*}
    J_{\mathrm{sum}} = \alpha E + \beta T + \gamma \int_{0}^{t_f} j(t)^2\,\mathrm{d}t,
    \qquad
    E = \int_{0}^{t_f} P_{\text{tr}}\,\mathrm{d}t,
\label{eq:weighted}
\end{equation*}
where $\alpha$, $\beta$, and $\gamma$ tune the policy along the energy–delay–comfort trade-off surface. In \ac{dp} form, the stage cost becomes
$g_k = \alpha \Delta t\,P_{\text{tr}} + \beta \Delta t + \gamma \Delta t\,j_k^2$.

\paragraph{Pareto Formulations.}
To avoid arbitrary weights, an \emph{$\epsilon$-constraint} variant minimises one metric while bounding the others:
\begin{equation*}
    \min\;E \;\;\text{s.\,t.}\;\;T \le \bar T, \qquad
    g_k = \Delta t\,P_{\text{tr}}, \qquad
    \Phi = \mathbbm{1}_{T \le \bar T},
\end{equation*}
generating Pareto-optimal fronts by sweeping the tolerance $\bar T$. A higher-level supervisory module can then select the point that best fits network objectives or driver preference.

\paragraph{Constraint Set.}
All formulations inherit hard bounds from road regulations and power-train physics:
\begin{align*}
    0 &\le v_k\le v_{\mathrm{lim}}(s_k), &
    |a_k| &\le a_{\max}, &
    |j_k| &\le j_{\max}, \nonumber\\
    P_{\text{eng},k} &\le P_{\text{eng}}^{\max}, &
    \mathrm{SoC}_{\min} &\le \mathrm{SoC}_k\le \mathrm{SoC}_{\max}, &
    t_{\mathrm{g,\,on}}\le t_k &\le t_{\mathrm{g,\,off}},
\end{align*}
where the final line enforces the \ac{spat}-derived arrival window. Constraints are handled either by pruning infeasible lattice nodes before the Bellman sweep or by adding large penalties to $g_k$.

\paragraph{Lattice Design and Complexity Control.}
Let $n_s$, $n_v$ and $n_{\mathrm{SoC}}$ be the discretisation levels of distance, speed, and battery charge. The complexity of the Bellman formula scales as $\mathcal O(N\,n_s n_v n_{\mathrm{SoC}} N_u)$, with $N_u$ the number of admissible controls. To render on-board computation feasible, three reduction strategies dominate:  
(1)~\textit{Variable Grids} contract the lattice where the value function is nearly affine,  
(2)~\textit{Multi-Stage} splitting solves the Bellman formula separately across signal phases or queue discharge events and stitches the partial value functions at boundaries, and  
(3)~\textit{Surrogate Models} replace costly $P_{\text{tr}}$ evaluations with neural or polynomial emulators trained offline.
\mynewline
The above framework, which includes a rigorous Bellman recursion, a cost functional that can be single or multi-objective, and systematic constraint handling, forms the basis for energy-aware \ac{glosa} speed guidance. Subsequent subsections build on this scaffold to incorporate stochastic queues, \ac{v2x} connectivity and cooperative manoeuvres.

\subsection{Queue–Connectivity–Aware \ac{eco-glosa}}
\label{subsec:eco_dp_queue}

This section reviews eight \ac{dp}-based eco-driving strategies that are designed to minimise fuel or electrical energy consumption while respecting signal timing and queue dynamics. The discussion begins with single-vehicle, offline \ac{dp} formulations for battery and hybrid powertrains, as explored by Park et al. \cite{Park2024} and Pulvirenti et al. \cite{Pulvirenti2023}. It then proceeds to cover real-time, multi-stage, and queue-aware controllers from researchers such as Kamalanathsharma et al. \cite{Kamalanathsharma2013} and He et al. \cite{He2015}. The section concludes with an examination of cooperative, \ac{v2i}/\ac{v2v}-enabled schemes developed by Yang \cite{Yang2017}, Ala \cite{Ala2016}, Yang \cite{Yang2021}, and Dong \cite{Dong2024}. For each method, the \ac{dp} formulation is summarised, precise savings and runtimes are reported, and key limitations and suggested remedies are highlighted.
\mynewline
To begin, Park et al. \cite{Park2024} tackle the offline \ac{bev} case by formulating an eco-driving problem for a Hyundai Ioniq 5. They use a two-state (distance, speed) discrete-time \ac{dp} model that minimises cumulative electrical power. This is achieved by exhaustively enumerating acceleration commands on a $1\unit{\second}$ grid while enforcing fixed departure and arrival times, distance and speed limits, and drivetrain comfort bounds. The solver explores a $150\unit{\second}$ horizon with a $0.05\unit{\metre}$ space step and a $0.1\unit{\metre\per\second}$ speed step, which incurs $337\unit{\second}$ of offline computation. Comparing the \ac{dp} trajectories with an \ac{idm} baseline shows sizeable battery savings. Over a $500\unit{\metre}$ sprint, a \enquote{Comfort-DP} profile with human-like acceleration cuts energy use by $34.3\%$, while an \enquote{Aggressive-DP} profile using full torque reduces it by $67.5\%$. Over a $900\unit{\metre}$ run, the respective reductions are $25.5\%$ and $47.8\%$. The analysis attributes these gains to two factors: the strict use of high-efficiency tangential and stationary points on the motor's Pareto frontier during acceleration and cruising, and full energy recuperation during terminal braking. The limitations of this approach include its reliance on perfect prior knowledge of boundary conditions and zero road grade, the absence of battery power limits at low \ac{soc}, and infeasible on-board runtimes without a GPU or surrogate policies. The authors, therefore, advocate for state-reduction heuristics, stochastic \ac{spat} and traffic integration, and real-vehicle validation to bridge the gap between offline optimality and real-time deployment.
\mynewline
While Park et al. focus on an exhaustive grid search for \acp{bev} under fixed boundary conditions, Pulvirenti et al. \cite{Pulvirenti2023} extend the \ac{dp} paradigm to plug-in hybrids by introducing a \ac{vgdp} framework. This framework exploits \ac{v2x} data and cloud computation to generate energy-optimal speed trajectories for a plug-in hybrid Mercedes E300de along a $96\unit{\kilo\metre}$ RDE-compliant route. The spatial-domain \ac{dp} state is defined as either $(v)$ for a stop-controlled scenario or $(v,t)$ for a traffic-light scenario, with longitudinal acceleration as the control variable. The bi-objective stage cost combines incremental traction energy with travel time via a weighting factor $\beta$. Instead of searching a fixed lattice, \ac{vgdp} shrinks the speed grid to an interval defined by a $500\unit{\metre}$ moving-average envelope ($\pm20\unit{\kilo\metre\per\hour}$) and bounds the time grid around an ETA corridor derived from communicated \ac{spat} data, speed limits, and grades. With a $5\unit{\metre}$ spatial step, this method reduces the node count by two orders of magnitude and slashes the solver wall-time on an Intel E5-2680 workstation from $24\unit{\minute}$ to under $1\unit{\minute}$ (a $95\%$ reduction), without degrading optimality. In \textit{Scenario 1}, with stop signs imposed at every observed standstill, the optimiser delivers smoother acceleration profiles that lower the fuel-equivalent energy by $29\%$ and shrink the cruise time by $10\%$, while still respecting all full-stop constraints. In \textit{Scenario 2}, all stops are replaced by traffic lights with a $60\unit{\second}$ cycle and $60\%$ green time, where the \ac{spat} data is assumed to be deterministic and known. Here, \ac{vgdp} schedules an almost uniform $49\unit{\kilo\metre\per\hour}$ cruise, eliminates complete halts, and achieves $54\%$ energy and $38\%$ travel-time savings relative to the reference drive. A sensitivity analysis shows only modest Pareto trade-offs as $\beta$ varies from $0.2$ to $0.8$. The limitations include the reliance on perfect \ac{spat} and queue-free flow, a single-vehicle scope, and the need for a high-resolution distance discretisation ($5\unit{m}$), which still burdens memory when scaled to network traffic. Future work is therefore directed at platoon-level coordination, stochastic queue integration, and on-road hardware validation of the cloud-assisted control loop.
\mynewline
Moving from cloud-assisted, offline trajectory planning to real-time control, Kamalanathsharma et al. \cite{Kamalanathsharma2013} propose a \textit{\ac{msdp} eco-speed controller}. This controller frames the approach to a red phase as a least-cost path-finding problem on a two-stage lattice. The upstream stage chooses a constant deceleration, $d$, and a cruise time, $t_{\mathrm c}$, that brings the vehicle to the stop-bar exactly at the green onset. The downstream stage then selects a throttle trajectory to re-accelerate the vehicle to its desired speed. Both stages minimise the cumulative fuel consumption, as predicted by the VT-CPFM model, subject to comfort bounds ($d\le3\unit{\metre\per \second\squared}$, $a^{+}\le1.1\unit{\metre\per\second\squared}$), grade, weather, and microscopic resistance forces. A heuristic $\mathcal{A}$-star recursion evaluates each $0.1\unit{\second}$ step, pruning dominated nodes so that real-time feasibility is maintained despite the use of full powertrain and aerodynamics equations.
Agent-based MATLAB simulations over a $200\unit{\metre}$ upstream and $400\unit{\metre}$ downstream zone, with random arrivals across an $84\unit{\second}$ cycle and eighteen combinations of speed limits $(25/35/45\unit{\mile\per\hour} \approx 40/56/72\unit{\kilo\metre\per\hour})$, pavement conditions (dry, wet, snow), and grades ($\pm5\%$), report mean savings of $19.5\%$ in fuel and $32\%$ in travel time relative to an ITE-calibrated human baseline. The best-case results reached $37.2\%$ and $41.5\%$ under snowy, $25\unit{\mile\per\hour} \approx 40\unit{\kilo\metre\per\hour}$) conditions. Car-following tests that couple the Rakha–Pasumarthy–Adjerid model show that a non-instrumented follower inherits an approximate $15\%$ fuel reduction when tracking an MS-DP lead vehicle, which indicates spill-over benefits. The limitations of this approach are its reliance on perfect \ac{spat}, a single-vehicle optimisation that ignores queue spill back, sensitivity to lane-changing below a $30\%$ \ac{cv} penetration rate, and validation being confined to simulation. Oversaturated demand, at or above $800\unit{\veh\per\hour}$, breaks the kinematic-wave predictor and can render the benefits negative. The authors therefore call for \ac{v2v}-aided queue sensing, speed-harmonisation overlays, multi-intersection coordination, robustness to packet loss and driver non-compliance, and field trials to confirm the observed potential for approximately $20\%$ fuel savings.
\mynewline
In a similar real-time vein, He et al. \cite{He2015} formulate a \ac{qost} that embeds stochastic queue-tail constraints and non-blocking requirements into a multi-stage optimal-control problem, which enables sub-second computation. In this manner, a single car never impedes traffic moving upstream and does not cause the green light to go out for other vehicles. A five-variable quadratic surrogate, which includes terminal time, two accelerations, and two switch instants, solves each stage in approximately $1.5\unit{\second}$ on off-the-shelf hardware, making real-time deployment plausible. In a two-intersection benchmark, \ac{qost} eliminates a five-second idle time at the first queue. However, because it cruises at an eco-speed of $30\unit{\mile\per\hour} \approx 48\unit{\kilo\metre\per\hour}$, the run time grows from $77.3\unit{\second}$ without advice to $90.0\unit{\second}$, an increase of $16\%$. Simultaneously, stops fall to zero and fuel use drops by $41\%$. A six-signal field trial on Minneapolis TH-55 repeats this pattern. The advised vehicle traverses the corridor without stopping, yet its travel time rises from $236\unit{\second}$ to $258\unit{\second}$ ($+9\%$) as queues are absorbed upstream. This results in a $29\%$ fuel saving and a net fuel-economy improvement from $17\unit{\mile\per\gallon} \approx 13.8\unit{\liter\per100\kilo\metre}$ to $24\unit{\mile\per\gallon} \approx 9.8\unit{\liter\per100\kilo\metre}$.
By design, the model also compels the subject vehicle to clear the stop-line before a threshold, $T_{\tilde B}$, so as not to lengthen the residual queue or trap following traffic in the next red phase, thereby avoiding secondary delay propagation. The authors acknowledge that the traffic-flow benefits are therefore mixed. While idling, shockwaves, and spill back are curtailed, the mean travel time for the advised vehicle, and hence the point throughput, can worsen unless the eco-speed is increased or applied jointly to a platoon. Key challenges include (i) a reliance on high-fidelity, real-time queue estimation, where biases can jeopardise both feasibility and flow gains; (ii) piecewise-constant accelerations that introduce jerk peaks; and (iii) a single-vehicle scope, which cannot exploit cooperative gaps or optimise a multi-vehicle order. Proposed remedies include embedding stochastic queue predictors, smoothing trajectories with powertrain-aware penalties, extending the formulation to the simultaneous optimisation of several vehicles at the front of the queue, and implementing rolling re-optimisation when signals skip phases. These are all aimed at transforming the current fuel-centric design into one that also boosts corridor-level throughput under varying demand patterns.
\mynewline
Building on single-vehicle queue awareness, Yang et al. \cite{Yang2017} introduce a \textit{queue-aware Eco-Cooperative Adaptive Cruise Control} (Eco-CACC-Q) method. This is a cooperative, \ac{v2i}-enabled adaptive cruise control system that uses \ac{spat} packets and kinematic-wave queue estimates to optimize fuel consumption across a platoon. Using this information, advisory speed limits are calculated and updated every second to minimize the instantaneous VT-CPFM fuel rate and to ensure that the probe vehicle reaches the stop-bar precisely when the final vehicle in the queue is released. A receding-horizon quadratic program simultaneously selects an upstream constant deceleration, $a^{-}$, a cruise speed, $v_{\mathrm c}$, over the remaining approach, and a downstream acceleration, $a^{+}$, on a $500\unit{\metre}/200\unit{\metre}$ control segment. The search is bounded by comfort limits of $0\le a^{-}\le3\unit{\metre\per \second\squared}$ and $0\le a^{+}\le2.5\unit{\metre\per\second\squared}$, as well as by \ac{soc}-neutrality constraints.
Single-lane integration simulations with a $20\%$ \ac{mpr} and a uniform $500\unit{\veh\per\hour}$ inflow show that Eco-CACC-Q trims the average fuel use by $11.4\%$ relative to the base car-following model and by $4.5\%$ compared with the non-queue Eco-CACC variant. It also eliminates complete stops and cuts the speed variance from $29.7$ to $15.3\unit{\kilo\metre\per\hour}$. The savings scale almost linearly with penetration, reaching $18.0\%$ at $100\%$ \ac{mpr}. On two-lane approaches, the algorithm is beneficial only above a $30\%$ \ac{mpr}. Below that threshold, aggressive lane changes around slower controlled vehicles inflate acceleration spikes and raise network fuel use by approximately $5\%$. At full penetration, the multi-lane test bed records an $18.3\%$ overall saving, with a $19.2\%$ saving for the \ac{cacc} fleet itself. Key limitations include a dependence on accurate queue-length prediction, as errors from lane-changing or oversaturation negate the benefits. Other limitations include audio or control-actuation latency and the assumption of isolated intersections. At very high demand, the method fails once the LWR-predicted queue exceeds the control horizon. The authors, therefore, call for \ac{v2v}-aided queue sensing, corridor-level coordination with speed-harmonisation controllers, robustness analyses against packet loss and driver non-compliance, and real-vehicle validation to confirm the simulation gains.
\mynewline
Complementing the algorithmic development, Ala et al. \cite{Ala2016} perform an extensive INTEGRATION-based sensitivity study of Eco-CACC-Q, quantifying its performance over varying penetration rates, green splits, and demand levels. Although their contribution is a broad sensitivity study rather than a new controller, they evaluate the Eco-CACC-Q algorithm, which exploits \ac{v2i} \ac{spat} packets and kinematic-wave queue estimates to compute an advisory trajectory each second. This trajectory is composed of a bounded upstream deceleration, $a^{-}\in[0,3]\unit{\metre\per\second\squared}$, a cruise speed, $v_{\mathrm c}$, timed to reach the stop-bar just as the queue clears, and a downstream re-acceleration, $a^{+}\le 2.5\unit{\metre\per\second\squared}$.
Single-lane tests with a fixed $500\unit{\metre}$ upstream and $200\unit{\metre}$ downstream control horizon and a demand of $300\unit{\veh\per\hour}$ show fuel-consumption savings that scale almost linearly with the \ac{mpr}, peaking at $19\%$ when the \ac{mpr} is $100\%$. On two-lane approaches, the algorithm yields negative benefits below a $30\%$ \ac{mpr} because lane-changing around slower controlled vehicles injects oscillations. Above that threshold, the savings recover and reach $19\%$ at full penetration. Varying the green split from $0.3$ to $0.7$ of an $84\unit{\second}$ cycle changes the baseline fuel use but alters the Eco-CACC-Q savings by at most $2\%$. Increasing the upstream control length from $200\unit{\metre}$ to $700\unit{\metre}$ boosts single-lane gains from $12\%$ to $18\%$, with marginal returns beyond $500\unit{\metre}$. The same pattern holds for two-lane links once the \ac{mpr} is greater than $30\%$. The demand sensitivity analysis reveals an optimal flow of $500\unit{\veh\per\hour}$ for a single lane and $600\unit{\veh\per\hour}$ for two lanes. Oversaturated flow, at $800\unit{\veh\per\hour}$, breaks the kinematic-wave queue predictor, causing rolling queues and stop-and-go waves, which turns the benefits negative. A four-leg junction case with asymmetric volumes of $1000\unit{\veh\per\hour}$ on the major road confirms the need for high penetration. Below a $25\%$ \ac{mpr}, fuel use rises, but at full penetration, network consumption drops by up to $25\%$. The limitations highlighted include: (i) a dependence on accurate real-time queue length, as errors or rolling queues undermine the trajectory; (ii) negative effects at low \ac{mpr} on multilane arterials due to lane-changing; (iii) algorithm breakdown under oversaturation; (iv) the assumption of isolated intersections and deterministic \ac{spat}; and (v) the absence of field validation. Future work should therefore target \ac{v2v}-aided queue sensing, speed-harmonisation overlays to throttle entry flow, multi-intersection coordination, and robustness analyses against packet loss and driver non-compliance.
\mynewline
Extending the approach to multi-signal corridors, Yang et al. \cite{Yang2021} propose \ac{ecomsq}, which is a modular, queue-aware, multi-signal controller that broadcasts second-by-second advisory speed limits to \acp{cv}. Leveraging \ac{v2i} \ac{spat} packets and kinematic-wave queue estimates, the algorithm solves a three-segment optimal-control problem for each approaching vehicle. This involves: (i) decelerating at a bounded rate of $a^{-}\in[0,3]\unit{\metre\per\second\squared}$ to a cruise speed, $v_{\mathrm c,1}$, that ensures the stop-bar is reached exactly as the last queued vehicle departs; (ii) re-accelerating or further decelerating to a second cruise speed, $v_{\mathrm c,2}$, that aligns with the downstream signal; and (iii) accelerating at $a^{+}\le 2.5\unit{\metre\per\second\squared}$ back to the link speed limit after clearing the final queue, all while maintaining \ac{soc} neutrality for the $48\unit{\volt}$ mild-hybrid powertrain model.
Implementation in the INTEGRATION micro-simulator shows that on a two-signal arterial with a demand of $600\unit{\veh\per\hour}$, fuel use falls by $7.0\%$ at $100\%$ \ac{mpr}, compared to $4.2\%$ for single-signal control, and the per-vehicle speed variance drops by $30\%$. Extending the analysis to a four-signal corridor with $600\unit{\metre}$ spacing yields savings of $7.7\%$ for single-lane links and $4.8\%$ for two-lane links. A sixteen-intersection grid posts up to a $15\%$ system-wide reduction, with benefits persisting for all \acp{mpr} because side-street and left-turn movements are single-lane constrained. A sensitivity analysis identifies several key factors: (a) an optimum demand of $400\unit{\veh\per\hour}$, which saves $13.5\%$; (b) short green splits ($35\%$ of a $120\unit{\second}$ cycle), which raise savings to $13.8\%$; (c) suboptimal offsets (e.g., $100\unit{\second}$ versus the $45\unit{\second}$ green-wave optimum), which boost savings to $13.0\%$; and (d) a $700\unit{\metre}$ signal spacing, which gives a peak improvement of $13.1\%$. Conversely, offsets near the green-wave optimum cut gains to $2.8\%$. Below a $30\%$ \ac{mpr} on multi-lane arterials, lane-changing around slower \acp{cv} can negate benefits, turning savings negative until the cooperative density exceeds this threshold. Under over-saturated demand ($1000\unit{\veh\per\hour}$), rolling queues breach the LWR predictor, which slashes savings to just $2.7\%$ even at a $10\%$ \ac{mpr}. The limitations, therefore, include a dependence on accurate real-time queue length and dissipation forecasts, as errors propagate into suboptimal cruise speed choices. Other limitations are sensitivity to lane-changing when the \ac{mpr} is low, diminished efficacy near optimal offsets or long green phases, and a lack of spill back handling in over-saturation. The authors call for \ac{v2v}-aided queue sensing, stochastic queue models, corridor-level coordination, and adaptive speed-harmonisation overlays to maintain benefits under high load and mixed traffic.
\mynewline
Finally, Dong et al. \cite{Dong2024} integrate overtaking manoeuvres into the eco-approach control by coupling lane selection via \ac{dp} with a Pontryagin-based speed optimizer, thereby achieving combined lane-planning and energy savings. Using a two-stage receding-horizon framework, the \textit{overtaking-enabled eco-approach control} (OEAC) method yields closed-form control torques while respecting safety headways, comfort bounds ($a^{-}\in[0,3]\unit{\metre\per\second\squared}$, $a^{+}\le2.5\unit{\metre\per\second\squared}$), and deterministic \ac{spat} constraints. The first stage casts the lane-selection task as a finite-horizon Markov decision process, which is solved by \ac{dp} and explicitly models disturbances from surrounding vehicles.
Extensive Monte-Carlo simulations over $10,000$ randomised urban scenarios show that the OEAC method cuts the composite \enquote{monetised} driving cost by an average of $20.91\%$ versus a constant-speed (CS) car-following baseline and by $5.62\%$ versus a regular eco-approach (READ) controller, with maxima of $59.97\%$ and $59.02\%$, respectively. In a representative moderate-flow case, with a vehicle density of $120\unit{\veh\per\kilo\metre}$, the OEAC executes a single lane change, which enables passage in the first green window. This cuts the travel time by $55.93\%$, traction energy by $36.43\%$, and the total cost by $54.12\%$ relative to the CS baseline. Compared to the READ controller, the reductions were $55.89\%$, $31.46\%$, and $53.01\%$, respectively. The computational overhead is negligible, with a mean step time of $0.2\unit{\milli\second}$ and a worst-case time of $9.3\unit{\milli\second}$ on an i9-12900K PC, which is well below the $10\unit{\milli\second}$ simulation tick. The limitations include a dependence on accurate, real-time queue-length and \ac{spat} data, as errors or oversaturation can negate the benefits. Other limitations are the reversion to pure car-following when safe overtakes are unavailable, negative fuel savings below a $30\%$ \ac{mpr} on multilane links due to cut-ins, and the validation being restricted to simulation.
\mynewline
The surveyed \ac{dp}-based \ac{glosa} methods span a range from offline, single-vehicle formulations for \acp{bev} and hybrids to real-time controllers and cooperative \ac{v2x}-enabled schemes. Offline \ac{dp} approaches, such as those by Park et al. and Pulvirenti et al., achieve substantial energy savings, which can be up to $67.5\%$ in \acp{bev} and $54\%$ for hybrids, but they do so at the cost of high computation time or a reliance on cloud resources. Real-time \ac{msdp} and \ac{qost} frameworks, developed by researchers like Kamalanathsharma et al. and He et al., reduce the run-time to sub-second levels while still maintaining fuel or energy reductions of $20\%$--$40\%$ by embedding queue and non-blocking constraints. Cooperative eco-cruise controllers, including those by Yang (2017), Ala (2016), and Yang (2021), leverage \ac{spat} data and kinematic-wave models to achieve fleet-level savings of $7\%$--$19\%$, depending on \ac{mpr}, intersection spacing, and signal timing. The OEAC framework by Dong et al. further incorporates lane-change decisions to realise composite cost reductions of up to $21\%$.
Despite these advances, common limitations persist across the literature. These include assumptions of perfect \ac{spat} and queue knowledge, a scope that is often limited to single vehicles or platoons, sensitivity to low penetration rates and oversaturated demand, and a predominance of simulation-based validation. Future research should therefore target stochastic \ac{spat} and queue modelling, the use of surrogate policies or machine learning for state reduction to enable on-board, real-time deployment, and multi-intersection and network-level coordination. Furthermore, robustness to communication latency and packet loss needs to be addressed, and extensive real-world experiments are required to validate and refine these \ac{dp}-based eco-driving strategies.

\section{Literature Synthesis and Research Gap}
\label{sec:research_gap}

The aforementioned literature review of \ac{glosa} uncovers a consensus that these systems can decrease stop-and-go behaviour and, under specific circumstances, reduce energy consumption. However, the reported findings are highly varied and context-specific, which points to several underlying research gaps.

\subsection{Synthesis of Existing Work}
Most studies concur that \ac{glosa} systems reduce the frequency of stops. Under light traffic conditions, this generally leads to lower energy use. The reported average fuel savings span a wide range, from $11$--$35\%$ for flow-focused controllers to $25$--$68\%$ \cite{Guo2019,Typaldos2023,Cai2008} for eco-centric variants \cite{Park2024,Pulvirenti2023,Dong2024}. This significant spread can be attributed to three main factors. First, the optimisation objectives differ, with some studies prioritising delay reduction while others focus on fuel minimisation. Second, the car-following models used for simulation diverge significantly. Most studies rely on default \ac{idm} parameters, which tend to underestimate real-world acceleration and deceleration behaviours in dense traffic. In contrast, calibrated \ac{eidm} parameters more accurately reflect realistic driving dynamics, which can lead to different performance outcomes. Third, the \ac{mpr} and demand levels vary considerably across studies. Many simulations are conducted with relatively low traffic volumes of around $500\unit{\veh\per\hour}$ \cite{Yang2017, Ala2016}, which does not adequately represent conditions at heavily utilised urban intersections, where densities can frequently exceed $1000\unit{\veh\per\hour}$. These observations indicate that while current evidence is promising, it remains fragmented and context-specific, which highlights systematic biases that give rise to the following research gaps.
\mynewline
The following research gaps are identified as a result of the systematic biases that are revealed by the consolidated picture.

\subsection{Identified Research Gaps}
\begin{enumerate}
    \item \textbf{Penetration–Density Interaction.} Only a handful of studies have systematically varied both the \ac{mpr} and traffic density on a calibrated urban node. Most have kept demand at approximately $500\unit{\veh\per\hour}$, thereby under-representing peak-period conditions.
    \item \textbf{Fuel-Model Sensitivity.} No work has compared emission maps, such as HBEFA4 and the PHEMlight5 model, under identical scenarios. This impedes cross-study coherence and makes it difficult to generalise findings.
    \item \textbf{Driver-Heterogeneity Representation.} Evaluations have almost always applied the default \ac{idm}. Calibrated \ac{eidm} parameters for stop-and-go traffic remain largely unexplored, despite evidence that the \ac{idm} underestimates queue discharge times at high demand.
    \item \textbf{Eco–Flow Trade-Off at a Network Scale.} The systemic impact of eco-driving vehicles operating in an egoistic manner on overall throughput is poorly quantified. Existing results are limited to isolated junctions.
    \item \textbf{Benchmark Realism.} No standard intersection or openly shared \ac{sumo} scenario exists. Most studies simulate stylised layouts, which hinders replication. Field data are scarce outside of small pilot trials.
\end{enumerate}
Filling these gaps encapsulates this thesis's contribution.

\subsubsection{Thesis Contribution Anchor}
\label{subsubsec:thesis_contribution_anchor}

This thesis advances the field in three primary ways. First, it replicates the flow-oriented results of Lenz \cite{Lenz2024} by designing an \ac{eco-glosa} controller for the Stuttgart-Neckartor intersection. This is tested under identical conditions with traffic volumes up to $3500\unit{\veh\per\hour}$, using a controller based on \ac{dp} and a calibrated \ac{eidm}. Second, the research maps the penetration-density surface, which allows for the identification of thresholds where eco-guidance benefits or harms fuel consumption and traffic flow. This analysis also quantifies the collateral effects on non-equipped vehicles across all \ac{mpr} bands. Finally, the work provides an open-source codebase and a fully reproducible \ac{sumo} package. This is intended to serve as a standard benchmark for future studies in \ac{dp} and even \ac{rl}.
\mynewline
The next chapter outlines the methodological framework that operationalises these goals.