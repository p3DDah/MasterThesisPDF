\section{Objectives and Research Questions}
\label{sec:Objectives_and_Research_Questions}

Building on the challenges outlined in the preceding sections, this section articulates the concrete objectives of the thesis and derives the research questions that guide the empirical investigation.

\subsection{Objectives}
\label{subsec:Objectives}
The thesis pursues four interrelated objectives. 
\emph{First}, an \ac{eco-glosa} controller is designed to minimize fuel consumption explicitly. This is achieved by casting the speed-trajectory optimization problem as a \ac{dp} scheme subject to traffic-signal constraints. 
\emph{Second}, this controller is implemented in \ac{sumo} and coupled to the simulator via \ac{traci}, which enables real-time updates of vehicle speed profiles under varying market-penetration levels. 
\emph{Third}, the Stuttgart–Neckartor intersection is reconstructed within \ac{sumo}, integrating measured demand patterns and real signal-timing plans to ensure a high degree of realism. To capture nuanced driver behaviour at signalised junctions, the \ac{eidm} available in \ac{sumo} is utilized. The \ac{eidm} by Salles et al. enhances the standard IDM by Treiber et al. \cite{Treiber_2000} with reaction-time delays, estimation errors, and lane-change considerations, and is specifically tuned to reproduce realistic start-up and jerk dynamics in queued traffic. \cite{Salles2022}
\emph{Finally}, evidence from extensive simulation experiments, which vary both \acp{mpr} and traffic densities, is synthesized to derive deployment guidelines. These guidelines identify the operating regimes in which \ac{eco-glosa} yields net energy savings without degrading flow.

\subsection{Research Questions}
\label{subsec:Research_Questions}
The empirical work is structured around three \acp{rq}:
\begin{enumerate}[label=RQ\arabic*, ref=RQ\arabic*]
  \item \label{rq1} \emph{Fuel Efficiency}: To what extent does the proposed \ac{eco-glosa} algorithm reduce average fuel consumption in the Stuttgart-Neckartor scenario relative to conventional driving (Standard) and \ac{flow-glosa} across penetration rates from $10\%$ to $100\%$?
  \item \label{rq2} \emph{Traffic-Flow Effects}: How does the presence of \ac{eco-glosa}-equipped vehicles impact network-level traffic performance, measured by average travel time, stop frequency, acceleration and braking intensity, and green-phase utilisation, when compared to \ac{flow-glosa} and Standard conditions under varying \acp{mpr}?
  \item \label{rq3} \emph{Critical Thresholds}: Which combinations of \ac{mpr} and \ac{eco-glosa} penetration rate yield statistically significant improvements, and which lead to deteriorations, in energy and flow metrics when benchmarked against the alternative scenarios?
\end{enumerate}