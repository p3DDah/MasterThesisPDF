\section{Objectives and Research Questions}
\label{sec:Objectives_and_Research_Questions}

Building on the challenges outlined in the preceding sections, we now articulate the concrete objectives of this thesis and derive the research questions that guide the empirical investigation.

\subsection{Objectives}
\label{subsec:Objectives}
The thesis pursues four inter-related objectives. 
First, we aim to design an \emph{\ac{eco-glosa}} controller that explicitly minimizes fuel consumption by casting the speed-trajectory optimization problem as a \ac{dp} scheme subject to traffic-signal constraints. 
Second, this controller will be implemented in \ac{sumo} and coupled to the simulator via \ac{traci}, enabling real-time updates of vehicle speed profiles under varying market-penetration levels. 
Third, we will reconstruct the Stuttgart–Neckartor intersection within \ac{sumo}, integrating measured demand patterns and real signal-timing plans to ensure a high degree of realism. To capture nuanced driver behaviour at signalised junctions, we utilise the \ac{eidm} available in \ac{sumo}. The \ac{eidm} by Salles et al. enhances the standard IDM by Treiber et al. \cite{Treiber_2000} with reaction-time delays, estimation errors, and lane-change considerations, and is specifically tuned to reproduce realistic start-up and jerk dynamics in queued traffic. \cite{Salles2022}
Finally, we will synthesize evidence from extensive simulation experiments --- varying both penetration rates and traffic densities --- to derive deployment guidelines that identify the operating regimes in which \ac{eco-glosa} yields net energy savings without degrading flow.

\subsection{Research Questions}
\label{subsec:Research_Questions}
The empirical work is structured around three \acp{rq}:
\begin{enumerate}[label=RQ\arabic*, ref=RQ\arabic*]
  \item \label{rq1} \emph{Fuel efficiency}: To what extent does the proposed \ac{eco-glosa} algorithm reduce average fuel consumption in the Stuttgart-Neckartor scenario relative to conventional driving and \ac{flow-glosa} across penetration rates from 10\,\% to 100\,\%?
  \item \label{rq2} \emph{Traffic-flow effects}: How does the presence of \ac{eco-glosa}-equipped vehicles impact network-level traffic performance—measured by average travel time, stop frequency, acceleration and braking intensity, and green-phase utilisation—when compared to flow-optimised and no-GLOSA conditions under varying traffic densities?
  \item \label{rq3} \emph{Critical thresholds}: Which combinations of traffic density and \ac{eco-glosa} penetration rate yield statistically significant improvements, and which lead to deteriorations, in energy and flow metrics when benchmarked against the alternative scenarios?
\end{enumerate}