\section{Thesis Structure}
\label{sec:Thesis_Structure}

This thesis is organised into seven chapters, charting a course from the initial problem statement through to the final conclusions and future outlook. The structure is designed to build a coherent and logical argument, beginning with the necessary theoretical foundations before proceeding to the methodology, empirical results, and their interpretation.
\mynewline
The investigation commences in \textbf{Chapter \ref{ch:Introduction}}, which establishes the context and motivation for the research. It delineates the pressing challenges in urban mobility that this work addresses, culminating in a precise problem statement, a set of concrete project tasks, and the specific research questions that guide the inquiry.
\mynewline
With the research scope defined, \textbf{Chapter \ref{ch:fundamental_concepts}} provides the essential theoretical and technical background. This chapter delves into the core functionalities of the \ac{sumo} traffic simulation suite (\ref{sec:SUMO}) and explains the principles and architectures of \ac{glosa} systems (\ref{sec:glosa}). Together, these sections furnish the reader with the knowledge required to comprehend the subsequent technical discussions.
\mynewline
Building upon this foundation, \textbf{Chapter \ref{ch:State_of_the_Art}} presents a critical review of the current state of the art. It surveys the literature on dynamic programming approaches for \ac{glosa}, examining strategies optimised for both traffic throughput (\ref{sec:flow_glosa}) and energy efficiency (\ref{sec:eco_glosa}). This analysis identifies the unresolved questions and methodological gaps in existing research (\ref{sec:research_gap}), thereby positioning the contribution of this thesis.
\mynewline
In response to these identified gaps, \textbf{Chapter \ref{ch:Methodology_System_Design}} details the design and implementation of the novel control algorithms developed for this study. It outlines the architectural details of both the baseline flow-optimised controller (\ref{sec:Baseline_Glosa_Algorithm}) and the proposed eco-driving GLOSA algorithm (\ref{sec:Proposed_Eco_Driving_GLOSA_Algorithm}), explaining their underlying optimisation frameworks.
\mynewline
To empirically test these algorithms, a rigorous experimental protocol was devised, as documented in \textbf{Chapter \ref{ch:SimulationSetupConfiguration}}. This chapter describes the simulation environment, which is a high-fidelity digital twin of the Stuttgart-Neckartor junction (\ref{sec:SimEnvironment}), the parameterisation of the extensive simulation scenarios (\ref{sec:exec_protocol}), and the specific performance metrics chosen for the evaluation (\ref{sec:performance_evaluation}).
\mynewline
The empirical heart of the thesis is \textbf{Chapter \ref{ch:ResultsDiscussion}}, which presents a comprehensive analysis of the simulation outcomes. It systematically evaluates the performance of the controllers across a wide range of traffic conditions, focusing on metrics such as traffic flow, vehicle speed, stop frequency, emissions, and computational load. A detailed discussion (\ref{sec:Results_Discussion}) synthesises these results to expose the fundamental trade-offs between the competing control strategies and directly answers the core research questions.
\mynewline
Finally, \textbf{Chapter \ref{ch:SummaryOutlook}} brings the thesis to a close. It offers a consolidated summary of the key findings and their broader implications. Subsequently, it provides an outlook on promising future research directions, suggesting how the methodologies and insights from this work can be expanded upon to further advance the field of intelligent and sustainable transportation systems.
\mynewline
Supplementary data, including the detailed emission lookup tables (\ref{app:emission-tables}), are provided in the \textbf{Appendices}.