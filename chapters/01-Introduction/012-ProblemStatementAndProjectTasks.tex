\section{Problem Statement and Project Tasks}
\label{sec:Problem_Statement_and_Project_Tasks}
In the following, we first articulate the fundamental research gaps and limitations of current \ac{glosa} implementations that motivate this work. This problem statement will identify both conceptual and methodological shortcomings, ranging from energy-efficiency considerations to scalability in mixed-traffic scenarios, that the project tasks are designed to address.

\subsection{Problem Statement}
\label{subsec:Problem_Statement}
Numerous studies have demonstrated that \acp{glosa} can improve traffic flow. However, their impact on fuel consumption and emissions has received far less attention. Most existing implementations optimize for throughput by minimizing stops and delays, without explicitly targeting energy efficiency \cite{COPPOLA2022103455}. Evaluations to date have also been limited. They are often small-scale tests involving only a handful of equipped vehicles. This makes it difficult to generalize results to realistic penetration rates. The interaction between the \ac{glosa} penetration rate and overall traffic density remains unclear, as does how this interplay affects aggregate fuel savings. In addition, many assessments rely on simplified car-following models like the \ac{idm} by Treiber et al. \cite{Treiber_2000} Such models may not capture driver heterogeneity or second-order effects in mixed-traffic scenarios. Previous work has tended to focus on the benefits for individual vehicles. It has not quantified network-level impacts when multiple vehicles adopt eco-driving strategies, such as \ac{eco-glosa}, simultaneously. Consequently, a meaningful comparison of reported savings is challenging. The choice of fuel consumption model, such as PHEMlight5, HBEFA4, or VT-Micro, can significantly affect outcomes, yet few studies evaluate these models under standardized traffic and control conditions.

\subsection{Project Tasks}
\label{subsec:Project_Tasks}
To address these gaps, this thesis will pursue the following tasks:
\begin{enumerate}
    \item \textbf{\ac{eco-glosa} Algorithm Design.} Develop an extension of the standard \ac{glosa} logic that explicitly minimizes fuel consumption using \ac{dp} and eco-driving principles.
    \item \textbf{Implementation in SUMO.} Create the \ac{eco-glosa} algorithm in Python and use the \ac{traci} interface to integrate it with \ac{sumo}.
    \item \textbf{Scenario Reconstruction.} Recreate the Stuttgart–Neckartor reference scenario in \ac{sumo}, incorporating measured vehicle trajectories and the real signal-timing plan.
    \item \textbf{Parameter Specification.} Define a comprehensive set of simulation parameters, including \ac{glosa} penetration rates (0–100\%) and traffic densities ($\sim 70\text{--}3500\unit{\veh\per\hour}$).
    \item \textbf{Simulation Experiments.} Execute three simulation scenarios, which are Standard (no \ac{glosa}), Baseline \ac{flow-glosa}, and Eco \ac{eco-glosa}, across the specified parameter grid.
    \item \textbf{Data Collection and Metrics.} Record key performance indicators such as average \ac{co2} and \ac{nox} emissions per kilometre, vehicle class–specific emission levels, average speed, acceleration, jerk, number of stops, and green phase flow efficiency.
    \item \textbf{Comparative and Break-even Analysis.} Conduct a direct comparison of performance metrics across all scenarios to quantify the trade-offs between the control strategies. A break-even analysis is performed to identify the thresholds at which \ac{eco-glosa} becomes more fuel-efficient than the alternative scenarios.
    \item \textbf{Sensitivity Analysis.}  Evaluate how results vary when using different fuel‐consumption models (e.g. PHEMlight5 vs. HBEFA4).
\end{enumerate}
