\section{Background and Motivation}
\label{sec:Background_and_Motivation}

Urban areas around the world are experiencing a steady increase in vehicle ownership and use, driven by rapid urbanisation and population growth. This surge strains existing road networks, causing substantial economic and productivity losses through travel delays and excess fuel consumption, and increasing greenhouse gas emissions and local air pollutants such as \acl{nox} (\acs{nox}) and \ac{pm}. In response, the European Union and national authorities have enacted ambitious \ac{co2} reduction targets for the transport sector, including legally binding climate neutrality by 2050 under the European Climate Law and alignment with the Paris Agreement, supported by comprehensive long-term strategies and investments in sustainable mobility technologies. \cite{eclts2050}
\mynewline
Efficient traffic flow mitigates environmental impacts by reducing carbon footprints and noise pollution, while improving safety by reducing driver stress and minimising stop‐and‐go behaviour. At the same time, the rising costs of fuel, vehicle maintenance, and time lost in congestion underscore the economic necessity of optimising network performance. Furthermore, multimodal coordination, such as prioritised signal phases for buses and emergency vehicles, demonstrates the need for intelligent adaptive solutions that can dynamically accommodate diverse traffic demands. 
\mynewline
A promising solution lies in \acp{c-its}, which leverage \ac{v2x} communications to exchange real-time information between vehicles, infrastructure, and other road users. The Green Light Optimal Speed Advisory (\ac{glosa}) system uses \ac{spat} messages broadcast by traffic signals to calculate an optimal speed trajectory, allowing vehicles to pass intersections without unnecessary stops, and thus reducing energy consumption and travel time. \cite{COPPOLA2022103455} The \ac{glosa} system is designed for integration into connected vehicle architectures and external platforms such as smartphone applications or aftermarket devices, allowing deployment in existing vehicle fleets, as well as next-generation autonomous vehicles. 
\mynewline
The benefits of \ac{glosa} are further amplified when integrated with \ac{asct} and network-level optimization algorithms. Extensions to autonomous driving scenarios, such as eco-driving strategies, vehicle platooning, and cooperative manoeuvring, create additional synergies that can boost energy savings and throughput. Finally, embedding \ac{glosa} within a comprehensive Car-to-X (\ac{c2x}) communication framework, standardized by ETSI under the ITS Protocol Reference Architecture and employing IEEE 802.11p (ITS-G5) or \ac{c-v2x} technologies, ensures interoperability and paves the way for broad deployment of intelligent, sustainable mobility solutions. \cite{etsi_tr_102638_v2_1_1}
