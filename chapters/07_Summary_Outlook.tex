\chapter{Summary and Outlook}
\label{ch:SummaryOutlook}

The final chapter of this thesis serves a dual purpose, namely to consolidate the research presented thus far and to provide a conclusion. Firstly, it provides a comprehensive summary of the work undertaken, systematically recounting the objectives, methodology, and principal findings derived from the extensive simulation campaign. The objective of this study is to synthesise the key contributions and crystallise the core insights regarding the complex interplay between ecological driving strategies and traffic-flow dynamics. Secondly, this chapter presents a forward-looking perspective, outlining promising avenues for future research that emerge directly from the conclusions and identified limitations of this study. The outlook delineates next-generation control paradigms and expanded research scopes with a view to charting a course for the continued development of intelligent transportation systems that are both environmentally sustainable and operationally efficient.

\section{Summary}
\label{sec:Summary}

The increasing challenges posed by urban congestion and transport-related emissions make the development of advanced driver-assistance systems necessary. Among these, \ac{glosa} technology presents a promising approach to mitigating the negative impacts of signalised intersections, which are a primary source of fuel inefficiency and delay in urban networks. However, the optimal control strategy for \ac{glosa} systems is not self-evident, giving rise to a fundamental conflict between optimising individual vehicle fuel economy and maximising network-wide traffic throughput. This thesis addressed this conflict by designing, implementing, and carefully evaluating two distinct controllers within a high-fidelity simulation of the real-world Stuttgart-Neckartor intersection.
\mynewline
The research was guided by three core questions. The first (\vref{rq1}) investigated the fuel-efficiency benefits of a purpose-built \ac{eco-glosa} algorithm relative to both a baseline without \ac{glosa} and a benchmark \ac{flow-glosa} controller across a wide spectrum of market penetration rates. The second (\vref{rq2}) examined the corresponding effects of these controllers on network-level traffic performance, using metrics such as travel time, stop frequency, and green-phase utilisation. Finally, the third research question (\vref{rq3}) sought to identify the critical thresholds of traffic density and \ac{mpr} that determine whether the \ac{eco-glosa} strategy yields a net benefit or a net detriment to the system.
\mynewline
To answer these questions, a multi-faceted methodology was employed. A novel \ac{eco-glosa} controller was designed, casting the speed trajectory optimisation problem as a \ac{dp} scheme whose cost function aimed to minimise instantaneous fuel consumption. This was contrasted with a simpler, heuristic-based \ac{flow-glosa} controller designed to maximise traffic flow by ensuring vehicles arrived at the start of the green phase. Both controllers were implemented in the microscopic traffic simulator \ac{sumo} and coupled via the \ac{traci} interface to enable real-time vehicle control. A detailed digital twin of the Stuttgart-Neckartor intersection was used, incorporating measured traffic demand patterns and real-world signal timing plans. To ensure a baseline of realistic driving dynamics, the \ac{eidm} was employed across the entire simulated vehicle population. This foundational model governed all longitudinal and lateral movements, including car-following logic and lane-change decisions. Its implementation served as a critical fallback mechanism to guarantee collision-free interactions and maintain stable, human-like traffic behaviour for all agents in the simulation. The robustness of the environmental findings was ensured by evaluating emissions using two distinct models: the widely used HBEFA4 and the more physically grounded PHEMlight5. The experimental campaign systematically varied traffic demand from $69$ to $3462\unit{\veh\per\hour}$, from under-saturated to fully saturated conditions, and \acp{mpr} from $0\%$ to $100\%$.
\mynewline
The results of this investigation yielded several key insights. Regarding fuel efficiency (\vref{rq1}), the \ac{eco-glosa} controller demonstrated a pronounced, bimodal performance. In low-to-moderate traffic densities ($q < 1385\unit{\veh\per\hour}$), it successfully reduced \ac{co2} emissions by up to $7.7\%$ compared to standard driving. This was achieved by effectively smoothing speed profiles and avoiding unnecessary stops. However, in high-density traffic, this strategy proved counterproductive. The controller's narrow focus on individual vehicle economy led it to prescribe speed profiles that were incompatible with the dense traffic stream, inducing severe stop-and-go waves. This resulted in a notable increase in fuel consumption, with emissions rising by over $160\%$ in some scenarios. The \ac{flow-glosa} controller, while not explicitly optimising for fuel, delivered substantial emission reductions in these high-density regimes by effectively preventing stops --- the most fuel-intensive part of urban driving.
\mynewline
Concerning traffic-flow effects (\vref{rq2}), the \ac{flow-glosa} controller was unequivocally superior across all metrics in congested conditions. It consistently increased mean speeds, raised green-phase throughput, and, most significantly, was capable of preventing severe gridlock at high \acp{mpr} in saturated scenarios. The \ac{eco-glosa} controller, conversely, degraded traffic performance as density increased, causing lower speeds and a higher frequency of stops, thereby creating the very congestion it was intended to mitigate.
\mynewline
This led directly to the answer for \vref{rq3}, which identified clear operating regimes. Three distinct zones were identified based on traffic density ($\gls{q}$) and penetration rate ($\gls{p}$). In the under-saturated regime ($\gls{q} < 700\unit{veh/h}$), \ac{eco-glosa} provides reliable benefits compared to the Standard and \ac{flow-glosa} case. In the saturated regime ($\gls{q} > 2300\unit{veh/h}$), \ac{flow-glosa} is dominant and essential for preventing gridlock. In the wide and volatile transition range between these two, the risk of \ac{eco-glosa} inducing congestion makes the throughput-focused \ac{flow-glosa} the most prudent and reliable choice for deployment.
\mynewline
In conclusion, the principal contribution of this thesis is the systematic demonstration that an egoistic, vehicle-level optimisation for fuel economy can lead to severe, negative network-level externalities. It establishes that a \enquote{one-size-fits-all} \ac{glosa} strategy is not viable. For congested urban corridors, the research provides strong evidence that strategies prioritising traffic flow are a more robust and effective means of achieving system-wide emission reductions than those that narrowly focus on individual vehicle eco-driving.

\section{Outlook}
\label{sec:Outlook}

Building on the conclusions and limitations of this thesis, several promising avenues for future research can be identified. These avenues range from direct, incremental extensions of the current work to the exploration of more advanced, paradigm-shifting control methodologies.
\mynewline
A primary and immediate extension is the development of \textit{hybrid and adaptive control strategies}. The clear delineation of operating regimes in this thesis strongly suggests that a static controller is inherently limited. The logical next step is to design a meta-controller capable of dynamically switching between the \ac{eco-glosa} and \ac{flow-glosa} logic based on real-time traffic conditions. This adaptive system would use data from loop detectors or aggregated information from \acp{cav} to estimate the current traffic state ($\gls{q}, \gls{p}$) and select the most appropriate control objective. Such a system would combine the best of both worlds: capitalising on the fuel-saving potential of \ac{eco-glosa} in light traffic while defaulting to the robust, jam-preventing capabilities of \ac{flow-glosa} as congestion builds.
\mynewline
A second avenue of research involves refining the core objective function of the \ac{eco-glosa} algorithm itself. The current implementation can be characterised as a \textit{defensive} eco-driving strategy; its optimiser exclusively prioritises emission minimisation, a process that often leads to advising slower speeds to avoid high-cost acceleration states. A promising alternative would be to develop an \textit{offensive} or \textit{balanced} eco-driving mode. This approach would modify the underlying cost function of the \ac{dp} algorithm to incorporate a penalty for increased travel time. By assigning a variable weight to the time component, the controller's behaviour could be tuned along a spectrum --- from the purely defensive eco-mode to a more forceful, throughput-aware profile. Such a controller would still aim to find the most fuel-efficient path but would be incentivised to do so without significantly extending travel duration, creating a trade-off between the pure \ac{flow-glosa} and the defensive \ac{eco-glosa}.
\mynewline
While a rule-based hybrid or balanced controller is a viable next step, its effectiveness is ultimately constrained by the complex and dynamic nature of urban traffic. The fluctuations and emergent behaviours in these environments call for more advanced solutions. This leads to the promising field of \textit{\acf{rl}}. The \ac{dp}-based approach of the \ac{eco-glosa} controller, while optimal, was shown to be computationally expensive. An \ac{rl} agent could learn an effective speed control policy directly from interactions with the simulation environment, potentially discovering more efficient and robust strategies than a hand-crafted optimiser. More powerfully, an \ac{rl} agent's state representation can be augmented to include not just the vehicle's own state but also information about the local traffic environment (e.g., speed of the lead vehicle, distance to queue, local traffic density). This would enable the agent to learn a context-aware policy. Such a policy allows it to implicitly understand when a classic eco-driving trajectory is feasible and when it is not. In doing so, the agent learns the principles of a hybrid controller in a more integrated fashion.
\mynewline
Furthermore, this research highlighted that the failure of \ac{eco-glosa} stemmed from its \enquote{egoistic} objective function. This points towards a crucial paradigm shift from egoistic to \textit{cooperative control}. A truly intelligent system would involve vehicles coordinating their actions for the greater good of the traffic stream. Formulating this as a cooperative \ac{dp} problem is largely intractable, as the state space would explode with the number of interacting vehicles, falling victim to the \enquote{curse of dimensionality}. This is where \textit{Multi-Agent Reinforcement Learning (\ac{marl})} emerges as a frontier technology. In a \ac{marl} framework, each \ac{cav} acts as an independent agent that learns a policy. By carefully designing a shared reward function that incentivises system-level goals (e.g., maximising total throughput while minimising total fuel consumption of the cohort), the agents can learn complex, cooperative behaviours. They could learn to form tightly packed platoons to maximise green-band utilisation or to create gaps strategically to allow other vehicles to merge smoothly, all without a centralised controller dictating their every move.
\mynewline
Finally, the scope of the simulation environment itself must be expanded to increase the external validity of the findings. In the future, work should move forward along three main lines:
\begin{enumerate}
    \item \textbf{Network-level analysis:} The single-intersection model should be extended to a corridor with multiple, coordinated intersections and eventually to a full urban grid. This is essential for studying critical network phenomena such as queue spill-back, route choice adaptation, and the performance of green waves in the presence of \ac{glosa}-equipped vehicles.
    \item \textbf{Fleet heterogeneity:} The assumption of a homogenous fleet should be relaxed. Future simulations must incorporate a mix of vehicle types, including internal combustion engines, hybrids, and fully \acp{ev}. The optimal energy strategy for an \ac{ev}, which can recover energy through regenerative braking, is fundamentally different from that of a conventional vehicle. Investigating how these different powertrain types interact within a \ac{glosa} system is a critical and understudied research area.
    \item \textbf{System integration:} The \ac{glosa} system should not be considered in isolation. A vital research direction is its integration with adaptive traffic signal control. A truly symbiotic system would involve bidirectional communication, where vehicles inform the infrastructure of their intended trajectories, and the traffic signals, in turn, adjust their timing to better accommodate the approaching platoons.
\end{enumerate}

The logical trajectory of this research path leads from the digital realm to physical reality. The most promising controllers developed in expanded simulations would need to be validated through \ac{hil} testing and, eventually, in \acp{fot}. Rigorous, real-world validation is therefore an essential final step. Only through such empirical testing can the full potential of \ac{glosa} systems be unlocked. This will ultimately enable the creation of more intelligent, efficient, and sustainable urban transport.