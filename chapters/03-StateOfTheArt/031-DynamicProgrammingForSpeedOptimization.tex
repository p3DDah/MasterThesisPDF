\section{Dynamic Programming for Throughput-Oriented \ac{glosa}}
\label{sec:flow_glosa}

\ac{dp} is a principled method for solving sequential decision problems. In the context of \ac{glosa}, it can be used to compute speed advisories that maximise intersection throughput and minimise vehicle delays. Throughput-oriented \ac{glosa} aims to reduce cumulative delay, the number of stops, and total travel time, subject to traffic signal constraints. Fuel consumption is treated as a secondary objective in this mode; it will become the primary focus in Section~\ref{sec:eco_glosa}. A clear delineation between the two modes clarifies their respective trade-offs.
\mynewline
The core idea is to model each vehicle’s approach as a discrete-time, discrete-space control problem. State variables capture the vehicle’s position and speed at each time step. Decision variables represent a finite set of acceleration actions. A cost function assigns an incremental delay penalty per time step. A terminal constraint enforces that the vehicle crosses the stop line during a green phase. Solving this problem via \ac{dp} yields a locally or globally optimal speed profile under the given objectives and constraints.

\subsection{Objectives and Problem Formulation}
\label{subsec:flow_dp_formulation}

The primary objective of throughput-oriented \ac{glosa} is to minimise the cumulative delay experienced by a vehicle when approaching a signalised intersection. Cumulative delay is defined as the difference between actual travel time and the ideal travel time at free-flow speed. Secondary objectives include minimising the number of full stops and the total travel time. Although fuel consumption reduction is desirable, it remains a lower-priority goal in this mode.
\mynewline
To capture these objectives, a constrained optimal control problem is formulated on a discrete grid. Time is divided into $K$ intervals of length $\Delta t$. The state at step $k$ is represented as:
\begin{equation}
    s_k = \left(x_k, v_k\right),
\end{equation}
where $x_k$ is the distance from the stop line and $v_k$ is the vehicle's speed. The action at step $k$ is a discrete acceleration $a_k \in \mathcal{A}$, where:
\begin{equation}
    \mathcal{A} = \{a_1, a_2, \dots, a_{N_a}\},
\end{equation}
and $a_{\min} \le a_i \le a_{\max}$. The state evolves according to the kinematic equations:
\begin{equation}
    x_{k+1} = x_k + v_k\Delta t + \frac{1}{2}a_k\Delta t^2,
\end{equation}
\begin{equation}
    v_{k+1} = \max\left(0, v_k + a_k\Delta t\right).
\end{equation}
The instantaneous cost is defined as:
\begin{equation}
    c(s_k, a_k) = \Delta t - \frac{x_{k+1}-x_k}{v_{\mathrm{free}}} + \lambda_s\mathbf{1}_{\{v_{k+1}=0\}},
\end{equation}
where $v_{\mathrm{free}}$ is the free-flow speed, $\lambda_s$ is a stop penalty weight, and the indicator term penalises a full stop. This cost function measures the delay relative to free-flow travel, plus a penalty for stopping.
\mynewline
Signal timing constraints are enforced via a large terminal penalty. Let the green phase interval for the approaching movement be denoted as $[t_{g,\mathrm{start}}, t_{g,\mathrm{end}}]$ and let $t_k=k\Delta t$. If the vehicle has not crossed the stop line by the end of the horizon or arrives outside the green window, a penalty $M$ is added. The terminal cost is thus defined as:
\begin{equation}
c_T(s_K) =
\begin{cases}
0, & \text{if } x_K \ge L_s \text{ and } t_K\in[t_{g,\mathrm{start}},t_{g,\mathrm{end}}] \\
M, & \text{otherwise}
\end{cases}
,
\end{equation}
where $L_s$ is the distance to the stop line. The choice of a large penalty, $M\gg1$, ensures that any feasible green-phase arrival is preferred.
\mynewline
The full optimisation problem is therefore to find the sequence of actions that minimizes the total cost:
\begin{equation}
    \min_{\{a_k\}_{k=0}^{K-1}} \quad J = \sum_{k=0}^{K-1} c(s_k,a_k) + c_T(s_K),
\end{equation}
subject to the state dynamics and action bounds. The solution to this problem is obtained by standard backward induction in discrete \ac{dp}.
\mynewline
The key modelling choices include the following:
\begin{itemize}
    \item \textbf{State Discretisation:} Positions are quantised at a resolution of $\delta x$, and speeds are quantised at a resolution of $\delta v$. A finer resolution improves accuracy but also increases the computational load.
    \item \textbf{Action Set Size:} The number of discrete accelerations, $N_a$, represents a trade-off between solution fidelity and solver speed.
    \item \textbf{Horizon Selection:} The time horizon, defined as $T=K\Delta t$, must extend beyond the latest feasible end time of a green phase.
    \item \textbf{Cost Weights:} The stop penalty, $\lambda_s$, and the green-phase violation penalty, $M$, must be tuned to reflect the system's priorities.
\end{itemize}
In practice, the \ac{dp} grid and cost parameters are configured offline. During runtime, real-time signal status and vehicle state information are fed into the \ac{dp} solver. An algorithm, such as value iteration or policy iteration, computes the optimal sequence of actions. The first action in this sequence provides the immediate speed advisory. In subsequent steps, the problem is re-optimised as new signal updates become available.
\mynewline
Throughput-oriented \ac{dp} methods can significantly reduce vehicle delay and travel time. For instance, Guo et al. \cite{Guo2019} report travel time reductions of $23.6$--$35.7\%$ compared to adaptive signal control. Similarly, Cai et al. \cite{Cai2008} show a $12\%$ delay reduction relative to the baseline TRANSYT model and a $23\%$ travel time reduction compared to uncoordinated operations. The computational complexity of these methods grows with finer discretisation and longer planning horizons. Therefore, a careful selection of state and action resolutions is essential. Nevertheless, with sensible parameter settings, computation times can remain well below the planning horizon, which enables real-time implementation. The following section, \vref{subsec:flow_dp_algorithms_limitations}, surveys various algorithmic approaches and their limitations.

\subsection{Algorithmic Approaches and Limitations}
\label{subsec:flow_dp_algorithms_limitations}

Throughput-oriented \ac{dp} for \ac{glosa} has evolved considerably since the early 2000s. Seven main research trajectories can be classified by their approach to the trade-off between advisory optimality and real-time feasibility, ranging from exact, full-horizon optimisers to lightweight, driver-centric heuristics. The methods are assessed in the subsequent section on the basis of their quantitative performance enhancements, computational or practical expenses, and the primary constraints that have been noted.
\mynewline
The computational baseline is full-horizon, grid-based \ac{sdp}, whose optimality is provable yet costly. This approach involves discretising both space (a vehicle's position relative to the stop line) and speed into a finite two-dimensional lattice. It then performs backward value iteration on the Bellman equation under a terminal \enquote{green-phase} constraint. Typaldos and Papageorgiou \cite{Typaldos2021} formulate the stochastic \ac{glosa} problem as a discrete-time \ac{sdp}. They report that due to the exponential growth of the discretised state-control domain, the one-shot \ac{sdp} algorithm may require several minutes of CPU time on a standard PC, which renders on-board real-time execution infeasible. Coarsening the discretisation, for example by increasing the grid step, reduces computation time but shrinks the feasible state domain. This could potentially exclude viable green-phase arrival states or degrade the quality of the advisory.
\mynewline
To reduce this cost without sacrificing optimality, Typaldos et al. \cite{Typaldos2023} shrink the state space by using a moving corridor. Receding-horizon \ac{dp}, which is often realised in an \ac{mpc} framework, has recently been accelerated through \ac{dddp}. This algorithm restricts each \ac{dp} iteration to a slender state-space corridor that is centred on the previously accepted trajectory. The corridor is shifted forward as new signal-state information becomes available, thereby retaining optimality while severely curtailing the search volume. In the benchmark stochastic \ac{glosa} scenario, \ac{dddp} replicated the full-horizon \ac{sdp} optimum of $J^\star=1.17517$, yet it slashed the total computation time from $614\unit{\second}$ to $0.69\unit{\second}$, which is nearly three orders of magnitude faster. Parametric studies covering a broad range of corridor widths and discretisation steps reported per-iteration CPU times spanning from $0.28\unit{\second}$ to $2.30\unit{\second}$, while consistently achieving the same optimal cost. This makes advisory refresh rates of approximately $1\unit{\hertz}$ feasible on standard on-board hardware. The chief drawback of corridor truncation is its sensitivity to the window size. If the green phase commences outside the predicted corridor, admissible arrival trajectories may be excluded, whereas overly wide corridors inflate the runtime. Typaldos et al. therefore advocate for adaptive corridor re-expansion and discretisation refinement to trade off convergence reliability against computational load.
\mynewline
Instead of pruning the state space, Cai et al. \cite{Cai2008} approximate the value function itself. Their method, \textit{Adaptive traffic-signal control via \ac{adp}}, embeds Bellman’s principle in a forward, rolling-horizon scheme. The value function is approximated by a linear feature map, which is updated online with either \ac{td} or \ac{pl}. Every $0.5\unit{s}$ (fine resolution) or $5\unit{s}$ (coarse resolution), the controller receives $10\unit{s}$ of detector data, predicts a further $10\unit{s}$, and evaluates three options: hold, immediate stage switch, or a deferred switch within the head window. It then implements only the first increment while propagating parameter updates.
In a three-stage isolated intersection, the fine-resolution \ac{adp} matched the full-horizon \ac{dp} optimum of $4.27\unit{veh\cdot s\per\second}$ within $0.35\unit{veh\cdot s\per\second}$ (\ac{adp}\textsubscript{TD} $4.62$, \ac{adp}\textsubscript{PL} $4.66$). This approach also cut the per-hour run-time from $720\unit{min}$ (\ac{dp}) to $5.5\unit{min}$ and $12\unit{min}$, respectively. Relative to optimised TRANSYT fixed-time plans ($13.95\unit{veh\cdot s\per\second}$), the \ac{adp} reduced delay by approximately $67\%$. Working at a $0.5\unit{s}$ resolution rather than $5\unit{s}$ yielded a further $41\%$ improvement, with the coarse \ac{adp}\textsubscript{PL} at $8.64\unit{veh\cdot s\per\second}$. Under a four-hour, time-varying demand profile, both learning rules converged to statistically indistinguishable means (\ac{adp}\textsubscript{PL} $3.24$ vs. \ac{adp}\textsubscript{TD} $3.28$; $p=0.43$). The \ac{td} method required less than half the CPU time because it computes only a single Bellman update per step. Sensitivity tests showed the best performance at a high discount rate of $h=0.12$, indicating that near-term costs dominate and justifying the simple linear approximation. The chief limitations are the reliance on large discounting, which dampens the value-function’s contribution, a linear state representation, and isolation to a single junction. The authors recommend extending the work to network-level coordination, parallelising the per-state cost evaluation, deriving constant-time queue-update formulas, and exploring non-linear or piecewise-linear value functions for stochastic arrivals and adaptive phase ordering.
\mynewline
A complementary path compresses complexity analytically. Samra et al. \cite{Samra2015} reformulate the \ac{dp} state to obtain linear time. They propose a linear-time and space \ac{dp} algorithm for optimal traffic-signal duration at an isolated intersection. The algorithm re-encodes the \ac{dp} state around the current time step and prunes non-promising predecessors early. This collapses the complexity of the classic \ac{cop} formulation from $O(T^{3})$ time and $O(T^{2})$ memory to $O(|P|T)$ for both metrics. A \textit{C} implementation on a $2.10\unit{\giga\hertz}$ Intel Core2-Duo achieved a speed-up of over $2700\times$ compared to the \ac{cop} for a 15-minute horizon with $T=1024$ one-second steps, finishing in $0.27\unit{\second}$ while matching the global optimum. The method generalises to heterogeneous minimum-green and clearance constraints by prefix-encoding constrained phases, which keeps the state space linear in $T$. In Green Light District simulations, the algorithm cut the average junction waiting time by approximately $63\%$ (a factor of $2.7$) under heavy demand (spawn rate of $0.4$) for a three-phase intersection, and by approximately $66\%$ (a factor of $3$) for a four-phase node relative to the strongest reinforcement-learning baseline, with consistently low run-to-run variance.
The limitations include a reliance on accurate short-term arrival forecasts, as prediction errors break optimality guarantees. The linear scaling is only in the horizon length and not yet across networks. Furthermore, an $O(|P|)$ per-state cost evaluation becomes salient for large phase sets. The authors, therefore, recommend several future steps. These include embedding the algorithm in coordinated frameworks such as RHODES, parallelising across cores to handle multi-intersection deployments, deriving constant-time queue updates to reduce the per-state cost to $O(1)$, and extending the formulation to stochastic arrivals and adaptive phase ordering while preserving worst-case linear bounds.
\mynewline
When analytical compression is not feasible, meta-heuristics such as \acp{ga} can be used to trade exactness for scalability. Researchers have turned to \emph{metaheuristic graph-search} methods to alleviate the combinatorial explosion inherent in multi-segment \ac{glosa}. Seredynski and Khadraoui \cite{Seredynski2013} model the advisory problem as a sequence of $n$ road segments. They encode each candidate solution as a vector of segment-wise speed advisories and solve the problem using a generational \ac{ga} that employs selection, single-point crossover, and mutation operators, iterating until a stop condition is met. Simulation results in non-congested networks with up to fifteen segments show that the multi-segment \ac{ga} approach significantly outperforms single-segment \ac{glosa} in both travel-time and fuel-efficiency metrics. Although the \ac{ga} circumvents the exhaustive enumeration required by \ac{dp}, its runtime still grows exponentially with the number of segments. Scaling beyond moderate route lengths remains infeasible on standard hardware without additional heuristic pruning or parallelization. The authors suggest that incorporating admissible heuristic functions, such as the remaining distance divided by the free-flow speed, could further prune the search space while guaranteeing constraint satisfaction.
\mynewline
Recently, researchers have embraced integrated control, which co-optimises signals and \ac{cav} trajectories. The \textit{joint optimisation of vehicle trajectories and intersection controllers} (DP–SH) method by Guo et al. \cite{Guo2019} couples a \ac{dp} signal optimiser with the \ac{sh} trajectory generator. This approach co-designs signal phases, green durations, and \ac{cav} speed profiles in a single forward recursion. The \ac{sh} algorithm restricts each car’s search to at most seven analytically solvable quadratic arcs. This allows the \ac{dp} to treat phases as stages and embed \ac{sh} as a subroutine, retrieving the optimal control whenever the horizon-wide value function improves by less than $5\%$. With a $122\unit{\second}$ planning horizon and an $8\unit{\second}$ step size, the algorithm computes a full four-phase plan in $9.8$–$15.2\unit{\second}$ for up to twenty vehicles per approach on a $1.8\unit{\giga\hertz}$ laptop, which is well below $10\%$ of real time. Larger step sizes can cut the runtime to $2.8\unit{\second}$ with only marginal performance loss. Across twelve demand scenarios, with segment lengths of $400$–$1200\unit{\metre}$ and saturation ratios of $0.6$--$1.5$, the DP–SH method with subsequent \ac{sh} parameter refinement trimmed the mean travel time by $23.6\%$--$35.7\%$ and fuel consumption by $11.8\%$--$31.5\%$ relative to adaptive signal control. Internal parameter tuning of acceleration bounds yielded a further $10\%$--$18\%$ fuel saving without sacrificing throughput. The framework remains robust in mixed traffic. Even at a $10\%$ \ac{cav} penetration, the average delay and consumption still drop by $3.1\%$ and $6.2\%$, respectively, rising monotonically to $34.8\%$ and $31.5\%$ as penetration approaches $100\%$. Sensitivity tests indicate that the computational load grows linearly with the horizon length but only weakly with the number of vehicles per phase. However, horizons below approximately $400\unit{\metre}$ may preclude feasible \ac{sh} trajectories under heavy queues, requiring either minimum-length constraints or upstream spill back prediction. The principal challenges are (i) non-convex parameter tuning that can stall in local minima, (ii) a reliance on accurate entry-boundary forecasts, (iii) a focus on isolated intersections, as spill back chains and multi-signal coordination remain open issues, and (iv) real-time scalability when the horizon or demand scales. The authors therefore advocate for parallelisation, distributed feedback control, an extension to corridor-level or signal-free \ac{cav} reservation policies, and the incorporation of stochastic capacity for a truly network-wide deployment.
\mynewline
At the opposite end of the spectrum, DC-\ac{glosa} achieves field-ready simplicity by relegating computation to a smartphone. The \textit{Driver-centric Green Light Optimal Speed Advisory (DC-GLOSA)}, proposed by Suramardhana and Jeong \cite{Suramardhana2014}, equips each vehicle with a smartphone-based on-board unit. This unit receives \ac{spat} messages from the roadside unit, estimates ego-motion via GPS, and issues one of seven instantaneous accelerations or braking advisories. The advisory is chosen by comparing the current \ac{ttc} against six analytically defined reference mobility models. These models include maximum acceleration, constant acceleration, constant speed, constant braking, maximum braking, and emergency deceleration to stop. This approach respects speed bounds of $v_{\min}=20\unit{\kilo\metre\per\hour}$ and $v_{\max}=40\unit{\kilo\metre\per\hour}$ while limiting the longitudinal jerk to $a\in[-3.82, 3.82]\unit{\metre\per\second\squared}$. The algorithm incrementally updates the advisory whenever a new \ac{spat} frame or a significant state change arrives. It defers freshly triggered voice prompts until the previous one has finished, which avoids cognitive overload by queuing at most the most recent message.
Field trials were conducted with a single vehicle at a three-arm signalised junction on the Pusan National University campus, with signal timings of $T_{\text{green}}=30\unit{\second}$, $T_{\text{amber}}=3\unit{\second}$, and $T_{\text{red}}=100\unit{\second}$. The results showed that DC-GLOSA reduced the average intersection-stopping time by $23.9\%$ over a baseline with no advisory. This was evaluated across 54 runs, spanning approach distances of $D=50\text{--}200\unit{\metre}$ and residual-green intervals of $T_{\text{ref}}=10\text{--}70\unit{\second}$. While the strategy improves throughput and driver comfort, two practical issues persist. First, the advisory latency from smartphone audio playback, which is approximately $2$--$3\unit{\second}$, can render a message stale in rapidly changing scenarios. Second, the scheme relies on accurate \ac{spat} and driver compliance. A failure to receive timely updates or aggressive driving can force emergency braking when $D_{\text{stop}}>D$. The authors therefore recommend future work on adaptive message suppression, multi-vehicle coordination to resolve conflicting advisories, and the integration of powertrain awareness and eco-cost functions to extend the current \ac{ttc} and emission-minimisation.
\mynewline
Viewed along the complexity axis, three patterns emerge. Exact methods, such as grid-based \ac{sdp} and \ac{dddp}, guarantee optimality but can strain real-time hardware. Approximate or analytical reductions, like \ac{adp} and linear-time \ac{dp}, temper this cost while preserving most of the benefits. Finally, heuristic and driver-centric schemes, such as \ac{ga}, \ac{dp}–\ac{sh}, and DC-\ac{glosa}, prioritise deployment ease and system-wide coordination, albeit at the risk of sub-optimality.

\paragraph{Key Quantitative Insights across Methods:}
\begin{itemize}
    \item \emph{Delay or Stopping-Time Reduction:} A reduction of $40\%$ was achieved in grid-based \ac{sdp}, $39\%$ in \ac{dddp} corridor studies, $23.9\%$ in the DC-\ac{glosa} field trial, and $30\%$--$36\%$ in \ac{dp}–\ac{sh} mixed-traffic scenarios.
    \item \emph{Computation Time per Advisory:} This ranged from $5$--$10\unit{\minute}$ for full-horizon \ac{sdp}, $0.28$--$2.30\unit{\second}$ for \ac{dddp}, $5.5$--$12\unit{\minute}$ per simulated hour with \ac{adp}, $0.27\unit{\second}$ for linear-time \ac{dp}, $2.8$--$15.2\unit{\second}$ for \ac{dp}–\ac{sh}, and sub-second for DC-\ac{glosa} smartphone prompts.
    \item \emph{Scalability Constraints:} These include exponential state-space growth with \ac{sdp}, corridor-width sensitivity in \ac{dddp}, feature-map bias for \ac{adp}, a per-state $O(|P|)$ cost for each phase in linear-time \ac{dp}, combinatorial explosion as the segment count increases in \ac{ga}, a reliance on-demand forecasts in \ac{dp}–\ac{sh}, and audio-latency and compliance issues in DC-\ac{glosa}.
\end{itemize}

\paragraph{Principal Limitations and Open Challenges.} 
First, \emph{computational tractability} remains the dominant hurdle: exact full-horizon \ac{sdp} and its corridor-pruned variant (DDDP) guarantee optimality but incur exponential or corridor-width-sensitive costs that exceed typical on-board budgets. Second, \emph{approximation bias} affects receding-horizon and feature-based ADP schemes --- incorrect corridor sizes may exclude feasible trajectories, while linear value‐function approximations and high discounting can distort long-term performance. Third, \emph{per-state cost scaling} persists: Samra et al.’s linear-time DP still requires $O(|P|)$ evaluations under large phase sets, and genetic algorithms face combinatorial explosion as segment count grows. Fourth, \emph{forecast and communication dependences} challenge DP–SH and DC-GLOSA: the former demands accurate boundary predictions and low-latency V2X links, the latter suffers 2–3s audio latency and driver compliance issues. Fifth, \emph{constraint enforcement} in approximate or continuous DP often relies on heuristic corrections for hard green-phase satisfaction, undermining formal guarantees. Sixth, \emph{secondary objectives} --- ride comfort, fuel economy and emissions --- remain largely orthogonal, multiplying the action space and complicating convergence. Finally, \emph{network scalability} is unresolved: nearly all methods target isolated intersections or single vehicles, and extending them to stochastic, multi-junction corridors with spill back and dynamic coordination stands as the grand challenge.
\mynewline
In conclusion, throughput-oriented \ac{dp} for GLOSA has delivered prototype-scale improvements --- up to 40\,\% delay reduction in grid-based \ac{sdp}, 39\,\% with receding-horizon DDDP, and 30–36\,\% in integrated co-optimisation --- while driver-centric and continuous methods produce sub-second advisories with moderate gains. Real-time deployment will require algorithms that reconcile computational efficiency, rigorous constraint satisfaction, and true multi-objective optimality. Future work must unite uncertainty-aware control, low-latency V2X communication, and network-level coordination to translate current prototypes into robust, large-scale field systems.
