\section{Dynamic Programming for Throughput-Oriented \ac{glosa}}
\label{sec:flow_glosa}

\ac{dp} is a principled method for solving sequential decision problems. In the context of \ac{glosa}, it can be used to compute speed advisories that maximise intersection throughput and minimise vehicle delays. Throughput-oriented \ac{glosa} aims to reduce cumulative delay, the number of stops, and total travel time, subject to traffic signal constraints. Fuel consumption is treated as a secondary objective in this mode; it will become the primary focus in Section~\ref{sec:eco_glosa}. A clear delineation between the two modes clarifies their respective trade-offs.
\mynewline
The core idea is to model each vehicle’s approach as a discrete-time, discrete-space control problem. State variables capture the vehicle’s position and speed at each time step. Decision variables represent a finite set of acceleration actions. A cost function assigns an incremental delay penalty per time step. A terminal constraint enforces that the vehicle crosses the stop line during a green phase. Solving this problem via \ac{dp} yields a locally or globally optimal speed profile under the given objectives and constraints.

\subsection{Objectives and Problem Formulation}
\label{subsec:flow_dp_formulation}

The primary objective of throughput-oriented \ac{glosa} is to minimise the cumulative delay experienced by a vehicle when approaching a signalised intersection. Cumulative delay is defined as the difference between actual travel time and the ideal travel time at free-flow speed. Secondary objectives include minimising the number of full stops and the total travel time. Although fuel consumption reduction is desirable, it remains a lower-priority goal in this mode; Section~\ref{sec:eco_glosa} will invert this priority.
\mynewline
To capture these objectives, we formulate a constrained optimal control problem on a discrete grid. Let time be divided into $K$ intervals of length $\Delta t$. The state at step $k$ is 
\[
s_k = \bigl(x_k,\;v_k\bigr),
\]
where $x_k$ is the distance from the stop line and $v_k$ is the vehicle speed. The action at step $k$ is a discrete acceleration $a_k\in\mathcal{A}$, where
\[
\mathcal{A}=\{a_1,\,a_2,\,\dots,\,a_{N_a}\},
\]
and $a_{\min}\le a_i\le a_{\max}$. The state evolves according to the kinematic equations:
\[
x_{k+1} = x_k + v_k\,\Delta t + \tfrac12\,a_k\,\Delta t^2,
\]
\[
v_{k+1} = \max\bigl(0,\;v_k + a_k\,\Delta t\bigr).
\]

The instantaneous cost is defined as
\[
c(s_k,a_k) 
= \Delta t \;-\; \frac{x_{k+1}-x_k}{v_{\mathrm{free}}}
\;+\;\lambda_s\,\mathbf{1}_{\{v_{k+1}=0\}},
\]
where $v_{\mathrm{free}}$ is the free-flow speed, $\lambda_s$ is a stop penalty weight, and the indicator term penalises a full stop. This cost measures delay relative to free-flow, plus a penalty for stopping.
\mynewline
Signal timing constraints are enforced via a large terminal penalty. Denote the green phase interval for the approaching movement as $[t_{g,\mathrm{start}},\,t_{g,\mathrm{end}}]$. Let $t_k=k\,\Delta t$. If the vehicle has not crossed the stop line by the end of the horizon or arrives outside the green window, a penalty $M$ is added:
\[
c_T(s_K) = 
\begin{cases}
0, & \text{if } x_K \ge L_s \text{ and } t_K\in[t_{g,\mathrm{start}},t_{g,\mathrm{end}}],\\
M, & \text{otherwise},
\end{cases}
\]
where $L_s$ is the distance to the stop line. The choice of $M\gg1$ ensures that any feasible green-phase arrival is preferred.

The full optimisation problem is therefore:
\[
\min_{\{a_k\}_{k=0}^{K-1}}
\quad
J = \sum_{k=0}^{K-1} c(s_k,a_k)\;+\;c_T(s_K),
\]
subject to the state dynamics and action bounds. The solution is obtained by standard backward induction in discrete \ac{dp}.
\mynewline
Key modelling choices include:
\begin{itemize}
  \item \textbf{State discretisation:} Positions are quantised at resolution $\delta x$, speeds at resolution $\delta v$. Finer resolution improves accuracy but increases computational load.
  \item \textbf{Action set size:} The number of accelerations $N_a$ trades off solution fidelity and solver speed.
  \item \textbf{Horizon selection:} The time horizon $T=K\,\Delta t$ must exceed the latest feasible green-phase end.
  \item \textbf{Cost weights:} The stop penalty $\lambda_s$ and green-phase violation penalty $M$ must be tuned to reflect system priorities.
\end{itemize}

In practice, the \ac{dp} grid and cost parameters are configured offline. At runtime, real-time signal status and vehicle state feed into the \ac{dp} solver. A value iteration or policy-iteration algorithm computes the optimal action sequence. The first action yields the immediate speed advisory. Subsequent steps re-optimise as new signal updates arrive.
\mynewline
Throughput-oriented \ac{dp} methods can significantly reduce vehicle delay and travel time. Guo et al.~\cite{Guo2019} report travel time reductions of 23.6–35.7\% compared to adaptive signal control. Cai et al.~\cite{Cai2008} show a 12\% delay reduction relative to baseline TRANSYT model and a 23\% travel time reduction compared to uncoordinated operations. Computational complexity grows with finer discretization and longer planning horizons, so careful selection of state and action resolutions is essential. Nevertheless, with sensible parameter settings, computation times remain well below the planning horizon, enabling real-time implementation. The following subsection~\ref{subsec:flow_dp_algorithms_limitations} surveys algorithmic approaches and their limitations.


\subsection{Algorithmic Approaches and Limitations}
\label{subsec:flow_dp_algorithms_limitations}

Throughput‐oriented \ac{dp} for \ac{glosa} has evolved considerably since the early 2000s. We classify seven main research trajectories --- ranging from exact, full‐horizon optimisers to lightweight, driver‐centric heuristics --- by their approach to the trade‐off between advisory optimality and real‐time feasibility. The methods are assessed in the subsequent section on the basis of the quantitative performance enhancements, the computational or practical expenses, and the primary constraints that have been noted.
\mynewline
We begin with the computational baseline, full-horizon grid-based \ac{sdp}, whose optimality is provable yet costly. \emph{grid-based backward \ac{dp}}, discretising both space (vehicle position relative to the stop line) and speed into a finite two-dimensional lattice, and performing backward value iteration on the Bellman equation under a terminal “green-phase” constraint. Typaldos and Papageorgiou \cite{Typaldos2021} formulate the stochastic \ac{glosa} problem as a discrete-time \ac{sdp} and report that, due to the exponential growth of the discretised state–control domain, the one-shot \ac{sdp} algorithm may require several minutes of CPU time on a standard PC, rendering on-board real-time execution infeasible. Coarsening the discretisation (i.e.\ increasing the grid step) reduces computation time but shrinks the feasible state domain, potentially excluding viable green-phase arrival states or degrading advisory quality.
\mynewline
To cut this cost without sacrificing optimality, Typaldos et al.~\cite{Typaldos2023} shrink the state space via a moving corridor. \textit{Receding‐horizon \ac{dp}} --- often realised in an \ac{mpc} framework --- has recently been accelerated through Discrete Differential Dynamic Programming (\ac{dddp}). The algorithm restricts every \ac{dp} iteration to a slender state–space corridor centred on the previously accepted trajectory and shifts this corridor forward as fresh signal-state information becomes available, thereby retaining optimality while severely curtailing the search volume. In the benchmark stochastic \ac{glosa} scenario, \ac{dddp} replicated the full-horizon \ac{sdp} optimum of $J^\star=1.17517$ yet slashed total computation time from $614\ \mathrm{s}$ to $0.69\ \mathrm{s}$ --- nearly three orders of magnitude faster. Parametric studies covering a broad range of corridor widths and discretisation steps reported per-iteration CPU times spanning $0.28\ \mathrm{s}$ to $2.30\ \mathrm{s}$ while consistently achieving the same optimal cost, making advisory refresh rates of \(\sim1\,\text{Hz}\) feasible on standard on-board hardware. The chief drawback of corridor truncation is its sensitivity to window size: if the green phase commences outside the predicted corridor, admissible arrival trajectories may be excluded, whereas overly wide corridors inflate runtime. Typaldos et al.\ therefore advocate adaptive corridor re-expansion and discretisation refinement to trade off convergence reliability against computational load. 
\mynewline
Instead of pruning the state space, Cai et al. \cite{Cai2008} approximate the value function itself. \textit{Adaptive traffic-signal control via Approximate Dynamic Programming (ADP)} embeds Bellman’s principle in a forward, rolling-horizon scheme whose value function is approximated by a linear feature map updated online with either temporal-difference reinforcement learning (TD) or numerical perturbation learning (PL). Every $0.5\;\text{s}$ (fine resolution) or $5\;\text{s}$ (coarse resolution) the controller receives 10\,s of detector data, predicts a further 10\,s, and evaluates three options --- hold, immediate stage switch, or deferred switch within the head window --- implementing only the first increment while propagating parameter updates. In a three-stage isolated intersection, the fine-resolution ADP matched the full-horizon \ac{dp} optimum of $4.27\;\text{veh·s s}^{-1}$ within $0.35\;\text{veh·s s}^{-1}$ (ADP\textsubscript{TD} $4.62$, ADP\textsubscript{PL} $4.66$) yet cut per-hour run-time from $720\;$min (DP) to $5.5\;$min and $12\;$min, respectively. Relative to optimised TRANSYT fixed-time plans ($13.95\;\text{veh·s s}^{-1}$) the ADP reduced delay by $\approx67\%$, and working at $0.5\;\text{s}$ rather than $5\;\text{s}$ yielded a further $41\%$ improvement (coarse ADP\textsubscript{PL} $8.64\;\text{veh·s s}^{-1}$). Under a four-hour, time-varying demand pprofile,both learning rules converged to statistically indistinguishable means (ADP\textsubscript{PL} $3.24$ vs. ADP\textsubscript{TD} $3.28$; $p=0.43$), with TD requiring less than half the CPU time because it computes a single Bellman update per step. Sensitivity tests showed the best performance at a high discount rate $h=0.12$, indicating that near-term costs dominate and justifying the simple linear approximation. Chief limitations are reliance on large discounting (dampening the value-function’s contribution), linear state representation, and isolation to a single junction; the authors recommend extending to network-level coordination, parallelising the per-state cost evaluation, deriving constant-time queue-update formulas, and exploring non-linear or piecewise-linear value functions for stochastic arrivals and adaptive phase ordering.
\mynewline
A complementary path compresses complexity analytically: Samra et al. \cite{Samra2015} reformulate the \ac{dp} state to obtain linear time. They propose a linear-time–and-space dynamic-programming algorithm for optimal traffic-signal duration at an isolated intersection that re-encodes the \ac{dp} state around the current time step and prunes non-promising predecessors early, collapsing the complexity of the classic \ac{cop} formulation from $O(T^{3})$ time and $O(T^{2})$ memory to $O(|P|T)$ for both metrics; a C implementation on a 2.10 GHz Intel Core2-Duo achieved $>2700\times$ speed-up over \ac{cop} for a 15-min horizon with $T\!=\!1024$ one-second steps, finishing in $0.27$ s while matching the global optimum.  The method generalises to heterogeneous minimum-green and clearance constraints by prefix-encoding constrained phases, keeping the state space linear in $T$.  In Green Light District simulations, the algorithm cut average junction waiting time by $\approx63\%$ (2.7 ×) under heavy demand (spawn rate 0.4) for a three-phase intersection and by $\approx66\%$ (3 ×) for a four-phase node relative to the strongest reinforcement-learning baseline, with consistently low run-to-run variance.  Limitations include reliance on accurate short-term arrival forecasts—prediction errors break optimality guarantees --- linear scaling only in horizon length but not yet across networks, and an $O(|P|)$ per-state cost evaluation that becomes salient for large phase sets. The authors therefore recommend (i) embedding the algorithm in coordinated frameworks such as RHODES, (ii) parallelising across cores to handle multi-intersection deployments, (iii) deriving constant-time queue updates to reduce per-state cost to $O(1)$, and (iv) extending the formulation to stochastic arrivals and adaptive phase ordering while preserving worst-case linear bounds.
\mynewline
When analytical compression is impossible, meta-heuristics such as genetic algorithms trade exactness for scalability. Researchers turned to \emph{metaheuristic graph‐search} methods to alleviate the combinatorial explosion of multi‐segment \ac{glosa}. Seredynski and Khadraoui \cite{Seredynski2013} model the advisory problem as a sequence of $n$ road segments and encode each candidate solution as a vector of segment‐wise speed advisories. They solve this via a generational \ac{ga} employing selection, single‐point crossover, and mutation operators, iterating until a stop condition is met. Simulation results in non‐congested networks with up to fifteen segments show that the multi‐segment \ac{ga} approach significantly outperforms single‐segment \ac{glosa} in both travel‐time and fuel‐efficiency metrics. Although \ac{ga} circumvents the exhaustive enumeration required by \ac{dp}, its runtime still grows exponentially with the number of segments, and scaling beyond moderate route lengths remains infeasible on standard hardware without additional heuristic pruning or parallelization. The authors suggest that incorporating admissible heuristic functions --- such as remaining distance divided by free‐flow speed --- could further prune the search space while guaranteeing constraint satisfaction. 
\mynewline
Recently, researchers have embraced integrated control, co-optimising signals and \ac{cav} trajectories. \textit{Joint optimisation of vehicle trajectories and intersection controllers} (DP–SH) by Guo et al.~\cite{Guo2019} couples a dynamic-programming signal optimiser with the Shooting Heuristic (SH) trajectory generator to co-design signal phases, green durations and \ac{cav} speed profiles in a single forward recursion; SH restricts each car’s search to at most seven analytically solvable quadratic arcs, so \ac{dp} treats phases as stages and embeds SH as a sub-routine, retrieving the optimal control whenever the horizon-wide value function improves by less than 5 \%.  With a 122 s planning horizon and 8 s step size, the algorithm computes a full four-phase plan in 9.8–15.2 s for up to twenty vehicles per approach on a 1.8 GHz laptop --- well below 10\% of real time --- and larger step sizes cut runtime to 2.8 s at only marginal performance loss.  Across twelve demand scenarios (segment lengths 400–1200 m, saturation ratios 0.6–1.5) DP–SH with subsequent SH parameter refinement trimmed mean travel time by 23.6–35.7\% and fuel by 11.8–31.5\% relative to adaptive signal control, while internal parameter tuning of acceleration bounds yielded a further 10–18\% fuel saving without sacrificing throughput.  In mixed traffic, the framework remains robust: even at 10\% \ac{cav} penetration, average delay and consumption still drop 3.1\% and 6.2\%, rising monotonically to 34.8\% and 31.5\% as penetration approaches 100\%. Sensitivity tests show computational load grows linearly with horizon length but only weakly with vehicles per phase. However, horizons below \(\approx\) 400 m may preclude feasible SH trajectories under heavy queues, requiring either minimum-length constraints or upstream spill back prediction. Principal challenges are (i) non-convex parameter tuning that can stall in local minima, (ii) reliance on accurate entry-boundary forecasts, (iii) focus on isolated intersections—spill back chains and multi-signal coordination remain open—and (iv) real-time scalability when horizon or demand scales; the authors therefore advocate parallelisation, distributed feedback control, extension to corridor-level or signal-free \ac{cav} reservation policies, and incorporation of stochastic capacity for truly network-wide deployment.
\mynewline
At the opposite end of the spectrum, DC-GLOSA achieves field-ready simplicity by relegating computation to a smartphone. \textit{Driver-centric Green Light Optimal Speed Advisory (DC-GLOSA)} proposed by Suramardhana and Jeong~\cite{Suramardhana2014} equips each vehicle with a smartphone-based on-board unit that (i) receives \ac{spat} messages from the roadside unit, (ii) estimates ego-motion via GPS, and (iii) issues one of seven instantaneous acceleration/braking advisories chosen by comparing the current time-to-cross (TTC) against six analytically defined reference mobility models—maximum acceleration, constant acceleration, constant speed, constant braking, maximum braking, and emergency deceleration to stop—thereby respecting speed bounds $v_{\min}=20\,$km h$^{-1}$ and $v_{\max}=40\,$km h$^{-1}$ while limiting longitudinal jerk to $a\in[\,{-}3.82,\,3.82\,]\,$m s$^{-2}$. The algorithm incrementally updates the advisory whenever a new \ac{spat} frame or significant state change arrives, deferring freshly triggered voice prompts until the previous one has finished, avoiding cognitive overload and queuing at most the most recent message. Field trials with a single vehicle at a three-arm signalised junction on the Pusan National University campus ($T_{\text{green}}=30$ s, $T_{\text{amber}}=3$ s, $T_{\text{red}}=100$ s) showed that DC-GLOSA reduced average intersection-stopping time by $23.9\%$ over a baseline with no advisory across 54 runs spanning approach distances $D=50\text{–}200$ m and residual-green intervals $T_{\text{ref}}=10\text{–}70$ s. While the strategy improves throughput and driver comfort, two practical issues persist: (a) advisory latency from smartphone audio playback ($\approx 2$–$3$\,s) can render a message stale in rapidly changing scenarios, and (b) the scheme relies on accurate \ac{spat} and driver compliance; failure to receive timely updates or aggressive driving can force emergency braking when $D_{\text{stop}}>D$. The authors therefore recommend future work on adaptive message suppression, multi-vehicle coordination to resolve conflicting advisories, and integration of power-train awareness plus eco-cost functions to extend the current TTC and emission-minimisation.
\mynewline
Viewed along the complexity axis we have now traversed, three patterns emerge. Exact methods (grid-based SDP and DDDP) guarantee optimality but strain real-time hardware; approximate or analytical reductions (ADP and linear-time DP) temper this cost while preserving most benefits; heuristic and driver-centric schemes (GA, DP–SH, DC-GLOSA) prioritise deployment ease and system-wide coordination, albeit at the risk of sub-optimality.
\mynewline
\textbf{Key quantitative insights across methods:}
\begin{itemize}
  \item \emph{Delay or stopping‐time reduction:} 40\,\% in grid-based SDP, 39\,\% in DDDP corridor studies, 23.9\,\% in the DC-GLOSA field trial, and 30–36\,\% in DP–SH mixed-traffic scenarios.
  \item \emph{Computation time per advisory:} 5–10\,min for full-horizon SDP, 0.28–2.30\,s for DDDP, 5.5–12\,min per simulated hour with ADP, 0.27\,s for linear-time DP, 2.8–15.2\,s for DP–SH, and sub-second for DC-GLOSA smartphone prompts.
  \item \emph{Scalability constraints:} exponential state-space growth with SDP, corridor-width sensitivity in DDDP, feature-map bias for ADP, per-state $O(|P|)$ cost for each phase in linear-time DP, combinatorial explosion as segment count increases in GA, demand-forecast reliance in DP–SH, and audio-latency plus compliance issues in DC-GLOSA.
\end{itemize}

\textbf{Principal limitations and open challenges.} First, \emph{computational tractability} remains the dominant hurdle: exact full-horizon \ac{sdp} and its corridor-pruned variant (DDDP) guarantee optimality but incur exponential or corridor-width-sensitive costs that exceed typical on-board budgets. Second, \emph{approximation bias} affects receding-horizon and feature-based ADP schemes --- incorrect corridor sizes may exclude feasible trajectories, while linear value‐function approximations and high discounting can distort long-term performance. Third, \emph{per-state cost scaling} persists: Samra et al.’s linear-time DP still requires $O(|P|)$ evaluations under large phase sets, and genetic algorithms face combinatorial explosion as segment count grows. Fourth, \emph{forecast and communication dependences} challenge DP–SH and DC-GLOSA: the former demands accurate boundary predictions and low-latency V2X links, the latter suffers 2–3s audio latency and driver compliance issues. Fifth, \emph{constraint enforcement} in approximate or continuous DP often relies on heuristic corrections for hard green-phase satisfaction, undermining formal guarantees. Sixth, \emph{secondary objectives} --- ride comfort, fuel economy and emissions --- remain largely orthogonal, multiplying the action space and complicating convergence. Finally, \emph{network scalability} is unresolved: nearly all methods target isolated intersections or single vehicles, and extending them to stochastic, multi-junction corridors with spill back and dynamic coordination stands as the grand challenge.
\mynewline
In conclusion, throughput-oriented \ac{dp} for GLOSA has delivered prototype-scale improvements --- up to 40\,\% delay reduction in grid-based \ac{sdp}, 39\,\% with receding-horizon DDDP, and 30–36\,\% in integrated co-optimisation --- while driver-centric and continuous methods produce sub-second advisories with moderate gains. Real-time deployment will require algorithms that reconcile computational efficiency, rigorous constraint satisfaction, and true multi-objective optimality. Future work must unite uncertainty-aware control, low-latency V2X communication, and network-level coordination to translate current prototypes into robust, large-scale field systems.
