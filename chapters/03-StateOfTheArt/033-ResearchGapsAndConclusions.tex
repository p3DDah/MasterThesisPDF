\section{Literature Synthesis and Research Gap}
\label{sec:research_gap}

The aforementioned literature review of \ac{glosa} uncovers a consensus that these systems can decrease stop-and-go behaviour and, under specific circumstances, reduce energy consumption. However, the reported findings are highly varied and context-specific, which points to several underlying research gaps.

\subsection{Synthesis of Existing Work}
Most studies concur that \ac{glosa} systems reduce the frequency of stops. Under light traffic conditions, this generally leads to lower energy use. The reported average fuel savings span a wide range, from $11$--$35\%$ for flow-focused controllers to $25$--$68\%$ \cite{Guo2019,Typaldos2023,Cai2008} for eco-centric variants \cite{Park2024,Pulvirenti2023,Dong2024}. This significant spread can be attributed to three main factors. First, the optimisation objectives differ, with some studies prioritising delay reduction while others focus on fuel minimisation. Second, the car-following models used for simulation diverge significantly. Most studies rely on default \ac{idm} parameters, which tend to underestimate real-world acceleration and deceleration behaviours in dense traffic. In contrast, calibrated \ac{eidm} parameters more accurately reflect realistic driving dynamics, which can lead to different performance outcomes. Third, the \ac{mpr} and demand levels vary considerably across studies. Many simulations are conducted with relatively low traffic volumes of around $500\unit{\veh\per\hour}$ \cite{Yang2017, Ala2016}, which does not adequately represent conditions at heavily utilised urban intersections, where densities can frequently exceed $1000\unit{\veh\per\hour}$. These observations indicate that while current evidence is promising, it remains fragmented and context-specific, which highlights systematic biases that give rise to the following research gaps.
\mynewline
The following research gaps are identified as a result of the systematic biases that are revealed by the consolidated picture.

\subsection{Identified Research Gaps}
\begin{enumerate}
    \item \textbf{Penetration–Density Interaction.} Only a handful of studies have systematically varied both the \ac{mpr} and traffic density on a calibrated urban node. Most have kept demand at approximately $500\unit{\veh\per\hour}$, thereby under-representing peak-period conditions.
    \item \textbf{Fuel-Model Sensitivity.} No work has compared emission maps, such as HBEFA4 and the PHEMlight5 model, under identical scenarios. This impedes cross-study coherence and makes it difficult to generalise findings.
    \item \textbf{Driver-Heterogeneity Representation.} Evaluations have almost always applied the default \ac{idm}. Calibrated \ac{eidm} parameters for stop-and-go traffic remain largely unexplored, despite evidence that the \ac{idm} underestimates queue discharge times at high demand.
    \item \textbf{Eco–Flow Trade-Off at a Network Scale.} The systemic impact of eco-driving vehicles operating in an egoistic manner on overall throughput is poorly quantified. Existing results are limited to isolated junctions.
    \item \textbf{Benchmark Realism.} No standard intersection or openly shared \ac{sumo} scenario exists. Most studies simulate stylised layouts, which hinders replication. Field data are scarce outside of small pilot trials.
\end{enumerate}
Filling these gaps encapsulates this thesis's contribution.

\subsubsection{Thesis Contribution Anchor}
\label{subsubsec:thesis_contribution_anchor}

This thesis advances the field in three primary ways. First, it replicates the flow-oriented results of Lenz \cite{Lenz2024} by designing an \ac{eco-glosa} controller for the Stuttgart-Neckartor intersection. This is tested under identical conditions with traffic volumes up to $3500\unit{\veh\per\hour}$, using a controller based on \ac{dp} and a calibrated \ac{eidm}. Second, the research maps the penetration-density surface, which allows for the identification of thresholds where eco-guidance benefits or harms fuel consumption and traffic flow. This analysis also quantifies the collateral effects on non-equipped vehicles across all \ac{mpr} bands. Finally, the work provides an open-source codebase and a fully reproducible \ac{sumo} package. This is intended to serve as a standard benchmark for future studies in \ac{dp} and even \ac{rl}.
\mynewline
The next chapter outlines the methodological framework that operationalises these goals.