\section{Dynamic Programming for Energy-Aware GLOSA}
\label{sec:eco_glosa}

\ac{glosa} algorithms advise road users on speed adjustments that maximise the probability of crossing signalised intersections during green phases.  When the objective is to minimise energy use rather than merely reduce delay, \ac{dp} offers a mathematically rigorous means of embedding high-fidelity power-train models, signal timing constraints and safety limits in a single optimal-control framework. \ac{dp}’s appeal lies in its ability to guarantee local optimality over a discretised state–action space, to accommodate non-convex efficiency maps and integer variables (e.g.\ gear or hybrid mode selection), and to incorporate hard constraints such as arrival windows or legally mandated speed bounds.
\mynewline
Each energy-conscious \ac{glosa} \ac{dp} needs to respond to three modelling questions. \textbf{(i)~Objective formulation:} Does the cost functional target sole fuel minimisation, or does it balance energy with delay, comfort, or even pollutant exposure? \textbf{(ii)~Power-train representation:} Should the stage cost rely on tractive-energy proxies (mass--speed--acceleration products) or on calibrated maps such as PHEMlight5, HBEFA4 or VT-CPFM that convert wheel demands into fuel mass and CO$_2$ in grams? \textbf{(iii)~Constraint encoding:} How are vehicle dynamics (speed, longitudinal acceleration, jerk), legal limits and power-train ceilings (engine torque, battery state-of-charge, electric motor power) enforced without exploding the dimension of the state space? The following subsection synthesises how recent studies have tackled these three questions, highlighting common design patterns, trade-offs and emerging trends toward multi-criteria optimisation.

\subsection{Energy and Multi-Criteria Objective Formulations}
\label{subsec:eco_dp_objectives}

An \ac{eco-glosa} controller seeks a longitudinal control signal $u(t)$—usually a bounded acceleration or wheel torque—such that the resulting vehicle trajectory minimises an energy-centric performance index while guaranteeing legal and safety compliance at signalised intersections. Let the system state be $\mathbf x(t) = [s(t),\,v(t),\,\mathrm{SoC}(t)]^{\!\top}$, comprising travelled distance $s$, speed $v$, and, for electrified power-trains, battery state of charge. Continuous-time dynamics can be abstracted as
\begin{equation*}
    \dot{\mathbf x}(t) = \mathbf f\bigl(\mathbf x(t), u(t)\bigr),
    \qquad
    u_{\min} \le u(t) \le u_{\max},
\end{equation*}
where $\mathbf f$ captures longitudinal equations of motion, rolling and aerodynamic resistances, grade, and simple battery equivalence for hybrids. Discretising the formula on a uniform temporal grid $t_k = k \Delta t$ yields the canonical \ac{dp} recursion
\begin{equation*}
    V_k(\mathbf x_k) = \min_{u_k \in \mathcal U} \Bigl\{\, g_k(\mathbf x_k, u_k) + V_{k+1}\bigl(\mathbf x_{k+1} \bigr) \Bigr\},
    \qquad
    \mathbf x_{k+1} = F_k(\mathbf x_k, u_k),
\end{equation*}
with terminal cost $V_{N}(\mathbf x_N)=\Phi(\mathbf x_N)$ encoding mandatory arrival time $t_f$, stop-bar position $s_f$ and permissible speed range $v\in[v_{\min},v_{\max}]$ at the intersection.

\textbf{Single-criterion energy minimisation.}
The most direct objective is the integral of instantaneous fuel or electrical power $P_{\text{tr}}$:
\begin{equation*}
\label{eq:single}
J_{\text{fuel}} = \int_{0}^{t_f} P_{\text{tr}}\bigl(\mathbf x(t), u(t)\bigr)\,\mathrm{d}t
\quad\Longrightarrow\quad
g_k = \Delta t \cdot P_{\text{tr}}\bigl(\mathbf x_k, u_k\bigr).
\end{equation*}
For conventional engines, $P_{\text{tr}}$ is mapped to mass flow $\dot m_{\text{fuel}}$ via calibrated look-up tables such as \ac{phem}light5 or \ac{hbefa}4; for battery-electric vehicles it equals traction power minus regenerative recovery, and for hybrids it combines engine fuel and battery energy through an equivalence factor $\xi$ so that $g_k = \Delta t \cdot \left[ \dot m_{\text{fuel}} + \xi P_{\text{bat}} \right]$.

\textbf{Weighted-sum multi-objective costs.}
Minimising~\eqref{eq:single} alone can drive the optimum toward unrealistically low speeds. A pragmatic remedy is to augment the single-criterion formula with travel time $T = t_f - t_0$ and comfort penalties on jerk $j = \dot a$:
\begin{equation*}
    J_{\mathrm{sum}} = \alpha E + \beta T + \gamma \int_{0}^{t_f} j(t)^2\,\mathrm{d}t,
    \qquad
    E = \int_{0}^{t_f} P_{\text{tr}}\,\mathrm{d}t,
\label{eq:weighted}
\end{equation*}
where $\alpha$, $\beta$, and $\gamma$ tune the policy along the energy–delay–comfort trade-off surface. In \ac{dp} form, the stage cost becomes
$g_k = \alpha \Delta t\,P_{\text{tr}} + \beta \Delta t + \gamma \Delta t\,j_k^2$.

\textbf{Pareto formulations.}
To avoid arbitrary weights, an \emph{$\epsilon$-constraint} variant minimises one metric while bounding the others:
\begin{equation*}
    \min\;E \;\;\text{s.\,t.}\;\;T \le \bar T, \qquad
    g_k = \Delta t\,P_{\text{tr}}, \qquad
    \Phi = \mathbbm{1}_{T \le \bar T},
\end{equation*}
generating Pareto-optimal fronts by sweeping the tolerance $\bar T$. A higher-level supervisory module can then select the point that best fits network objectives or driver preference.

\textbf{Constraint set.}
All formulations inherit hard bounds from road regulations and power-train physics:
\begin{align*}
0 &\le v_k\le v_{\mathrm{lim}}(s_k), &
|a_k| &\le a_{\max}, &
|j_k| &\le j_{\max}, \nonumber\\
P_{\text{eng},k} &\le P_{\text{eng}}^{\max}, &
\mathrm{SoC}_{\min} &\le \mathrm{SoC}_k\le \mathrm{SoC}_{\max}, &
t_{\mathrm{g,\,on}}\le t_k &\le t_{\mathrm{g,\,off}},
\end{align*}
where the final line enforces the \ac{spat}-derived arrival window. Constraints are handled either by pruning infeasible lattice nodes before the Bellman sweep or by adding large penalties to $g_k$.

\textbf{Lattice design and complexity control.}
Let $n_s$, $n_v$ and $n_{\mathrm{SoC}}$ be the discretisation levels of distance, speed, and battery charge. The complexity of the Bellman formula scales as $\mathcal O(N\,n_s n_v n_{\mathrm{SoC}} N_u)$, with $N_u$ the number of admissible controls. To render on-board computation feasible, three reduction strategies dominate:  
(1)~\textit{Variable grids} contract the lattice where the value function is nearly affine,  
(2)~\textit{Multi-stage} splitting solves the Bellman formula separately across signal phases or queue discharge events and stitches the partial value functions at boundaries, and  
(3)~\textit{Surrogate models} replace costly $P_{\text{tr}}$ evaluations with neural or polynomial emulators trained offline.

The above framework, which includes a rigorous Bellman recursion, a cost functional that can be single or multi-objective, and systematic constraint handling, forms the basis for energy-aware \ac{glosa} speed guidance. Subsequent subsections build on this scaffold to incorporate stochastic queues, vehicle–to–everything connectivity and cooperative manoeuvres.

\subsection{Queue- and Connectivity-Aware Energy-Optimal GLOSA Methods}
\label{subsec:eco_dp_queue}

This section talks about eight dynamic programming-based eco-driving strategies that make sure they use the least amount of fuel (or electricity) while still respecting signal timing and queue dynamics. We begin with single-vehicle, offline \ac{dp} formulations for battery and hybrid power-trains (Park et al. \cite{Park2024}, Pulvirenti et al. \cite{Pulvirenti2023}), proceed through real-time multi-stage and queue-aware controllers (Kamalanathsharma et al. \cite{Kamalanathsharma2013}, He et al. \cite{He2015}), and conclude with cooperative, \ac{v2i}/\ac{v2v}-enabled schemes (Yang \cite{Yang2017}, Ala \cite{Ala2016}, Yang \cite{Yang2021}, Dong \cite{Dong2024}). For each method, we summarise the \ac{dp} formulation, report precise savings and runtimes, and highlight key limitations and suggested remedies.
\mynewline
To begin with, Park et al. \cite{Park2024} tackle the offline BEV case: they formulate a battery-electric eco-driving problem for a Hyundai Ioniq 5 as a two-state (distance, speed) discrete-time \ac{dp} model that minimises cumulative electrical power by exhaustively enumerating acceleration commands on a 1 s grid while enforcing fixed departure/arrival times, distance and speed limits, and drive-train comfort bounds; the solver explores a 150 s horizon with 0.05 m space and 0.1 m s$^{-1}$ speed steps, incurring 337 s of offline computation. Comparing the \ac{dp} trajectories with an \ac{idm} baseline shows sizeable battery savings: over a 500 m sprint, “Comfort-DP’’ (human-like acceleration) cuts energy by 34.3\% and “Aggressive-DP’’ (full torque) by 67.5\%; over a 900 m run the reductions are 25.5\% and 47.8\%, respectively.  Analysis attributes gains to (i) strict use of high-efficiency tangential and stationary points on the motor Pareto frontier during acceleration and cruising/coasting, and (ii) full recuperation during terminal braking. Limitations include reliance on perfect prior knowledge of boundary conditions and zero road grade, absence of battery power limits at low SoC, and infeasible on-board runtimes without GPU or surrogate policies; the authors therefore advocate state-reduction heuristics, stochastic SPaT and traffic integration, and real-vehicle validation to bridge the gap between offline optimality and real-time deployment.
\mynewline
While Park et al. focus on exhaustive grid search for \acp{bev} under fixed boundary conditions, Pulvirenti et al. \cite{Pulvirenti2023} extend the \ac{dp} paradigm to plug-in hybrids by introducing a variable-grid DP (VGDP) framework that exploits \ac{v2x} data and cloud computation to generate energy-optimal speed trajectories for a plug-in hybrid Mercedes E300de along a 96 km RDE-compliant route. The spatial-domain \ac{dp} state is either $(v)$ for stop-controlled Scenario 1 or $(v,t)$ for traffic-light Scenario 2, while the control is longitudinal acceleration, and the bi-objective stage cost combines incremental traction energy with travel time via a weight $\beta$. Rather than searching a fixed lattice, VGDP shrinks the speed grid to the interval defined by a 500 m moving-average envelope ($\pm20$ km h$^{-1}$) and bounds the time grid around an ETA corridor derived from communicated \ac{spat}, speed limits and grades; with a 5 m spatial step this cuts the node count by two orders of magnitude and slashes solver wall-time on an Intel E5-2680 workstation from 24 min to under 1 min (-95 \%), without degrading optimality. In Scenario 1, stop signs imposed at every observed standstill, the optimiser delivers smoother acceleration profiles that lower fuel-equivalent energy by 29 \% and shrink cruise time by 10 \% while still respecting all full-stop constraints. In Scenario 2 all stops are replaced by 60 s/60 \%-green traffic lights whose \ac{spat} is assumed deterministic and known: VGDP schedules an almost uniform 49 km h$^{-1}$ cruise, eliminates complete halts, and achieves 54 \% energy and 38 \% travel-time savings relative to the reference drive, with sensitivity analysis showing only modest Pareto trade-offs as $\beta$ varies from 0.2 to 0.8. Limitations include the reliance on perfect SPaT and queue-free flow, single-vehicle scope, and the need for high-resolution distance discretisation (5 m) that still burdens memory when scaled to network traffic; future work is directed at platoon-level coordination, stochastic queue integration and on-road hardware validation of the cloud-assisted control loop.
\mynewline
Moving from cloud-assisted, offline trajectory planning to real-time control, Kamalanathsharma et al. \cite{Kamalanathsharma2013} propose a \textit{multi-stage dynamic-programming (MS-DP) eco-speed controller} that frames the approach to a red phase as a least-cost path-finding problem on a two-stage lattice: the upstream stage chooses a constant deceleration $d$ and cruise time $t_{\mathrm c}$ that bring the vehicle to the stop-bar exactly at green onset, while the downstream stage selects a throttle trajectory to re-accelerate to the desired speed; both stages minimise cumulative fuel predicted by the VT-CPFM model subject to comfort bounds ($d\le3\,\mathrm{m\,s^{-2}}$, $a^{+}\le1.1\,\mathrm{m\,s^{-2}}$), grade, weather, and microscopic resistance forces. A heuristic $\mathcal A$-star recursion evaluates each 0.1 s step, pruning dominated nodes so that real-time feasibility is maintained despite the use of full powertrain and aerodynamics equations. Agent-based MATLAB simulations over a $200$ m upstream and $400$ m downstream zone, with random arrivals across an 84 s cycle and eighteen combinations of speed limit (25/35/45 mph, dry–wet–snow pavement, and $\pm5\,\%$ grades, report mean savings of $19.5\,\%$ fuel and $32\,\%$ travel time relative to an ITE-calibrated human baseline, with the best cases reaching $37.2\,\%$ and $41.5\,\%$ under snowy 25 mph (ca. 40 km/h) conditions.  Car-following tests coupling the Rakha–Pasumarthy–Adjerid model show that a non-instrumented follower inherits \(\approx15\,\%\) fuel reduction when tracking an MS-DP lead vehicle, indicating spill-over benefits. Limitations are reliance on perfect \ac{spat}, single-vehicle optimisation (queue spill back is ignored), sensitivity to lane-changing below a 30\,\% \ac{cv} penetration, and validation confined to simulation; oversaturated demand ($\ge800$ veh h$^{-1}$\,lane$^{-1}$) breaks the kinematic-wave predictor and can render benefits negative. The authors therefore call for \ac{v2v}-aided queue sensing, speed-harmonisation overlays, multi-intersection coordination, robustness to packet loss and driver non-compliance, and field trials to confirm the observed \(\sim20\,\%\) fuel-saving potential.
\mynewline
In a similar real-time vein, He et al. \cite{He2015} formulate a queue-aware optimal speed trajectory (QOST) that embeds stochastic queue-tail constraints and non-blocking requirements into a multi-stage optimal-control problem, enabling sub-second computation. In this manner, a single car never impedes traffic moving upstream and doesn't cause the green light to go out for other cars. A five-variable quadratic surrogate (terminal time, two accelerations, two switch instants) solves each stage in $\approx1.5\,$s on off-the-shelf hardware, making real-time deployment plausible. In the two-intersection benchmark, QOST eliminates the five-second idle at the first queue but because it cruises at an eco-speed of $30\,$mph, the run time grows from $77.3$s (no advice) to $90.0$s—an increase of $16\%$—while stops fall to zero and fuel use drops $41\%$. A six-signal field trial on Minneapolis TH-55 repeats the pattern: the advised vehicle traverses the corridor without stopping, yet travel time rises from $236$ to $258$s (+9\%) as queues are absorbed upstream, delivering a $29\%$ fuel saving and a net fuel-economy jump from $17$ to $24$mpg. By design, the model also compels the subject vehicle to clear the stop-line before a threshold $T_{\tilde B}$ so as not to lengthen the residual queue or trap following traffic in the next red, thereby avoiding secondary delay propagation. The authors acknowledge that the traffic-flow benefits are therefore mixed: idling, shock waves and spillback are curtailed, but mean travel time for the advised vehicle—and hence point throughput—can worsen unless the eco-speed is lifted or applied jointly to a platoon. Key challenges are (i) reliance on high-fidelity, real-time queue estimation—biases here jeopardise both feasibility and flow gains; (ii) piecewise-constant accelerations that introduce jerk peaks; and (iii) the single-vehicle scope, which cannot exploit cooperative gaps or optimise multi-vehicle order. Proposed remedies include embedding stochastic queue predictors, smoothing trajectories with power-train-aware penalties, extending the formulation to simultaneous optimisation of several front-of-queue vehicles, and rolling re-optimisation when signals skip phases—all aimed at turning the current fuel-centric design into one that also boosts corridor-level throughput under varying demand patterns.
\mynewline
Building on single-vehicle queue awareness, Yang et al. \cite{Yang2017} introduce a \textit{queue-aware Eco-Cooperative Adaptive Cruise Control} (Eco-CACC-Q) method, a cooperative \ac{v2i}-enabled adaptive cruise control that uses \ac{spat} packets and kinematic-wave queue estimates to optimize fuel consumption across a platoon. Using this information, they calculate advisory speed limits that are updated every second to minimize the instantaneous VT-CPFM fuel rate and ensure that the probe vehicle reaches the stop-bar precisely when the final vehicle in the queue is released. A receding-horizon quadratic program simultaneously selects an upstream constant deceleration $a^{-}$, a cruise speed $v_{\mathrm c}$ over the remaining approach, and a downstream acceleration $a^{+}$ on a 500 m/200 m control segment; the search is bounded by comfort limits $0\le a^{-}\!\le\!3\,$m s$^{-2}$, $0\le a^{+}\!\le\!2.5\,$m s$^{-2}$ and SoC-neutrality constraints. Single-lane integration simulations with a 20\% market-penetration rate (MPR) and a uniform 500 veh h$^{-1}$ inflow show Eco-CACC-Q trims average fuel use by 11.4\% relative to the base car-following model and by 4.5\% compared with the non-queue Eco-CACC variant, eliminating complete stops and cutting speed variance from 29.7 to 15.3 km h$^{-1}$.  Savings scale almost linearly with penetration, reaching 18.0\% at 100\% MPR. On two-lane approaches, the algorithm remains beneficial only above a 30\% MPR; below that threshold, aggressive lane changes around slower controlled vehicles inflate acceleration spikes and raise network fuel use by \(\approx\) 5\%.  At full penetration, the multi-lane test bed records 18.3\% overall savings (19.2\% for the CACC fleet itself). Key limitations include dependence on accurate queue-length prediction—errors from lane-changing or oversaturation negate benefits—audio or control-actuation latency, and the assumption of isolated intersections; at very high demand the method fails once the LWR-predicted queue exceeds the control horizon. The authors therefore call for \ac{v2v}-aided queue sensing, corridor-level coordination with speed-harmonisation controllers, robustness analyses against packet loss and driver non-compliance, and real-vehicle validation to confirm the simulation gains.
\mynewline
Complementing the algorithmic development, Ala et al. \cite{Ala2016} perform an extensive INTEGRATION-based sensitivity study of queue-aware Eco-Cooperative Adaptive Cruise Control (Eco-CACC-Q), quantifying its performance over varying penetration rates, green splits, and demand levels. They evaluate the Eco-CACC-Q algorithm. Their contribution is a broad INTEGRATION-based sensitivity study rather than a new controller. Eco-CACC-Q exploits \ac{v2i} \ac{spat} packets and kinematic-wave queue estimates to compute, each second, an advisory trajectory composed of a bounded upstream deceleration $a^{-}\!\in[0,3]\,\mathrm{m\,s^{-2}}$, a cruise speed $v_{\mathrm c}$ timed to reach the stop-bar just as the queue clears, and a downstream re-acceleration $a^{+}\!\le 2.5\,\mathrm{m\,s^{-2}}$.  Single-lane tests with a fixed 500 m upstream / 200 m downstream control horizon and 300veh h$^{-1}$ demand show fuel-consumption savings that scale almost linearly with market-penetration rate (\ac{mpr}), peaking at 19\,\% when MPR = 100\,\%.  On two-lane approaches, the algorithm yields negative benefits below 30\,\% MPR because lane-changing around slower controlled vehicles injects oscillations; above that threshold savings recover and reach 19\,\% at full penetration. Varying the green split (0.3–0.7 of an 84 s cycle) changes baseline fuel use but alters Eco-CACC-Q savings by at most 2\,\%.  Increasing the upstream control length from 200 to 700 m boosts single-lane gains from 12\,\% to 18\,\%, with marginal returns beyond 500 m; the same pattern holds for two-lane links once MPR > 30\,\%. Demand sensitivity reveals an optimal 500 veh h$^{-1}$ (single lane) and 600 veh h$^{-1}$ (two lanes), while oversaturated flow (800 veh h$^{-1}$) breaks the kinematic-wave queue predictor, causes rolling queues and stop-and-go waves, and turns benefits negative. A four-leg junction case with asymmetric volumes (1000 veh h$^{-1}$ through on the major road) confirms the need for penetration: below 25\,\% MPR fuel use rises, but at full penetration network consumption drops by up to 25\,\%. Limitations highlighted include (i) dependence on accurate real-time queue length—errors or rolling queues undermine the trajectory, (ii) negative effects at low MPR on multilane arterials due to lane-changing, (iii) algorithm breakdown under oversaturation, (iv) assumption of isolated intersections and deterministic \ac{spat}, and (v) absence of field validation; future work therefore targets \ac{v2v} --- aided queue sensing, speed-harmonisation overlays to throttle entry flow, multi-intersection coordination, and robustness analyses against packet loss and driver non-compliance.
\mynewline
Extending to multi-signal corridors, Yang et al.\ \cite{Yang2021} propose Eco-MS-Q, a modular, queue-aware multi-signal controller that broadcasts second-by-second advisory speed limits to \acp{cv}. Leveraging \ac{v2i} \ac{spat} packets and kinematic-wave queue estimates, the algorithm solves a three-segment optimal-control problem for each approaching vehicle: (i) decelerate at bounded rate $a^{-}\!\in[0,3]\,\mathrm{m\,s^{-2}}$ to a cruise speed $v_{\mathrm c,1}$ that ensures the stop-bar is reached exactly as the last queued vehicle departs; (ii) re-accelerate or further decelerate to a second cruise $v_{\mathrm c,2}$ that aligns with the downstream signal; (iii) after clearing the final queue, accelerate at $a^{+}\!\le 2.5\,\mathrm{m\,s^{-2}}$ back to the link speed limit while maintaining \mbox{SoC} neutrality for the 48V mild-hybrid power-train. Implementation in the INTEGRATION micro-simulator shows that, on a two-signal arterial with $600\,\mathrm{veh\,h^{-1}}$ demand, fuel use falls by $7.0\,\%$ at $100\,\%$ \ac{cv} penetration (MPR) versus $4.2\,\%$ for single-signal control, and per-vehicle speed variance drops by $30\,\%$.  Extending to a four-signal corridor (600m spacing) yields savings of $7.7\,\%$ for single-lane links and $4.8\,\%$ for two-lane links, while a 16-intersection grid posts up to $15.0\,\%$ system-wide reduction, with benefits persisting for all MPRs because side-street and left-turn movements are single-lane constrained.  Sensitivity analysis identifies (a) an optimum demand of $400\,\mathrm{veh\,h^{-1}\,lane^{-1}}$ (saving $13.5\,\%$), (b) short green splits ($35\,\%$ of a 120s cycle) that raise savings to $13.8\,\%$, (c) suboptimal offsets (e.g.\ $100$s versus the $45$s green-wave optimum) that boost savings to $13.0\,\%$, and (d) $700$m signal spacing giving peak $13.1\,\%$ improvement. Conversely, offsets near the green-wave optimum cut gains to $2.8\,\%$.  Below $30\,\%$ MPR on multi-lane arterials, lane-changing around slower \acp{cv} can negate benefits, turning savings negative until cooperative density exceeds that threshold. Under over-saturated demand ($1000\,\mathrm{veh\,h^{-1}}$) rolling queues breach the LWR predictor, slashing savings to $2.7\,\%$ even at $10\,\%$ MPR.  Limitations therefore include: dependence on accurate real-time queue length and dissipation forecasts (errors propagate into suboptimal $v_{\mathrm c}$ choices); sensitivity to lane-changing when MPR is low; diminished efficacy near optimal offsets or long greens; and lack of spill back handling in over-saturation. The authors call for \ac{v2v}-aided queue sensing, stochastic queue models, corridor-level coordination, and adaptive speed-harmonisation overlays to maintain benefits under high load and mixed traffic.
\mynewline
Finally, Dong et al.\ \cite{Dong2024} integrate overtaking manoeuvres into eco-approach control by coupling lane selection via \ac{dp} with a Pontryagin-based speed optimizer, achieving combined lane-planning and energy savings. Using a two-stage receding-horizon framework, the \textit{overtaking-enabled eco-approach control} (OEAC) method yields closed-form control torques while respecting safety headways, comfort bounds ($a^{-}\!\in[0,3]\,\mathrm{m\,s^{-2}}$, $a^{+}\!\le2.5\,\mathrm{m\,s^{-2}}$), and deterministic \ac{spat} constraints. Stage 1 casts the lane-selection task as a finite-horizon Markov decision process solved by dynamic programming, explicitly modelling disturbance from surrounding vehicles.(( Extensive Monte-Carlo simulation over \(10\,000\) randomised urban scenarios shows OEAC cuts the composite “monetised” driving cost by an \emph{average} 20.91\,\% versus a constant-speed (CS) car-following baseline and by 5.62\,\% versus a regular eco-approach (READ) controller, with maxima of 59.97\,\% and 59.02\,\%, respectively. In a representative moderate-flow case (vehicle density \(=120\,\mathrm{veh\,km^{-1}}\)) OEAC executes a single lane change, enabling passage in the first green window and cutting travel time 55.93\,\%, traction energy 36.43\,\% and total cost 54.12\,\% relative to CS, and 55.89/31.46/53.01\,\% against READ. Computational overhead is negligible: mean step time 0.2 ms (worst 9.3 ms) on an i9-12900K PC, well below the 10 ms simulation tick. \emph{Limitations} include dependence on accurate, real-time queue-length and \ac{spat} data—errors or oversaturation negate benefits—reversion to pure car-following when safe overtakes are unavailable, negative fuel savings below 30\,\% \ac{cav} penetration on multilane links due to cut-ins, and validation restricted to simulation. 
\mynewline
The surveyed dynamic-programming-based \ac{glosa} methods span from offline, single-vehicle formulations for BEVs and hybrids to real-time controllers and cooperative \ac{v2x}-enabled schemes. Offline \ac{dp} approaches (Park et al.; Pulvirenti et al.) achieve substantial energy savings --- up to 67.5\% in BEVs and 54\% for hybrids --- at the cost of high computation time or reliance on cloud resources. Real-time MS-DP and QOST frameworks (Kamalanathsharma et al.; He et al.) reduce run-time to sub-second levels while maintaining 20–40\% fuel or energy reductions by embedding queue and non-blocking constraints. Cooperative eco-cruise controllers (Yang 2017; Ala 2016; Yang 2021) leverage \ac{spat} data and kinematic-wave models to achieve fleet-level savings of 7–19\% depending on market penetration, intersection spacing, and signal timing. The OEAC framework (Dong et al.) further incorporates lane-change decisions to realize up to 21\% composite cost reductions.
Despite these advances, common limitations persist: perfect \ac{spat} and queue knowledge assumptions; single-vehicle or platoon-only scopes; sensitivity to low penetration rates and oversaturated demand; and the predominance of simulation-based validation. Future research should target stochastic \ac{spat} and queue modelling, surrogate-policy or machine-learning-based state reduction for on-board real-time deployment, multi-intersection and network-level coordination, robustness to communication latency and packet loss, and extensive real-world experiments to validate and refine these DP-based eco-driving strategies.
