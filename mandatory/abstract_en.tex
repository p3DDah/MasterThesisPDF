Urban traffic congestion and its associated environmental impact represent a critical challenge for sustainable mobility. \acp{c-its}, particularly \ac{glosa} applications, offer a promising technological pathway to mitigate these issues by optimising vehicle trajectories at signalised intersections. However, a fundamental conflict exists between controllers that optimise for network-level traffic throughput and those that prioritise vehicle-level fuel efficiency. The performance trade-offs of these differing strategies, especially under varying traffic densities and \acp{mpr}, remain poorly understood.
\mynewline
This thesis systematically investigates this trade-off by developing, implementing, and rigorously evaluating two distinct \ac{glosa} controllers: a flow-optimised algorithm (\ac{flow-glosa}) designed to maximise throughput, and a novel eco-driving controller (\ac{eco-glosa}) based on a \ac{dp} framework to explicitly minimise fuel consumption. These algorithms were tested within a high-fidelity digital twin of the Stuttgart–Neckartor intersection, a real-world urban arterial, using the \ac{sumo} platform. To ensure a high degree of realism, the simulations incorporated the \ac{eidm} to capture nuanced stop-and-go behaviours, and emissions were calculated using both the HBEFA4 and the physics-based PHEMlight5 models for robust validation. A comprehensive experimental campaign was executed, sweeping traffic volumes from under-saturated ($69\,\unit{veh/h}$) to fully gridlocked ($3462\,\unit{veh/h}$) conditions and \ac{glosa} penetration rates from $0\,\%$ to $100\,\%$.
\mynewline
The results reveal a stark, density-dependent performance dichotomy. The \ac{eco-glosa} controller proved highly effective in low-to-moderate traffic, reducing \ac{co2} emissions by up to $27\,\%$ compared to uncontrolled driving by smoothing speed profiles and eliminating stops. However, in congested scenarios ($q > 2077\,\unit{veh/h}$), its egoistic optimisation paradoxically degraded network performance, inducing severe stop-and-go waves that increased emissions by over $160\,\%$ and catastrophically reduced traffic flow. In contrast, the \ac{flow-glosa} controller demonstrated exceptional robustness. In saturated conditions, it successfully dissolved traffic jams at high \ac{mpr} ($>80\,\%$), restoring free-flow speeds and thereby achieving profound emission reductions of up to $64\,\%$ by eliminating the highly inefficient stop-start cycles.
\mynewline
This research concludes that a \enquote{one-size-fits-all} \ac{glosa} strategy is not viable. In congested urban environments, optimising for traffic flow is a more effective and reliable means of achieving system-wide environmental benefits than myopic, vehicle-level eco-driving optimisation. The findings strongly advocate for the development of hybrid, adaptive \ac{glosa} systems that dynamically adjust their control objectives based on real-time traffic state estimation, transitioning from an eco-centric to a flow-centric strategy as traffic density increases.
