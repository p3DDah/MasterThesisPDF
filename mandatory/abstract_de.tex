Urbane Verkehrsüberlastung und die damit verbundenen Umweltauswirkungen stellen eine entscheidende Herausforderung für die nachhaltige Mobilität dar. Kooperative Intelligente Verkehrssysteme (C-ITS), insbesondere Anwendungen wie der Grüne-Welle-Fahrassistent (GLOSA), bieten einen vielversprechenden technologischen Ansatz zur Minderung dieser Probleme durch die Optimierung von Fahrzeugtrajektorien an lichtsignalgesteuerten Knotenpunkten. Es besteht jedoch ein fundamentaler Zielkonflikt zwischen Regelungsstrategien, die den netzwerkweiten Verkehrsdurchsatz optimieren, und solchen, die die fahrzeugindividuelle Kraftstoffeffizienz priorisieren. Die Leistungsfähigkeit und die Kompromisse dieser unterschiedlichen Strategien, insbesondere unter variierenden Verkehrsdichten und Marktdurchdringungsraten, sind bislang nur unzureichend erforscht.
\mynewline
Diese Masterarbeit untersucht diesen Zielkonflikt systematisch durch die Entwicklung, Implementierung und rigorose Bewertung von zwei distinkten GLOSA-Reglern: einem flussoptimierten Algorithmus (Flow-GLOSA), der auf die Maximierung des Durchsatzes ausgelegt ist, und einem neuartigen, verbrauchsoptimierten Regler (Eco-GLOSA), der auf einem Framework der dynamischen Programmierung basiert, um den Kraftstoffverbrauch explizit zu minimieren. Diese Algorithmen wurden in einem hoch realistischen digitalen Zwilling des Stuttgarter Neckartors, einer realen städtischen Hauptverkehrsader, unter Verwendung der Simulationsplattform Simulation of Urban Mobility (SUMO) getestet. Um ein hohes Maß an Realismus zu gewährleisten, wurde das Erweiterte Intelligente Fahrermodell (EIDM) zur Erfassung nuancierter Stop-and-Go-Verhaltensweisen implementiert und die Emissionen wurden zur robusten Validierung sowohl mit dem HBEFA4- als auch mit dem physikbasierten PHEMlight5-Modell berechnet. Es wurde eine umfassende Simulationskampagne durchgeführt, die Verkehrsaufkommen von unter- bis vollständig gesättigten Bedingungen ($69\,\unit{veh/h}$ bis $3462\,\unit{veh/h}$) sowie Durchdringungsraten des Grüne-Welle-Fahrassistenten von $0\,\%$ bis $100\,\%$ abdeckte.
\mynewline
Die Ergebnisse offenbaren eine ausgeprägte, dichteabhängige Leistungsdichotomie. Der verbrauchsoptimierte Regler erwies sich bei geringem bis moderatem Verkehrsaufkommen als äußerst effektiv und reduzierte die \ac{co2}-Emissionen durch die Glättung von Geschwindigkeitsprofilen und die Vermeidung von Stopps um bis zu $16\%$ im Vergleich zur ungeregelten Fahrt. In überlasteten Szenarien ($q > 2077\,\unit{veh/h}$) führte seine egoistische Optimierung jedoch paradoxerweise zu einer Verschlechterung der Netzwerkleistung, indem er schwere Stop-and-Go-Wellen induzierte, die die Emissionen um über $160\,\%$ erhöhten und den Verkehrsfluss katastrophal reduzierten. Im Gegensatz dazu zeigte der flussoptimierte Regler eine außergewöhnliche Robustheit. Unter gesättigten Bedingungen löste er bei hohen Marktdurchdringungsraten ($>80\,\%$) erfolgreich Staus auf, stellte freie Fließgeschwindigkeiten wieder her und erzielte dadurch tiefgreifende Emissionsminderungen von bis zu $64\,\%$ durch die Eliminierung der hochgradig ineffizienten Start-Stopp-Zyklen.
\mynewline
Die Forschung kommt zu dem Schluss, dass eine universell einsetzbare \enquote{one-size-fits-all}-Strategie für den Grüne-Welle-Fahrassistenten nicht zielführend ist. In überlasteten städtischen Umgebungen ist die Optimierung des Verkehrsflusses ein wirksameres und zuverlässigeres Mittel zur Erzielung systemweiter Umweltvorteile als eine kurzsichtige, fahrzeugzentrierte Eco-Driving-Optimierung. Die Erkenntnisse sprechen sich nachdrücklich für die Entwicklung hybrider, adaptiver GLOSA-Systeme aus, die ihre Regelungsziele auf der Grundlage von Echtzeit-Verkehrszustandsschätzungen dynamisch anpassen und mit zunehmender Verkehrsdichte von einer verbrauchsoptimierten zu einer flussorientierten Strategie übergehen.
